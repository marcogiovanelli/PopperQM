

%—the alleged action at a distance supposedly implied by the \s{Copenhagen interpretation} of \qm. 

%Any change in the measurement setup at $A$ alters the joint state, not the reduced state of $B$.

%$$ \langle \psi | \hat{y}_B - \hat{y}_A | \psi \rangle = 0 \quad \text{and} \quad \langle \psi | \hat{p}_A + \hat{p}_B | \psi \rangle = 0 $$
%are properties of the composite system’s preparation, not of either subsystem taken alone. Any change in the measurement setup at A alters the joint state (and hence the joint measurement statistics), not the pre-existing marginal state of B. The reduced density operator for B, $$ \rho_B = \mathrm{Tr}_A(|\psi\rangle\langle\psi|), $$ %remains unaffected by local operations on A, in full agreement with standard quantum mechanics and the no-signaling theorem.




%\footnote{In both cases Popper starts from an entangled sharp-position, spread-momentum state. In 1934, momentum conservation ($p_A + p_B = \const$) was used to infer, from the momentum of $A$, the momentum of $B$ without preparing its ensemble. In 1982, the relative-position correlation ($x_A-x_B=\const$) was used to argue that changing the preparation of $A$ would alter the measurement statistics of $B$ without changing its own preparation}


\todo{indirect vs. direct} \todo{specific elements of physical reality not mirrored in the theory}\todo{separation principle}


%Teconstructing the \emph{past of an electron} via an arbitrarily sharp non-predictive measurement together with a conservation law appears to allow an arbitrarily predictive measurement of the \emph{future of a light quantum}. In other terms, reconstructing the \emph{past of an electron} via an arbitrarily sharp non-predictive measurement together with a conservation law appears to allow an arbitrarily predictive measurement of the \emph{future of a light quantum}. 


%(I) Heisenberg’s formulas \cref{eq:ur} should receive a statistical interpretation as \s{spread limitations}, and (II) their interpretation as \s{accuracy limitations} is not a logical consequence of quantum mechanics. As a consequence,
%\citep{Pechenkin2002}


%Once again, if the momentum remains unchanged, then the position, although it remains unknown cannot have been disturbed; otherwise, as we have seen, some kind of action at a distance would have to be introduced. If this is the case, the path before the momentum measurement at $X_1$ can be reconstructed once the final position is determined via the Geiger counter at $X_2$.



%Contrary to Popper’s claim, in \qm\ (a) predictive measurements \emph{are} always preparations, and (b) non-predictive measurements\footnote{A measurement that reveals an outcome but does not leave behind a system in a well-defined post-measurement state} always \emph{presuppose} a prior preparation. te.

%systems do have properties before being measured, namely those
%relationally instantiated through preparation


%Against the indeterministic metaphysics he saw as prevailing, 

\todo{In classical physics, controllable and conditional selections are interchangeable. For instance, in an elastic collision where $p_A + p_B = p_0$, conditioning on $p_A = p_A’$ merely picks out those systems in which $B$ has $p_B = p_0 - p_A’$—exactly the same subensemble one would obtain by controllably filtering the events where $B$ has that momentum. In quantum physics, they are not interchangeable: conditional selection merely updates our knowledge about $B$ given a measurement outcome on A, which, however, depends on the choice of $A$’s preparation; it does not correspond to a new, controllably prepared ensemble of $B$-particles. In the classical case, if one actively filters out all collisions where the outgoing momentum of $A$ has a specific value $p_A$, one automatically selects a subensemble of events in which $B$ has the corresponding momentum $p_B$. However, in the quantum case, selecting $A$ for position would necessarily spread the momentum distribution of $B$, so that no definite $p_B$ can be assigned}

%In classical physics, conditionalization does not depend on the measurement context — it just reveals pre-existing correlations.
%	•	In quantum mechanics, conditionalization is context-dependent: the information gained about B depends on which observable of A is measured, i.e., on how A was prepared or interrogated.



\todo{it does not amount to preparing a new ensemble of B-particles with momentum p_B = p_0 - p_A’.}

\todo{check use [a_i] in Popper}
\todo{controllable selection and conditional selection}

\todo{Correlation IS enough} \todo{Both exploit the preparation ≠ correlation distinction, but in opposite directions} \todo{Uses correlation to claim knowledge without preparation}

\todo{controllable selection and conditional selection. But it's actually conditional selection - you're just sorting B particles based on random A outcomes, revealing pre-existing correlations}.

\todo{1. Controllable (or causal) selection – we actively prepare a subensemble by physically filtering systems according to some criterion. 2. Conditional (or statistical) selection – we passively define a subensemble after the fact, by conditioning on the value of a random variable obtained in a measurement.}


%it always assigns one relative to a given state

%no measurement outcome can be defined independently of a prior quantum state.

%\s{predictive measurements}, in the sense assumed by Popper, are not allowed in \qm.
 

%\todo{Preparation and measurement must be distinguished to be disting, but cannot be uncauped: predicitive selectis non-predicitge meaurere, as a predictive measruments!}



%—a principle that, despite its foundational importance, remains underemphasized in the literature.





%Popper's \TE has the unintended effect of highlighting the centrality of the \CH, which is rarely emphasized explicitly.\todo{prove}

%This is the moral of this episode in quantum mechanics it is not possible to perform a measurement without a previous preparation. 

%n particular, he apparently continues to maintain that the \CH was unwarranted, and that physical selections---that is, in modern terms, preparations---and predictive measurements can be \s{uncoupled}.


%A prepared state has no empirical content unless it is connected to a class of possible measurements, and vice versa. \todo{This what we can learn from Popper gross mistake}. \todo{Gunther} 


\todo{reduced density matrix}

\todo{No-signaling theorems}

%If we repeat the same preparation many times, the probability of a given outcome is its relative frequency,

%

%Eine Pseudoeigenschaft kann aber nicht unabhiingig von Priipariervedahren und Registriermethoden, d. h. nicht unab- hiingig von den a E Q' und den b o E no defi.niert werden, was eben den nicht objektiven Charakter der Pseudoeigenschaften zum Ausdruck bringt.


% is no meaning to the notion of a system’s property independent of this link

%Classical physics is a theory about measured states the measurement simply reveals a property that already existed in the prepared system; quantum mechanics is a theory about transition probabilities between prepared and measured states.

%the measurement simply reveals a property that already existed in the prepared system.


%he meaning of \s{state} (preparation) and “observable” (measurement) are defined only relative to each other.

%In quantum mechanics, the theory predicts probabilities of measurement outcomes conditional on a preparation procedure:

%Preparation establishes the physical state. Measurement reveals aspects of that state.  \todo{check older version}

%where measurement are supposed to reveal pre-existing definite values.



%Popper abaodn the ambito to preic, bowver, that one can use a correlation to bypass a preparation


%In both cases, however, the correlations are features of the joint preparation and cannot be used to circumvent the preparation–measurement coupling for the subsystems.


%Predict properties of B without preparing it Change B’s statistics without changing its preparation




%Popper’s 1934 ETP-like \TE used momentum conservation to show \qm allows for sharp predictive position and momentum measurement of an individual $B$ particle without preparing a corresponding super-pure ensemble of $B$ particles. Popper’s 1980s EPR-like \TE used relative position to show that  change $B$’s statistics without changing B preparation


%As we have seen, Popper’s 1934 ETP-like \TE used momentum conservation to show that \qm allows for sharp predictive position and momentum measurement of an individual $B$ particle without preparing a corresponding super-pure ensemble of $B$ particles. Popper’s 1982 EPR-like \TE used relative position to show that \qm implies that a change the statistics of the $B$ ensemble could be produced without changing $B$’s preparation. 

%Popper uses momentum conservation in a correlated AB-system to claim that one can predictively select a subensemble of B-particles based on the momentum of A, without preparing B anew.

%Predict properties of B without preparing it, Change B’s statistics without changing B preparation



%	Thus, he exploits the momentum correlation to avoid physically preparing a new ensemble of B-particles, claiming that this does not violate quantum mechanics.

%	1934: uses correlation to replace preparation,
%	•	1982: uses correlation to show influence despite unchanged preparation.

%s to show that changing the preparation of A affects the measurement statistics of B even when B’s preparation remains unchanged.


%and cannot be used to prepare individual subsystems independently.

%Crucially, correlations are features encoded in the joint preparation of the composite system and provide information about one subsystem conditioned on measurements of another, but they do not constitute a preparation of either individual subsystem and cannot be used to control or prepare individual subsystems independently.

%A relational constraint in the joint $AB$ state (such as momentum conservation or relative position) that links measurement outcomes on $A$ and $B$, allowing conditional inference but not substituting for actual preparation of individual subsystems.

%\todo{correlation}

%Use correlations to gain predictive power without proper preparation (1934)
%Use correlations to change statistics without changing preparation (1982)
%Use correlations to circumvent the preparation-measurement link


%he again tried to circumvent the preparation-measurement coupling by appealing to a non-statistical correlation---relative position instead of momentum conservation. Starting again from entangled state with vanishing relative position, Popper did forgo any attempt to predict the position and momentum of a single $B$ particle. However, he argues that the theory predicts an increase in the momentum spread of an ensemble of $B$ particles, not as a result of any change in their prepared state, but as a conditional effect of the sharp position selection applied to the corresponding $A$ particles. The \TE\ ultimately failed: a correlation cannot replace and actual preparation.

%Even when conditioning on distant correlated measurements, statistics must respect the actual preparation.

%In 1934 he thought thinking correlation does not substitute preparation, hence,   in 1980 that correlation does substitute for preparation, 


%He thinks allows him to predict statistics for B
% 
%He thinks conditioning on $A$'s position measurement allows him to predict statistics for B
 
 


%conditional measurement statistics cannot be divorced from the actual physical preparation

\footnoteh{Popper's setup is nothing but a special case of the EPR paper example, where particles are in a state with perfectly equal positions: $\psi\left(x_A, x_B\right)=\delta\left(x_A-x_B\right)$. As in EPR, Fourier-transforming to the momentum representation gives $\tilde{\psi}\left(p_A, p_B\right)=\frac{1}{2 \pi \hbar} \int d x_A d x_B\, \delta(x_A - x_B)\, e^{-\frac{i}{\hbar}\left(p_A x_A+p_B x_B\right)} = \delta(p_A + p_B)$. The total momentum is sharp $\left(p_A+p_B=0\right)$, but each particle's momentum is completely undetermined from the outset. When one conditions on $A$'s position using a slid, one is selecting a sub-ensemble of $B$ particles with correlated positions with $A$ that were already part of the maximally spread distribution. You now look at conditional statistics: "momentum distribution of B particles whose partner A passed through the slit". This conditional distribution must still be consistent with B's original preparation (maximal momentum spread). You cannot get conditional statistics that would require B to have been prepared differently: }

%You can learn about different correlations by conditioning, but you cannot escape the statistical constraints imposed by the actual physical preparation. Measurement and conditioning operations on A cannot make B's statistics inconsistent with how B was actually prepare

%Thus, there is no increase relative to $B$'s actual preparation\citep[166]{Redhead1995}}.

%
%
%\footnoteh{Popper did forgo any attempt to predict the position and momentum of a single $B$ particle. However, he still sought to predict an increase in the momentum spread of an ensemble of $B$ particles, not as a result of any change in their prepared state, but as a conditional effect of the sharp position selection applied to the corresponding $A$ particles} and once again ultimately failed%
%
\footnoteh{Popper's setup is nothing but a special case of the EPR paper example, where particles are in a state with perfectly equal positions: $\psi\left(x_A, x_B\right)=\delta\left(x_A-x_B\right)$. As in EPR, Fourier-transforming to the momentum representation gives $\tilde{\psi}\left(p_A, p_B\right)=\frac{1}{2 \pi \hbar} \int d x_A d x_B\, \delta(x_A - x_B)\, e^{-\frac{i}{\hbar}\left(p_A x_A+p_B x_B\right)} = \delta(p_A + p_B)$. The total momentum is sharp $\left(p_A+p_B=0\right)$, but each particle's momentum is completely undetermined from the outset. When one conditions on $A$'s position using a slid, one is selecting a sub-ensemble of $B$ particles with correlated positions with $A$ that were already part of the maximally spread distribution. Thus, there is no increase relative to $B$'s actual preparation\citep[166]{Redhead1995}}. 

\todo{that conditional ensemble statistics can be divorced from the actual physical preparation of those particles}
\todo{Popper once again attempted to divorce the conditional ensemble statistics can be divorced from the actual physical preparation of those particles }

\todo{that conditional ensemble statistics can be divorced from the actual physical preparation of those particles}

\todo{He wants the conditional spread to reflect only the correlation (sharp relative position) without reflecting B's actual prepared state (maximally spread momentum)}

%
%%- one is just revealing different conditional statistics from the same underlying preparation. Popper is indeed here confusing a a conditional selection with a proper with the controllable selection of a sub-ensemble by predictive selection
%
%% Popper is \q{confusing the selection of a sub-ensemble by conditionalising on a random variable, with the controllable selection of a sub-ensemble by predictive selection}.
%
%
%
%%  again Popper sought to bypass the preparation–measurement coupling by appealing to a non-statistical correlation---this time, to relative position rather than to momentum conservation\footnote{ His goal was predicting that the ensemble of B particles, when conditioned on $A$ particle passing through a narrow slit, will show increased momentum spread}.
%
%%Popper thinks conditioning on a non-statistical correlation (relative position) should allow him to predict ensemble statistics that bypass the preparation-measurement link. But it doesn't - the ensemble statistics of $B$ still reflect how $B$ was actually prepared, in the momentun 
%
%
%
%%However, once again Popper tried to bypass the preparataiton-measuremt link, to predict a spread of momentum measurement, without a previous position preparation.
%
%%Popper's error: He seems to think that the second preparation of A automatically constitutes a second preparation of B (sharpening B's position), while somehow B should retain its original momentum properties from the first preparation.
%
%%However, once again Popper confus the between \s{preparation} that is inadaqure. That measuriemgnt, that condution physical selection and merey epogelmogica
%
%%you're just selecting/revealing a subset of that existing spread.
%
%%'s reduced density matrix (tracing over A) remains unchanged by measurements on A alone
%%What changes is the conditional state - your description of B given what you learned from A
%%The increased conditional momentum spread reflects epistemic updating, not physical spreading
%
%\todo{never entered my head}
%
%
%%\footnoteh{One might argue indeed, the non-falsification of energy principle acts as its corroboration} 
%
%\todo{form of non-existence statements, thus that one single case the prohibition by asserting the existence of a thing (or the occurrence of an event) ruled
%out by the law, then the law is refuted. The problem is here what happen when we consistalty fail to find such a device.}.
%
%%Popper warned apologetic use of imaginary experiments, however, whether is critical use of imaginary experiments.
%
%%
%
%%
%
%%Particle has position and momentu, but we can only make statistical predicitios about them
%
%
%
%%demonstration that a particle can have, or rather has, both a sharp position and a sharp momentum. 
%
%%Once again Popper attempted to \s{outsmart} the \IR, by resorting to a non-statistical correlation\todo{check}, to refute undermined  his \s{formulae are applicable to all kinds of indirect measurements}.
%
%%so once y_A and p_A are “known”, the same values for B are thereby fixed (hence, in effect, “predictable”).
%
%%However, he implicitly assumed what amounts to a predictive relation --- namely, that by measuring or localizing particle A, one can determine (or “know”) the corresponding position and momentum of B
%
%%in principle, know both position and momentum of $B$ simultaneously---without ever interacting with it. %By narrowing slit A we can determine the position of photon A, and thereby also that of photon B; at the same time, by measuring the momentum of A, we can infer that of B. Thus, we can predict both position and momentum of B, without in any way disturbing it.
%
%%“Popper believed that by measuring particle A one could predict both the position and the momentum of particle B without any disturbance.”
%
%%[7, Vol.3,Ch.2,pp.2-3]
%
%%he emphasizes that while predictions of future events (such the trajectory of an individual particle) are only statistical with statistics limited by the Uncertainty Principle, that precise retrodiction of past events (the positions and momenta of the particle along its trajectory) are not only possible - “they are tests of the theory” [1, p.63,last line].
%
%%https://arxiv.org/pdf/quant-ph/0507009
%
%%all all dis ill bougee des. Because of my own involvement with a similar---although, unfortunately, mistaken---argument, I think it fair to say that I was one of the first philosophers to realize the significance of EPR , and Thave always been very interested in it. Among physicists it created quite a
% 
%
%%Popper, at the height of his fame, was taken much more seriously than in the 1930s \citep{DelSanto2017}. However, his attempt must also ultimately be considered a \s{gross mistake} \citepp{Ghirardi1988}{Redhead1995}.
%
%%Thus the so-called 'paradox' of Einstein, Podolsky, and Rosen (see n. 14 above) is not a paradox but a valid argument, for it • established just this: that we must ascribe to particles a precise position and momentum, which was denied by Bohr and his school/ (though it is admitted by Bopp).
%
%%First I would repeat that the predictions of the theory are statistical, with a scatter given by the Heisenberg formulae. The measurements which must be more precise than the scatter (as I have pointed out) may serve as tests of these predictions: these measurements are retrodiction
%
%%that we cannot prepare experiments
%
%
%\todo{reight selection and preparation, however, does nt unde ast ppreapta, alos that measurement anre coupled, cannot be uncpupled  }



%
%
%%My own imaginary experiment of section 77 was largely based on this asymmetry in Heisenberg’s experiment. (Cf. note *1 to appendix vi.) Yet my experiment is invalid just because the asymmetry invalidates Heisenberg’s whole discussion of measurement: only measurements resulting from physical selection (as I call it) can be used to illustrate Heisenberg’s formulae, and a physical selection, as I quite correctly pointed out in the book, must always satisfy the ‘scatter relations’. (Physical selection does disturb the system.)
%
%%That the inadequacy of Heisenberg’s argument has remained unnoticed for so long is no doubt due to the fact that the indeterminacy formulae clearly follow from the formalism of the quantum theory (the wave equation), and that the symmetry between position (q) and momentum (p) is also implicit in this formalism.
%
%
%\todo{May of the obecjtions, that about preparation.}
%\todo{The reason is that Heisenberg’s discussion fails to establish that measurements of position and of momentum are symmetrical;}.
%
%
%%---a \s{pure case} described by the same plane-wave function~$\psi$
%
%
%%In Popper's setup, after the collision at $S$, the $A$-particles possess different momenta; one can select a sub-ensemble of particles with the same sharp momentum by applying, for example, an accelerating potential. However, in \qm\ such a selection does not assign a definite momentum to any \emph{individual} particle; it merely defines a new, anonymous \s{pure} ensemble of systems. It therefore makes no sense to claim, as Popper suggests, that the \emph{same} electron that passes through the filter already possessed the \emph{same} momentum beforehand. Consequently, the assertion that the position of that \emph{same} electron, although unknown, remains unchanged, cannot be upheld within the quantum-mechanical formalism. It is particularly ironic that Popper appeals precisely to this \s{anonymity} argument when discussing position measurement. Since a slit-selection does change the momentum, producing diffraction, even if we filter again to select a sub-ensemble with sharp momentum, we cannot tell whether we are dealing with the same particle as before.
%
%\todo{reveal vs. prepare}
%
%%filter defines the conditions for a new ensemble characterized by a sharp momentum distribution, but it does not individualize any specific particle as having that momentum
%


%Teconstructing the \emph{past of an electron} via an arbitrarily sharp non-predictive measurement together with a conservation law appears to allow an arbitrarily predictive measurement of the \emph{future of a light quantum}. In other terms, reconstructing the \emph{past of an electron} via an arbitrarily sharp non-predictive measurement together with a conservation law appears to allow an arbitrarily predictive measurement of the \emph{future of a light quantum}. 


%Popper moves from a statistical correlation, constraining the joint probabilities for $p_B$ given the probability for $p_A$, to a physical claim about and individual particle $B$ actually having that definite $p_B$

%In other terms, reconstructing the \emph{past of an electron} via an arbitrarily sharp non-predictive measurement together with a conservation law appears to allow an arbitrarily predictive measurement of the \emph{future of a light quantum}. 

%In other terms, one can infer something about $B$ once one measures $A$, but one cannot control or reproduce that same condition for $B$ in the next trial without measuring $A$ again: 



% Es ist zwar möglich, den Impuls vor und nach der Messung genau zu kennen, aber eine Impulsmessung nimmt, je genauer sie ist, umso längere Zeit in Anspruch, und man weiss dann nicht, wann innerhalb dieses Zeitintervall[s] der Stoss zwischen Messapparat und Messobjekt stattgefunden hat, d. h. wie lange das Messobjekt den „Impuls vor dem Stoss“ und wie lange es den „Impuls nach dem Stoss“ hatte; dies genügt gerade, um eine Ortskenntnis nach dem Stoss, die man durch den Messprozess hindurch bis in die Zeit vor dem Stoss zurückverfolgen will, hinreichend zu verwischen.

%Diese Bahn ist jedoeh prinzipiell unkon-
%trolIierbar. Sie gilt n~imlich nur ffir das Zeitintervali zwischen
%dem Ende der Impulsmessung und denl Beginn der Orts-
%messung, in dem das Teilchen iiberhaupt keine Weehsel-
%wirkung mit seiner Umgebung hat, und EiBt sich nieht in
%den Zeitraum vet der Impulsmessung fortsetzen, da diese
%ihrerseits die Kenntnis des Orts gem/ifl der Ungenauigkeits-
%relation zerstSrt.
%Where, in this paragraph, I speak of ‘an aggregate of particles’ I should now speak of ‘an aggregate—or of a sequence—of repetitions of an experiment undertaken with one particle (or one system of particles)’. Similarly, in the following paragraphs; for example, the ‘ray’ of particles should be re-interpreted as consisting of repeated experi- ments with (one or a few) particles—selected by screening off, or by shutting out, particles which are not wanted.


%As Popper emphasizes, as not often been observed that \q{DaB der mathematischen Ableitbarkeit der HEISENBERG-Formeln aus den grundlegenden quantenmechanischen Gleichun-gen auch eine Ableitbarkeit der Interpretation jener Formeln aus der Interpretation dieser Grundgleichungen genau entsprechen muB}.

%\footnote{Indeed, \citet{Heisenberg1927a} derived the \IR from $S(q, p)=\frac{1}{\sqrt{2 \pi \hbar}} e^{i p q / \hbar}$, where the transformation function $S(q,p)$ expresses the amplitude that a state with definite momentum $p$ has the coordinate $q$}

\todo{Freire, O., Jr. (2004). Popper, Probabilidade e Teoria Qu antica. Episteme (Porto Alegre), 18, 103e127.}


%To do something more, I have also considered your suggestion of using a light filter. I imagine the experiment, in order to make it as physically simple as possible, as follows: In this way, only the light of the resonance line is reflected, while light of other frequencies passes through. Basically, this experiment is nothing other than the well-known method of residual rays. Now, the accuracy of the momentum measurement of the light quantum is evidently determined by the width of the resonance line. It holds that 
%\[
%\Delta p = \frac{h \Delta r}{e} \, ,
%\]
%and, on the other hand, the probable residence time of the light quantum in the reflecting atom is given by the lifetime...


%In response to Popper’s suggestion that one might use a light filter to measure the momentum of a photon with arbitrary precision, Heisenberg proposed a simple physical model based on Wood’s resonance–reflection experiments. He imagined a beam of inhomogeneous light reflected from a vapor, such as sodium or mercury vapor, which acts as a resonance filter: it reflects only photons whose frequency coincides with an atomic resonance line, while light of other frequencies passes through. 

%Heisenberg pointed out that the accuracy of the photon’s momentum measurement is limited by the width of the resonance line~$\Delta \nu$. The narrower the line, the more precisely one knows the photon’s frequency and thus its momentum, given by $\Delta p = h \Delta \nu / c$. Yet, the finite width of the line arises from the finite lifetime of the excited atomic state, $\Delta t \approx 1 / \Delta \nu$. During this time the photon is absorbed and re-emitted, so the instant and position of reflection cannot be sharply defined. Even if the photon’s position is known exactly at a later time, its position at the moment of reflection remains uncertain by $\Delta x = c \Delta t = c / \Delta \nu$. Combining these relations yields, at least heuristically, $\Delta x \, \Delta p \approx h$, that is, the Heisenberg uncertainty relation. 



%Each angle of scattering corresponds to a different momentum component $p_y$. The narrower the slit (smaller \Delta y), the wider the diffraction pattern (larger \Delta p_y). 



% predictive (or preparative) selection: Once one repeatedly fails to construct such a perpetual motion machine, one can either (a) give up and accept energy conservation or (b) devise a different setup that would ultimately violate it. Popper clearly adopted the second attitude.  %The argument developed by Popper 1 is invalid not because of some unjustified assumption about the source but, rather, because the universally accepted rules of quantum mechanics, even in the Copenhagen interpretation, do not entail what Popper claims them to entail.


%The term is due to Weyl (Zeitschrift fur Physik 46, 1927, p. 1) and J. von Neumann
%(Göttinger Nachrichten, 1927, p. 245). If, following Weyl (Gruppentheorie und Quanten-
%mechanik, p. 70; English translation p. 79; cf. also Born-Jordan, Elementare Quanten-mechanik,
%p. 315), we characterize the pure case as one ‘. . . which it is impossible to produce
%by a combination of two statistical collections different from it’, then pure cases
%satisfying this description need not be pure momentum or place selections. They could
%be produced, for example, if a place-selection were effected with some chosen degree
%of precision, and the momentum with the greatest precision still attainable.

%Popper asked \Weis whether an ensemble of type~(1b) could be called a \s{\emph{pure case}}. He replied that this was not the common parlance---a pure case having only one variable sharp---, but that no specific term existed for the case in which both momentum and position are sharp. For this reason, Popper coined the expression \s{\emph{super-pure case}}, which he used thereafter to label what von~\citet{Neumann1932} called a homogeneous dispersion-free ensemble .

%Weisskopf suggested that Popper might cite him in a note as having had discussions with him, if such a form seemed appropriate. Otherwise, Popper was, of course, free to use any of the arguments they had discussed. This is indeed, what Popper did referring quasi-verbatim to this letter.


%However, the in both that put in seec put in eignestate of, this and viced of the coorsp, that both are inc ...  thp distrn the momentum, and thus we cild not indetied in the beam.

%\footnote{The asymmetry lies on the fact that position is not an eigenstate of the Hamiltonian operator of a free particle, whereas momentum is. For this reason, momentum selection does not lead to any analogous \s{disturbance} as in the case of the position measurement, seems merely to let the particles go through the filter with unchanged momentum. However, the point that Popper missed is that momentum selection still projects the system into a momentum eigenstate; thus, the position becomes completely undetermined, since position is not an eigenstate of the momentum operator}

%Once again, Popper’s notion of \s{physical selection} is still not on par with the modern notion of \s{preparation}

%\todo{}


%\footnote{The asymmetry between the position and momentum filter is only apparent. It lies on the fact that position is not an eigenstate of the Hamiltonian operator of a free particle, whereas momentum is. For this reason, momentum selection does not lead to any analogous \s{disturbance} as in the case of the position measurement, seems merely to let the particles go through the filter with unchanged momentum. However, the point that Popper missed is that momentum selection still projects the system into a momentum eigenstate; thus, the position becomes completely undetermined, since position is not an eigenstate of the momentum operator. Once again, Popper’s notion of \s{physical selection} is still not on par with the modern notion of \s{preparation}}

%Heisenberg—who speaks of measuring or observing rather than of selecting—presents the
%situation, in form of a description of an imaginary experiment, as follows: if we wish to
%observe the position of the electron, we must use high frequency light which will strongly
%interact with it, and thus disturb its momentum. If we wish to observe its momentum,
%then we must use low frequency light which does leave its momentum (practically)
%unchanged, but cannot help us to determine its position. It is important that in this
%discussion the uncertainty of the momentum is due to disturbance, while the uncertainty of the position is not
%due to anything of the sort. Rather it is the result of avoiding any strong disturbance of the
%system. (See appendix *xi, point 9.)
%My old argument (which was based upon this observation) proceeded now as follows.
%Since a determination of the momentum leaves the momentum unchanged because it
%interacts weakly with the system, it must also leave its position unchanged, although it
%fails to disclose this position. But the undisclosed position may later be disclosed by a
%second measurement; and since the first measurement left the state of the electron
%(practically) unchanged, we can calculate the past of the electron not only between the two measurements, but also before the first measurement. his argument. (In other words, I still believe that my argument and my experiment of section 77 can be used to point out an inconsistency in Heisenberg’s discussion of the observation of an electron.) But I now believe that I was wrong in assuming that what holds for Heisenberg’s imaginary ‘observations’ or ‘measurements’ would also hold for my ‘selections’. As Einstein shows (in appendix *xii), it does not hold for a filter acting upon a light quantum. Nor does it hold for the electric field perpendicular to the direction of a beam of electrons, mentioned (like the filter) in the first paragraph of the present appen- dix. For the width of the beam must be considerable if the electrons are to move parallel to the x-axis, and as a consequence, their position before their entry into the field cannot be calculated with precision after they have been deflected by the field. This invalidates the argument of this appendix and the next, and of section 77.

%Hence, narrowing the frequency band $\Delta \nu$—that is, making the filter more selective in momentum—should, according to Popper, \emph{reduce} the spatial smearing rather than increase it! This is exactly the opposite of what the \IR\ implies. Popper can however counter, that the \IR\ expressed merely statistical relations limiting the homogeneity of an ensemble of light quanta, not constraints on individual ones. 



%Ein Impulsaussonderungsapparat ohne Ortsverschmierung würde
%also der Quantenmechanik nicht widersprechen, wenn er vom Typus (II} ist
%
%To construct a filter of type II, Popper suggests using colored glass filters followed by gas layers. Colored glass filters allow a relatively broad band of wavelengths to pass while blocking the others. They do not sharply distinguish between single frequencies, but rather transmit a whole range of nearby wavelengths, thereby acting as coarse selectors. Gas layers, by contrast, work according to a much sharper principle: they absorb light quanta whose energies coincide with their resonance frequencies, while letting light quanta with off-resonance energies pass essentially unchanged. In this way, two distinct outcomes arise. Photons that are on resonance are absorbed and subsequently re-emitted, which blurs their direction and timing. Photons that are off resonance, on the other hand, traverse the gas layer without disturbance, preserving their original properties. Momentum is unchanged, position is unknown, unless one admits action at a distance.\todo{circumvent Heisenberg's objection}


%In Popper's view case (b) stands out, a positon disrites rpiodece a diffrancatio, a momentum selection to aviod such distruncas (say a velocity fiter) simply go trhigh with the same momentum unchanged.\todo{check}

%L The free flight distance

%The change in the light quantum’s frequency or wavelength (due to the Doppler or Compton effect) tells you about the electron’s velocity (and hence its momentum).


% with arbitrary precision, possibly to address a common objection

%E.g. The momentum $p$ of an electron can be obtained by measuring position $q_0$ and $q_1$ at different times. If the time intervals between such position measurements are chosen to be very large, the momentum of the particle before the last position measurement can be determined with arbitrary accuracy . However, the momentum after the second position measurement, which alone is of physical interest, will, however, not be known with such precision, since the posisition measuremet alterns the the momentum again



%\footnote{A full account of the \vW--Hermann correspondence about Popper's \TE would require a separate study. It is interesting that that Hermann seems to ultimately got at the core of the issue, challanging the very idea the \q{physical reality of the calculation of past trajectories}. Hermann challenged Heisenberg’s concession that the reality of past processes is merely a \emph{matter of taste} \citep[63**]{Hermann1935}. According to Hermann, the reality of past trajectories is, so to speak, a \emph{matter of context}. In Popper’s parlance, each sequence of a selection and a measurement defines a complete and self-consistent \s{history}. Yet different histories cannot be combined into a single description of the same individual system. There is no context-independent \s{path of the electron} that can be reconstructed even \emph{post facto}}. \label{sec:einstein}. 

%More significant, that, at around that time, the \PZ asked \vW to review \LdF. After his refusal to avoid a conflict of interest after public dispute, \vW suggested as substitute Grete Hermann, who had spent five or six months in 1934 in Leipzig \letterp{Hermann}{\vW}{5}{3}{1935}. At her request of clarification, \vW\ recounted 


% Popper’s assumption that the filter could transmit sharply defined momenta without affecting temporal resolution therefore contradicted the very physical principles of wave propagation and the empirical behavior of real optical media.


%Heisenerg, rejecting the idea of transparent filter, could not work: Both reflection and transmission filtering obey the same uncertainty constraints\footnote{??} In the center of a transmission band, light quanta interact with the medium’s electrons elastically, i.e. without real absorption. Thus, the narrower the transparent band (the closer the band heads), the more likely light quanta are to linger in atoms — i.e. the longer the interaction time. Photons near the edges of the transparent region (the band heads) are actually absorbed with significant probability and re-emitted after a delay roughly equal to the lifetime of the excited state. This means that the narrower the transparent region, the more light quanta fall near the edges and thus the longer the average delay (interaction time).


%He proposed instead a setup that does not absorb light quanta but merely reflects or deflects them according to their momentum — a kind of “reflection filter.”
%
%For instance:
%	•	A diffraction grating, or
%	•	A mirror with a frequency- or angle-dependent reflectivity.
%
%The idea was: if one uses a reflection (instead of an absorbing) filter, then the light quanta are not destroyed and re-emitted; they are simply reflected, supposedly without random delay.
%Popper hoped this would allow a more precise, non-disturbing momentum measurement.

%	•	Narrower frequency window → longer interaction time → worse time resolution.
%	•	Wider window → shorter interaction → worse frequency resolution.
%
%Therefore, no optical filter of any kind (reflective or absorptive) can evade the uncertainty relations.


%On March 19 1935. If a transparete filted, does not work  material transparent only between two absorption lines acts like a filter that passes light quanta of certain energies. The precision with which the photon’s momentum (or energy) can be measured depends on the width and sharpness of those bands.

%This follows from the fact that a momentum measurement collapses the system into a momentum eigenstate, in which position is completely spread out, whereas a position measurement collapses it into a position eigenstate, in which momentum is indeterminate. Each measurement thus defines its own contextual \s{history}, precluding a unified description of the particle’s past trajectory.

%\citep{Hermann1935,Hermann1935a} is closely related to Popper’s statistical interpretation. 

%\footnoteh{A comparison with the EPR shows the reason why. After an elastic scattering (e.g., Compton-type) interaction, the electron and light quantum are in an entangled state. Just like in the EPR paper example, they share the same position $\delta(x_e - x_\text{ph})$  (the point of scattering), but neither has a definite momentum by itself; only the total momentum $\delta(p_e + p_\text{ph})$ is well defined (momentum conservation). Now suppose a momentum \s{selection} is performed on the electron at $X_1$ as in in Popper's setup. The electron collapses into a momentum eigenstate and therefore, as Popper now recognizes, has a completely spread position. Because of momentum conservation the light quantum simultaneously collapses into the corresponding momentum eigenstate. However, for this reason its position becomes completely uncertain. As a consequence, there is no way to exploit momentum conservation to construct a \s{predictive measurement} that yields both sharp $p$ and $p$ for the light quantum}


%Einstein, Podolsky, and Rosen use a weaker but valid argument: let Heisenberg’s inter-pretation be correct, so that we can only measure at will either the position or the
%momentum of the first particle at X. 


%Pure at the ensemble level; mixed at the individual level.

\todo{mixed stat}

%For this reason, one can reconstruct the path of an individual object only between the momentum selection and the position measurement.


%We can easily attribute to the objects that get through the slit the $\psi$ function of the pure case, but we cannot say which object, that is, which variable is the argument of this $\psi$ function. Without a supplementary check by an observer,

%\cop{} , but not before.

%Popper notion of betwene emaure physocal selection 


%\footnote{Single-particle interpretation: Einstein focused on the isolated-particle case. Predictive measurement = preparation, even for one system. But no apparatus (shutter + filter) can yield both p and q sharper than the UR; filtering momentum always smears position. Here Einstein affirms the measurement/preparation identity already at the single-particle level, directly rejecting Popper’s attempt to decouple them.}

%Einstein expressed dislike for the then fashionable \s{positivistic} emphasis on observables, which he dismissed as trivial. He maintained that in the atomic domain precise predictions were unattainable, and that theories could not simply be constructed from observational data but had to be invented. 

%\q{[S]ome of the precise predictions which I can obtain for the system $B$ (according to the freely chosen way of measuring $A$) may well be related to each other in the same way as are measurements of momentum and of position} \letterp{Einstein}{Popper}{11}{9}{1935}, that is, they

%If one interprets n the sense that they declare the production (or existence) of a collection of particles more homogeneous than a pure case to be incompatible with quantum mechanics—then their validity is not affected by the proposed experiment.

repeating Popper’s experiment many times yields a statistical distribution of $AB$-pairs with correlated by \emph{different} momentum and position each time, not an ensemble of $B$-particles all having the \emph{same} position and momentum. \todo{improve} 

%Thus, after $S$, electrons are in a statistical mixture of many momenta. 

%After the collision, the momenta of the electrons no longer form a single definite value but a mixture, since different electrons acquire different momenta depending on the corresponding changes in the light quanta. 

%More precisely, Popper considered \begin{inparaenum}[(a)] \item a non-predictive measurement on particle~$A$ (momentum followed by position), \item the application of the conservation of momentum, and \item a predictive measurement on a second particle~$B$, apparently sharper than the uncertainty relation allows. \end{inparaenum}



%  are made on identically prepared electrons, the values of $q$ and $p$ would, across the ensemble, spread in accordance with the uncertainty principle. 

%A precise measurement of position may be accompanied by the impossibility of a simultaneous velocity measurement, but this reflects only the statistical dispersion, not an absolute prohibition.\todo{improve and see margenau} 


%In Popper's reading the collapse of the wave function is nothing but the selection of a subensble, that is super-luminal velocity, as I explained in section 75 of L.Sc.D. ; for it simply is not a physical event-it is the result of the free choice of new initial conditions (or of using a new $b$ in ' $p(a, b)^{\prime}$ .Such proofs apply formal probability statements to single particles, whereas these must be expressed statistically, referring to ensembles of particles. 
%That there is no collapse of the wave function, but only the selection of subensbel, so that we have, if measu we select a sub-ensble in which precise momentu. In this way, hoever, that that must have no precise position.


%To verify a preparation uncertainty relation, one has to perform separate exper- iments for every state and every relevant observable (e.g., P and Q) to record the distribution. No single particle is thereby subjected to both a position and a momentum measurement. On the other hand, this is essential for any precision-disturbance trade-off.

%Classically, particle ensembles are always in a totally incoherent mixed state. 
%as the comparison with the EPR situation shows.
% (e.g., Compton-type) interaction 

\todo{change p and q to x and $p$}

%A comparison with the EPR situation described by Einstein in 1935 shows the reason why. After the elastic scattering at $S$, particles $A$ and $B$ are in an entangled state. Just like in the EPR example, they share the same position $\delta(x_A - x_B)$ (the point of scattering), but neither has a definite momentum by itself; only the total momentum $\delta(p_A + p_B)$ is well defined (momentum conservation). Now suppose a momentum \s{selection} is performed on $A$ at $X_1$, as in Popper's setup. The $A$ particle collapses into a momentum eigenstate $p_A$ and therefore, as Popper now recognizes, has a completely spread position. Because of momentum conservation, the $B$ particle simultaneously collapses into the corresponding momentum eigenstate $p_B$. However, for this reason, its position becomes completely uncertain. As a consequence, there is no way to exploit momentum conservation to construct a \s{predictive measurement} that yields both sharp $x$ and $p$ for the light quantum.



%However, one might concede to Popper that repeating the experiment many times would yield a statistical distribution of $AB$ particle pairs all with \emph{different}, but correlated position and momenta, but not an ensemble of light quanta all having the \emph{same} position and momentum. One can infer something about $B$ once you measure $A$, but one cannot control or reproduce that same condition on $B$ in the next trial without measuring $A$ again. The problem lies, elsewhere. Popper must have soon realized that the tricks works only if the momentum of $A$ is not changed by the selection at $X_1$, so that entire path of $A$ can be reconstructed up to the collision at $S$.

%He fails to realize that one could, in principle, lower the beam intensity to the point where particles pass one at a time, even without using the shutter \citep[25]{BacciagaluppiCrull2024}. 


%Popper adds, however, an interesting remark. Many \TE{}s aiming to challenge this conclusion consider single \emph{isolated} particles. In such cases, by repeating the experiment many times—each with arbitrarily sharp values of position and momentum—one could, in principle, construct a dispersion-free ensemble. However, in \qm\ experiments one always considers single particles embedded in a \emph{beam}. In this case, even an exceedingly precise determination of the position or momentum of one particle could not be extended to \emph{all} particles in the beam. Thus, repeating the experiment many times would not produce a super-pure case. Popper then insists that his own \TE\ should also be thought of as \q{nur innerhalb eines Teilchenschwarms}. Thus, even if the experiment allows for measuring arbitrarily sharp positions and momenta in a beam, it cannot be used to construct a super-pure case.

%Popper clearly confuses a \emph{beam} of particles with a statistical ensemble of repetitions of the same experiment. Indeed, with a sufficiently weak beam one could, in \qm, legitimately regard it as consisting of isolated particles. However, Popper might have been \s{right for the wrong reasons}. Repeating Popper’s experiment many times yields a statistical distribution of $AB$-pairs, each with different correlated momenta and positions, not an ensemble of $B$-particles all having the \emph{same} position and momentum. One can infer something about $B$ once $A$ is measured, but one cannot control or reproduce that same condition for $B$ in the next trial without measuring $A$ again. The problem is, once again, that in order to predict the position and momentum of the $B$-particle in each run of the experiment, one would need to reconstruct the position-and-momentum state of each corresponding $A$-particle \emph{before} the momentum measurement at $X_1$. In Popper's own words:

% However, I think that Popper might have been \s{right for the wrong reasons}. Repeating Popper’s experiment many times yields a statistical distribution of $AB$-pairs with correlated by \emph{different} momentum and position each time, not an ensemble of $B$-particles all having the \emph{same} position and momentum. \todo{improve} Repeating the experiment many times would yield a statistical distribution of $AB$ particle pairs all with \emph{different}, but correlated position and momenta, but not an ensemble of $B$ particles all having the \emph{same} position and momentum. One can infer something about $B$ once you measure $A$, but one cannot control or reproduce that same condition on $B$ in the next trial without measuring $A$ again.  The problem is, once again, that, in order to predict the position and the momentum of $B$-particle in each run of the experiment, one would need to push the reconstruction of  position-cum-momentum of each corresponding $A$-particle \emph{before} the momentum measurement at $X_1$. In Popper's own words:

%At end the point  Letzten Endes wird man sich eben doch auf den Formalismus im ganzen berufen müssen, den Herr Popper nicht verstanden zu haben  scheint.

%It wasultimately  Grete Hermann who provided the public response of the Leipzig group to Popper. 

% Es ist zwar möglich, den Impuls vor und nach der Messung genau  zu kennen, aber eine Impulsmessung nimmt, je genauer sie ist, umso längere Zeit  in Anspruch, und man weiss dann nicht, wann innerhalb dieses Zeitintervall[s] der  Stoss zwischen Messapparat und Messobjekt stattgefunden hat, d. h. wie lange das  Messobjekt den „Impuls vor dem Stoss“ und wie lange es den „Impuls nach dem  Stoss“ hatte; dies genügt gerade, um eine Ortskenntnis nach dem Stoss, die man  durch den Messprozess hindurch bis in die Zeit vor dem Stoss zurückverfolgen will,  hinreichend zu verwischen.


% she had just finished her essay on \qm \citep{Hermann1935}. At about the same time 

%He blamed Popper's confusion on the \s{statistical interpretation} of the wave function, which Hermann applies to a single system as well, although always relative to a particular experimental setup. The probabilistic interpretation tells us how to pass from one context to another---e.g., from a position measurement to a momentum measurement\footnote{Hermann's objection is inserted in a context in which she addresses the deeper question of \q{physical reality of the calculation of post trajectories}. Hermann's objection is inserted in a context in which she addresses the deeper question of \q{physical reality of the calculation of post trajectories}. Hermann challenged Heisenberg’s concession that the reality of past processes is merely a \emph{matter of taste} \citep[63**]{Hermann1935}. According to Hermann, the reality of past trajectories is, so to speak, a \textit{matter of context}. In Popper’s parlance, each sequence of selection and measurement defines a complete and self-consistent description. Yet different sequences (momentum-first vs. position-first) cannot be combined into a single description of the same individual system. There is no context-independent \s{path of the electron} that can be reconstructed even post facto. Popper’s insistence on pushing the retrodiction \emph{before} the momentum selection stands in direct contrast to this view. For a comparison Popper-Hermann, see \citet{Frappier2016}}.\todo{selection followed by a mesra}.

%While the responses of the Leipzig physicists engaged with the technical details of the apparatus, Hermann addressed the philosophical point. He blamed Popper's confusion on the \s{statistical interpretation} of the wave function, which Hermann applies to a single system as well, although always relative to a particular experimental setup. The probabilistic interpretation tells us how to pass from one context to another---e.g., from a position measurement to a momentum measurement\todo{check}


%The failure of the \TE\ should have convinced Popper that the \CI\ between preparations and measurements he sought to challenge was a central feature of \qm. On the contrary, he concluded that it was not the \CI\ itself, but the particular ETP-like set-up he had chosen to challenge it, that was at fault.

%The failure of the \TE\ should have convinced Popper that the \CI\ between preparations and measurements he sought to challenge was a central feature of \qm. Instead, he took it to show not the soundness of the \CI, but the inadequacy of the particular ETP-like set-up he had devised to challenge it.


%not an \s{additional hypothesis} (or even a contradictory one), as he argued, but a central feature of \qm\ that distinguishes it from classical physics.

%Popper ultimately, but only after considerable resistance, conceded that he his \TE was a mistake. However, he seems not to have fully appreciated the conceptual reason behind it. The paper concludes that the failure of the \TE\ should have convinced Popper that the \emph{coupling hypothesis} between preparations and measurements is not an \s{additional hypothesis} (or even a contradictory one), as he argued, but a central feature of \qm\ that distinguishes it from classical physics. On the contrary, Popper drew the moral that it was the particular kind of ETP set-up he had chosen that was at fault and attempted to challenge it via an EPR set-up. In the 1980s, \textcites{Popper1982}{Popper1982b}{Popper1985}{Popper1986}, Popper replaced the ETP-like experiment with an EPR-like one in an attempt to break the relation between preparation and measurement, which, however, was once again based on a \s{gross mistake}.


%as it were indeed a mere \s{selection} that simply let go trhig particel with right momentum. In this, using a preparation followed by a one needs to restroct berfere the prepeatin. recostc cuple between preparation and measurement bfore the prearati, using the a conservation alw to preodic that of another particle that have colledide with firs. Howevef, the were the falis.

%Indeed, contrary to classical mechanics \qm requires the dinstiocn between preparation and measurement; however, it cannot have a measurement without a preparation. 

%(1) However, Popper's repeated variations might suggest that at least this class of ETP-like experiments cannot do the job. However, failure to construct a \s{perpetuum mobile} gainst does not exclude the possibility of constructing another. (2) Popper called the \emph{coupling hypothesis} between preparations and measurements is not an \s{additional hypothesis} (or even a contradictory one), as he argued, but a central feature of \qm\ that distinguishes it from classical physics. Popper abandoned his \TE, calling it a \s{gross mistake}. He returned to the issue in \textcites{Popper1967}{Popper1968}, approaching it in a significantly changed philosophical setting. However, he seems not to have completely appreciated the conceptual reason behind his mistake. In the 1980s, \textcites{Popper1982}{Popper1982b}{Popper1985}{Popper1986}, Popper replaced the ETP-like experiment with an EPR-like experiment. However, one may conclude that Popper made the \s{gross mistake} again when he proposed a \TE\ similar to that of the 1930s. Once again, Popper attempted to bypass the preparation–measurement link, attempting to make predictions using EPR-type relations rather than a conservation law. However, in both cases, the link cannot be bypassed; once again, there was a preparation that Popper failed to recognize. The \qm\ formalism yields \emph{transition probabilities} between an initially prepared state and a possible measurement outcome. Without preparation, a \s{measurement} would have no probabilistic referent. 




%A measured state (or measurement outcome) presupposes that some system has been prepared in a defined initial condition to which the measurement applies.

%In the preface\footnote{From now on, I will refer to the German edition, since it is the one most commonly cited by German-speaking scholars}, \citet[**]{Heisenberg1930a} for the first time spoke of the \s{Kopenhagener Geist der Quantentheorie} the 


%However, Popper also seems to address a new objection that he might have received in the meantime, possibly from Weisskopf. From Popper's somewhat indirect response, the objection was probably along the following line: Since Popper's apparatus indeed allows to predict sharp position and momentum for a single particle, it also allows for the production of a super-pure case. By repeating Popper's \s{predictive measurement} on particle $A$ many times, one could select a dispersion-free ensemble of $A$-particles. This, of course, contradicts the \IR\ interpreted statistically. Thus, \latin{contra} Popper, the prohibition of a super-pure case would imply the prohibition of a predictive measurement (425).

%According to Popper, the confusion arises from the fact that most thought experiments used in the quantum dabate are phrased in terms of \emph{isolated} particles. If apparatus would be possible that isolates single particles and tests whether each one lies within very small ranges of position and momentum. Keeping only those that do, would yield a dispersion-free ensemble. However, in \qm\ one always uses \emph{beams} of particles, and thus Popper's experiment should also be understood in this way: it allows predictions for the path of the $A$-particle \q{innerhalb eines Teilchenschwarms} (426). The momentum filter at $X_1$ selects a beam of $A$-particles with identical momenta, while their positions remain statistically distributed. The position of each $A$-particle is determined only after detection at $X_2$, which allows one to reconstruct its path; in each run, however, a different path will be obtained. For \emph{each} $A$-particle, the position and momentum of the correlated $B$-particle can be predicted, but the $A$-distribution remains random. Thus, no super-pure or dispersion-free ensemble of $A$-particles is obtained.

%on einem oder dem anderen Teilchen des Schwarms genaue Kenntnisse liefert, nichtc aber von allen Teilchen eines gewissen (raumzeitlichen) Umgebungsbereiches

%\q{wenn wir annehmen, daß uns das Meß- verfahren zwarb von einem oder dem anderen Teilchen des Schwarms genaue Kenntnisse liefert, nichtc aber von allen Teilchen eines gewis- sen (raumzeitlichen) Umgebungsbereiches}



%To produce a super-pure case, Popper continues, one would need an additional selection of the incoming $A$ particle at $X_2$. Popper suggests that this could be achieved by means of a shutter or aperture (\german{Blende mit Momentverschluss}) that opens only for a brief instant---exactly at the time when the predicted particle is expected to arrive\footnote{I mention this detail, since Einstein refers to it in his response}. To avoid diffractive disturbance (which would change the particle’s momentum), the aperture must be sufficiently large in space and time. But a large aperture will also let other, nearby particles through (particles about which one has no precise information), so the single particle is not truly isolated. If instead the aperture is made small so that only the predicted particle passes, its smallness (in space or time) increases diffraction and thus disturbs the particle’s momentum—destroying the precise momentum information on which the prediction was based (426)\todo{Asymmetry position momentum?}.



%\begin{itemize}
%\item The filter changes the position. Popper argues that, contrary to a position selection (a narrow slit), which spreads the trajectory of the particle, a momentum selection (an electron spectrometer) lets particles of a certain momentum pass along the same path while blocking the others.\footnote{For the reason of this asymmetry, see below~**.} The assumption that momentum selection disturbs the position of the particles in the direction of flight in an unpredictable manner---that is, that the spatial coordinate of a particle in this direction changes unpredictably as a result of the momentum selection---is, since the velocity is not altered, equivalent to assuming that the particle, as a consequence of the momentum selection, jumps discontinuously (with superluminal speed) to another point along its path. This assumption, however, contradicts (modern) quantum mechanics, which indeed allows discontinuous particle jumps, but only for particles bound within the atom (discrete eigenvalue spectrum), not for free particles (which belong to the continuous eigenvalue spectrum). A theory that---perhaps in order to avoid the conclusions drawn in our text, or to save the uncertainty relations---would modify quantum mechanics so that the assumption of a positional disturbance as a result of momentum selection becomes admissible, could probably be carried out without contradiction, but even such a theory could not account for the observed statistical character of the results.
%
%\item If the position of the electron remains unchanged after the momentum measurement, one could in principle try to construct a \s{super-pure case} by \emph{reversing the normal order of selection}. First, one would perform a \emph{position selection}, localizing the particle very precisely by letting it pass through a narrow slit or a very short \s{momentary shutter} (\german{Momentverschluß}). Then, one would perform a \emph{momentum selection}, using a filter that only transmits particles within a narrow range of momenta. Since this second step seems to leave both momentum and position unchanged, one might believe that the resulting ensemble possesses simultaneously sharp~$\pos$ and sharp~$\mom$. However, the sharper the initial position selection, the smaller the probability that any particles will later pass through the momentum filter. In practice, only after a very large number of trials would a few particles randomly appear behind the filter---and these detections would be purely stochastic. Thus, even if one could obtain some information from such events, it would only concern the \emph{statistical distribution} of properties (for instance, that \s{some particle} in the opposite beam has a certain momentum), but \emph{not} the identity of which individual particle this refers to. Consequently, such a procedure cannot produce an ensemble more \s{pure} than the one already represented by a quantum-mechanical pure state, hut not about individual particles\footnote{See }\todo{no trajectories of single particles} 
%\end{itemize}


%However, one might concede to Popper that repeating the experiment many times would yield a statistical distribution of electron–photon pairs all with \emph{different}, but correlated position and momenta, but not an ensemble of photons all having the \emph{same} position and momentum. One can infer something about $B$ once you measure $A$, but one cannot control or reproduce that same condition on $B$ in the next trial without measuring $A$ again. The problem lies, elsewhere. Popper must have soon realized that the tricks works only if the momentum of $A$ is not changed by the selection at $X_1$, so that entire path of $A$ can be reconstructed up to the collision at $S$.





%Every predictive measurement is, by its very nature, also a preparation, since it leaves the system in a definite state from which further predictions can be made. By contrast, non-predictive measurements may be destructive: they yield information about a single event but do not prepare the system for subsequent observations.
%Non-predictive measurements such measurements do not prepare a useful post-measurement state and may destroy the system.  

%Popper’s idea was to imagine a measurement that gives predictive information about an individual system without preparing it in the corresponding eigenstate.

%In a destructive measurement, there is no surviving post-measurement state; hence the apparatus cannot prepare anything for further prediction. However, it can still serve as a diagnostic of what the preparation must have been.

%Preparation (filter): creates anonymous pure cases — an ensemble, not individuals.
%Measurement (registration): individualizes one member of the ensemble and collapses it into a definite, pure state. Without measurement, there are only anonymous cases (a statistical description); without preparation, there is no ensemble to measure. is only meaningful as a relation between a preparation and a measurement.


%Prepare an anonymise ensemble (via filtering), and then make predictions about \emph{individual} systems without referring to a later measurement.

%After all, how could the position change if the momentum remains the same?


%some photons are absorbed and re-emitted through resonance, while others are transmitted directly. He assumed that the latter—those passing through without interaction—must traverse the filter unchanged, retaining the same group velocity as before. Only the absorbed and subsequently re-emitted photons, he argued, could experience delay or blurring. For the directly transmitted photons, by contrast, no mechanism seemed available that could increase their positional spread. Hence, paradoxically, the narrower the frequency band $\Delta \nu$ selected by the filter (smaller $\Delta p$), the smaller the expected spatial dispersion effect—implying a reduction rather than an enlargement of the positional indeterminacy $\Delta q$. This conclusion appeared to stand in clear contrast with the \IR. For Popper, however, the \IR\ expressed merely statistical relations limiting the homogeneity of an ensemble of photons, not constraints on individual ones. The filter, he argued, selects a homogeneous ensemble of photons with the same momentum, whose individual positions remain unknown but unchanged. After all, how could the position change if the momentum remains the same?

\todo{check afgain}




%Popper argued that once a photon has passed through a filter \emph{without being absorbed}, there is no obvious mechanism that could delay or smear its position; all transmitted quanta, he suggested, should traverse the filter with the \emph{same group velocity as before}. Any blurring could only arise if the group velocity depended on wavelength—that is, the only mechanism for broadening a transmitted pulse would be dispersion, due to differences in group velocity across the transmitted frequency band. Yet, he noted, the better a filter of type II selects a narrow frequency band (small $\Delta p$), the smaller this dispersion effect becomes, so that the spatial indeterminacy $\Delta q$ is reduced rather than increased. This seemed to him a striking contrast with the uncertainty relations, which predict that narrowing momentum should enlarge position uncertainty. 



%What he overlooked is that passing through a filter does not simply preserve each photon’s velocity and position; rather, it \emph{prepares} a new quantum state with a correspondingly broadened spatial distribution.


% futher with Weisskopf, Poper probably to the questio, he know seems And ETP-type experiments clearly recognize that if such a procedure were to yield the desired information, this would constitute an empirical violation of the uncertainty relations; in the statistical interpretation, a violation that Popper wanted to avoid.

\todo{improve}. 


%It is indeed, possible to make a predictive measurement that is uncoupled from a previous physical selection.

%The notion that measurement and separation must be linked is among the most entrenched prejudices, and only such a presupposition explains why the straightforward arguments showing the contrary had not previously been made.

%\text{Predictive measurement without prepration} whenever a “predictive” measurement is destructive, its only meaningful role is retrodictive — it tells us about the past preparation rather than defining a future prediction

%Predictive by correlation $=$ Retrodictive about preparation, not a new preparation.

%EPR correlations allow inference without intervention —
%they let you predict what would be observed if you later measure B,
%but they do not prepare B in that condition.

%

%The state of B depends on which measurement was performed on A, and on its outcome.

%Thus predcivee meausret whc alos a prepration ... that predive only on the abseis of apreviw orepat.  \text{A statistical ensemble of pairs } (x_i, p_i),

%Every genuinely predictive measurement is also, by its very operation, a preparation — or more precisely, it defines a new ensemble conditioned on its outcome, thye are conditon. One can at most non-rpdicive measuret, as paths reconstruc that are more pre that more precie that \ID wold alled.

%Popper tacchad a new vsroo of the which he probably jpped hoped will be pubcla. Popper \q{Eine Impulsaussonderung (Elektronenspcktralapparat, Lichtfilter)
%hat nämlich im Gegensatz zu der Impulsmessung individueller Teilchen - etwa mit Hilfe des Dopplereffektes - die Eigenschaft, die Impulse bzw. die Impulskomponenten der ausgesonderten Teilchen nicht zu beeinflussen}. Thef irst onto was discs, that again. Here Popper disc daß \q{nur Elektronen mit einem gewissen vor- gegebenen Impulsbetrag IJ21 in den Spitzenzähler einfallen;} ist A ein Lichtstrahl, so erreichen wir ähnliches durch ein Filter..  onl t ep lih quat with a certain moemt. Notive the expresio, that inded filter, let fot that already had the same moemtim beofer. THis, oen,   
%
%One might clal that we do not mow evp what o filte does. But, that we must assime that let got th the photes other. snce other options are appear to Popper imposebe.  In suggested that  electrosco filre.  Hier könnte man nun einwenden, daß wir vielleicht von der Theorie des Filters zu wenig wissen: Vielleicht verändert die Impulsaussonde- rung mittels eines Filters doch die Lage des Quants, und vielleicht tritt entsprechendes bei allen Impulsaussonderungen auf? 1. Die das Filter passierenden Quanten werden (und zwar unstetig, da ihre Geschwindigkeit jedenfalls nicht in un- berechenbarer Weise gestört wird) in unberechenbarer Weise auf ihrer Bahn zurückversetzt. 2. Die Quanten gehen gar nicht durch das Filter durch, sondern der Filterprozeß ist ein Absorptions- und Emissions- vorgang, und zwar ein Vorgang mit gerichteter Absorption und
%Emission. Oncae ga, the no positon hs inca
%
%In the case of the filter, two possibilities in particular come into consideration:  1. The quanta passing through the filter are displaced on their paths in an unpredictable manner (and indeed discontinuously, since their velocity is in any case not disturbed in an unpredictable way).   2. The quanta do not pass through the filter at all; rather, the filtering process is one of absorption and emission — specifically, a process involving directed absorption and emission. That the most playbel hptey, is that also in the case of the pfilter simply go hr ucnajf

%In suggested that  electrosco filre.  Hier könnte man nun einwenden, daß wir vielleicht von der Theorie des Filters zu wenig wissen: Vielleicht verändert die Impulsaussonde- rung mittels eines Filters doch die Lage des Quants, und vielleicht tritt entsprechendes bei allen Impulsaussonderungen auf? 1. Die das Filter passierenden Quanten werden (und zwar unstetig, da ihre Geschwindigkeit jedenfalls nicht in un- berechenbarer Weise gestört wird) in unberechenbarer Weise auf ihrer Bahn zurückversetzt. 2. Die Quanten gehen gar nicht durch das Filter durch, sondern der Filterprozeß ist ein Absorptions- und Emissions- vorgang, und zwar ein Vorgang mit gerichteter Absorption und
%Emission. Oncae ga, the no positon hs inca

%This argument is, of course, not decisive; it is conceivable that theoretical considerations or the discovery of new effects might compel us to assume that the atoms of the filter absorb the incoming quanta and, after a time interval $\Delta t$ that varies randomly within a certain range (with $\Delta t$ increasing with the strictness of the filter, since otherwise the uncertainty relations would not be preserved), re-emit them in the original direction. Nor do I wish here to invoke the major theoretical difficulties of such an assumption.


%In a diffraction experiment, the position is specified at $x_1$. Ideally, the theory predicts that the particles’ energy—and hence their speed—remains constant, while their direction, and thus momentum, is spread. This can be tested by placing a photographic plate at $x_2$ and observing the distribution of spots. Each spot, a single particle hit at position $\delta y$, can be measured with arbitrary precision, allowing one to determine the angle $\delta \theta$ and the corresponding past momentum $\delta p_y$. The distribution of these past paths must conform to the theory’s predictions; otherwise, the theory must be rejected.

%\todo{metaphysics}


%withut recro, indeed \qm would be mtaifia


%\begin{itemize}
%  \item \emph{Type I apparatus:} An instrument decomposing even short light pulses into its different frequency (or wavelength) components. According to quantum mechanics, the better its spectral resolution, the more it must blur the positions of the light quanta along the beam direction. Otherwise, it would allow the realization of an impossible case: a pulse that is both short (localized in time) and monochromatic. After momentum measurement, quanta that had nearly identical positions before would then arrive at the counter at very different times. (The actual existence of such an apparatus is doubtful, but this does not affect the argument.)
%  \item \emph{Type II apparatus:} An instrument that does not resolve short pulses but only infinitely long wave trains. Short light pulses are merely attenuated, more or less, depending on duration (in extreme cases, either fully transmitted or fully blocked). The plausibility of such devices arises from the symmetry between monochromaticity with respect to beam width and beam duration: a truly monochromatic wave must be both infinitely wide and infinitely long. Known devices, such as diffraction gratings, already select momentum more precisely the wider the incident beam, differing from Type II only in that their resolution decreases with beam width rather than beam duration.
%\end{itemize}

%  \item  A purely hypothetical device that could spectrally analyze \emph{all} pulses, including arbitrarily short ones. If such an instrument existed, it would perfectly illustrate the uncertainty principle: sharper momentum resolution would necessarily blur position. Popper mentions it only as a limiting case to represent the orthodox view.  
%  \item \emph{Type II apparatus:} A more realistic filter, which does not resolve short pulses but only acts on infinitely long, strictly monochromatic wave trains. Short pulses are merely attenuated. Popper suggested that such filters might avoid producing additional positional blurring and thus could serve as a loophole to the orthodox interpretation.  
%further subdivide a quantum-mechanical ensemble into sub-ensembles– let alone into individual trajectories

%Nothing allows us, in advance, to divide the ensemble into subsets with different individual decay constants — there are no “slow” and “fast” atoms hidden inside. The decay is fundamentally stochastic and homogeneous across the ensemble. Hence, in predictive terms, you cannot meaningfully speak of sub-ensembles with different average lifetimes: all members share the same decay probability law.


%Once time has passed and some nuclei have decayed, one can retroactively classify them: those that decayed early, those that survived longer, etc. Now you can define sub-ensembles according to the observed outcome (decay time).
%For example: all nuclei that decayed before $t_1$, those that survived until $t_2$, and so on, But this classification is retrodictive — it’s only possible after the decay has occurred, not before. It’s a contextual subdivision defined by measurement results, not by intrinsic pre-existing properties.

%Predictively: the ensemble is homogeneous; no further subdivision is meaningful — every nucleus obeys the same decay probability.
%Retrodictively: once events have occurred, we can separate outcomes and speak of sub-ensembles, but this distinction depends on context (the time of observation or the measurement of decay).

%For example, a sample of undecayed radioactive nuclei, all characterized by the same average half-life, cannot be subdivided into sub-ensembles with different predicted average lifetimes: the decay law applies uniformly to all members of the ensemble. Only retroactively, after some or all of the nuclei have decayed, can one distinguish subgroups according to the actual decay times. Such a subdivision, however, is purely contextual—it reflects the measurement outcomes, not pre-existing differences in the prepared ensemble.

%Logical problems begin when such propositions are truncated by omission of reference to the preparation and/or registration.




%One can achieve a non-predictive reconstructions of past trajectories, yet only \emph{between} a selection and a subsequent measurement. Popper's \s{coupling hypothesis},  is not an additional hypothesis, is an integral part of \qm. It is nothing but the statement that all predictions in quantum theory concern \emph{transition probabilities} form the from the prepared state to measurement state. In this sense, the coupling is not a contingent feature of the setup — it is constitutive of what prediction means in quantum mechanics. based on a mistake.


%General Principle. — In attempts such as the ETP and Popper schemes, triangulation based on conservation laws or relative correlations fails because the path of the particles cannot be reconstructed prior to measurement. A momentum measurement necessarily destroys positional (and thus temporal) information, so the time of interaction cannot be inferred. More generally, no correlation can circumvent the preparation–measurement link: one may predict with certainty either position or momentum, but never both simultaneously, and the \IR\ always reasserts itself at the level of the correlations.

%However, this triangulation fails, since the correlations do not provide simultaneous sharp values of complementary observables: one can predict either position or momentum with certainty, depending on the measurement performed on $A$, but not both at once. 

%Thus the time of the particle’s passage cannot be reconstructed.

%His scheme relies on using momentum conservation to infer position as well, but the act of measuring momentum already destroys the sharpness of positional information, so the attempt to circumvent the \IR does not succeed.

%In classical physics, correlations arise from conservation laws. For example, if a particle at rest decays into two fragments, conservation of momentum ensures they fly apart with equal speeds in opposite directions. Measuring the momentum of one immediately gives that of the other, since each momentum \s{exists} independently of measurement. In quantum mechanics, however, the outcome for the first particle is random, and in an entangled pair no definite momentum eigenstate can be assigned in advance. Yet once one particle’s momentum is measured, the other’s can be predicted; and if one instead measures position, the partner’s position can likewise be predicted.

%\cop{In classical physics, correlations often arise due to certain conservation laws. For example, a particle at rest may decay into two identical fragments that, due to the conservation of total momentum, will then fly apart at the same speed but in opposite directions. If we measure the momentum of one of the fragments, we can therefore immediately infer that the momentum of the other particle must be equal in magnitude but of opposite sign.}. This seems straightforward in classical physics: \cop{The momentum of each fragment \s{exists} independently of the measurement performed on the first fragment, and the inference of the momentum of the second particle follows directly from the conservation of momentum.}  \cop{However, in the quantum setting the outcome of the measurement on the first particle is completely random.} It is not possible to decide in advance to put the particle in a momentum eigenstate. Nevertheless, once the momentum of the first particle is measured, the momentum of the other particle can be predicted. The point however, is if one deiced to measure the position the the position can be predicted. 

%But this is a preparation followed by a measurement. The initial momentum measurement disturbs the position. Conversely, if one decides to measure the position, then the momentum of the other particle becomes completely undetermined.

%The uncertainty principle restricts the degree of statistical homogeneity that can be achieved in an ensemble of similarly prepared systems, thereby limiting the precision of future predictions. However, it does not by itself forbid the retrodictive reconstruction of events from the combined data of preparation and measurement within the interval between these two operations. This asymmetry, at first still to disentangle preparation and measurement, is addressed by invoking conservation laws, so as to infer without preparation. That procedure, however, is incomplete, since it employs indirect rather than direct measurements, mediated by conservation laws. Yet the trick does not seem to work: we may establish correlations between momentum and position measurements, but not before the momentum measurement. Thus, even then, reconstructing the path prior to preparation remains impossible, since preparation itself erases information about the earlier state. Popper did not learn this lesson, and simply switched to entanglement, believing that this would once again allow him to beat the \IR. 


%This asymmetry has often been addressed by invoking conservation laws, in an attempt to infer properties without explicit preparation. Such procedures, however, remain incomplete, since they rely on indirect rather than direct measurements mediated by conservation principles. Moreover, the correlations they establish do not provide simultaneous access to both momentum and position prior to measurement; they only become definite once a particular observable is measured. Thus, even then, reconstructing the path prior to preparation remains impossible, since preparation itself erases information about the earlier state. Popper later turned to entanglement, convinced that it might again offer a way to overcome the \IR.




%\cop{To bypass this, one tries to avoid direct preparation in both variables and instead exploit correlations.
%	•	Conservation laws (momentum, energy, angular momentum) give correlations between subsystems.
%	•	Entanglement (in the modern language) is exactly such correlation.
%	•	By measuring one part, you infer something about the other without having “prepared” it in the usual sense.}
%	
%	Even with correlations, you don’t beat the \IR:
%	•	What you gain is conditional knowledge (if A has momentum p, then B must have -p).
%	•	But the underlying spreads still respect the uncertainty relations. The joint state of the total system satisfies them globally.
%	•	So while correlations let you retrodict or predict properties indirectly, they do not create a sharper ensemble than QM allows.

%Hope to “bypass” the preparation–measurement link.
%	•	Discover that the uncertainty principle reappears at the level of the correlations.

%Popper, never shwithc to entangele rather to the \IR did not help.


%\q{In conclusion, the uncertainty principle restricts the degree of statistical homogeneity which it is possible to achieve in an ensemble of similarly prepared systems, and thus it limits the precision which future predictions for any system can be made. But it does not impose any restriction on the accuracy to which an event can be reconstructed from the data of both state preparation and measurement in the time interval between these two operations.}. 

%preparation followed by a mearuem, thus before the cannot be determined. 


%The \s{perfect mirror symmetry} ($y_A=y_B$) is broken; instead, the correlation now carries the shape of the diffraction pattern. with slit open, you see a tight diagonal line; with slit closed, you see a diagonal modulated by the diffraction fringes. Now suppose you record coincidences: every time A clicks at some y_A, you note where B clicks at the same time y_B. Without slit: coincidences line up along y_A \approx y_B. With slit: coincidences are still along a diagonal, but modulated: some diagonal regions have many dots (bright fringes), others have few (dark fringes). One can see an interference pattern in coincidences even though locally each side just sees a broad blob.

%In the double-slit case, diffraction at $A$ modifies the way $A$’s spread relates to $B$’s spread, but not the distribution of $B$ alone. $B$’s individual results remain completely random until they are compared with $A$’s through classical communication. The \s{perfect mirror symmetry} ($y_A=y_B$) that held without the slit is replaced by correlations carrying the shape of the double-slit interference pattern. If one records coincidences—matching each detection at $A$ with a simultaneous detection at $B$—the difference becomes clear: with no slit, coincidences line up along $y_A \approx y_B$; with a double slit, coincidences still fall along a diagonal, but are now modulated, with bright and dark fringes. Thus, an interference pattern appears in the coincidence counts even though each side locally shows only a broad, featureless distribution.


%If slit $A$ is closed, indeed, nothing changes in the momentum distribution of $B$, which is already maximally spread . 

%Indeed, if you measured $A$ somewhere far away, you’d know immediately where $B$ is. However, the momentum is completly spread in both cases, so that all Geiger counter will be activated even without closing the the silt $A$. If $A$

%(up to experimental resolution). But until you actually measure, each one by itself still looks \s{fuzzy} in momentum. You only look at $B$, ignoring $A$: Nothing changes. 	You see the same broad, structureless distribution that you would have seen before A encountered the slit.\footnote{ghost diffracti that is }


%the marginal $P\left(p_A\right)=\int d p_B\left|\widetilde{\Psi}\left(p_A, p_B\right)\right|^2$ is uniform. In other words, momentum is "\\s{spread out} for both particles, from the outset. Because of the perfect correlation structure, at any later time the wavefunction has the same form. As shown before, the momentum distribution is maximally spread out. 

%- Before any slit: the state $\psi\left(x_A, x_B\right)=\delta\left(x_A-x_B\right)$ already implies each particle's momentum is completely spread out. The only thing well-defined is the correlation $p_A=-p_B$. Momentum of B was already maximally spread before the slit. The flaw in Popper’s thought experiment is that he assumed B initially had a sharp momentum, so that A’s slit would somehow “spread it.” In reality, the perfect position correlation he started with already meant both A and B had completely indefinite momenta. The slit only sharpens correlations, giving conditional diffraction patterns, not new physics. $\widetilde{\psi}\left(p_A, p_B\right)=\delta\left(p_A+p_B\right)$

%\todo{Refer to the notion of } Popper imagined that putting a slit on A would instantly change the momentum distribution at B. Wrong: the local distribution at B is unaffected → no signalling. Right: the correlation structure between A and B changes → ghost diffraction can appear in coincidences.








%Reconstructing the past of the particel Precisley one can only measre between prepeation of mearue, and cannot go back behyind the prepartion. However, the sung the cosnervation law . However, this passae. It was Hermann who insisted that the reconstruction of the past depends on the  succession of prepearation-measurement chsoen: if one chooses first to measure this, and then that,  the resulting reconstruction changes accordingly. The path prior to the  preparation itself is therefore meaningless. Popper, however, never absorbed  this lesson.

%Without a preparation, the theory has no way to assign probabilities to measurement results. Popper’s attempt to construct a perpetuum mobile against the \IR was ultimately based on the idea of having a predicige measurement that does not presuppose a preparation, or that cannot be used as a preparation. 



%Hhe framed this as an argument about the incompleteness of the wavefunction, not as a proof that B literally “has” exact values of position and momentum in the classical sense. Popper’s “with any precision we like” overstates Einstein’s position and shifts it closer to a realist assertion. %He framed this as an argument for the incompleteness of the wavefunction, not as a demonstration that $B$ literally possesses exact values of position and momentum in the classical sense. Popper’s claim that these can be known “with any precision we like” overstates Einstein’s position and pushes it toward a stronger realist assertion.
%Letter to Weisskopt Thanks to Viktor Weisskopt 

%That is Popper's argument. The uncertainty principle restricts the degree of statistical homogeneity that can be achieved in an ensemble of similarly prepared systems, thereby limiting the precision of future predictions. It does not, however, by itself forbid retrodictive reconstruction, from the combined data of preparation and measurement within the interval between these two operations. It was the thought, that to bypass the by using the correton,

%In fact, the argument is rather a version of the ETP argument. However, Popper reached a different conclusion. The reason ultimately seems to be that Popper misunderstood the very distinction between selection and measurement that he himself had introduced.

%1. Popper why preparation is often described as a filtering process: the apparatus selects systems with certain properties and discards the rest, so the resulting ensemble reflects only the preparation settings. Crucially, the filter does not merely disclose a pre-existing property. For this reason such operations are better seen as \s{preparations} rather than passive \s{selections.} The outgoing beams thus emerge in well-defined momentum states, precisely as a result of the preparation. One may add that different prepared states arise depending on the chosen preparation. The measurement setup determines which observable(s) are relevant, and only those are assigned definite values. This means the individuality of the object is lost in advance: before the observation, you do not know which atom will belong to the transmitted ensemble.

%2. A \s{measurement} is different, however, in that it has meaning only relative to a prior preparation. As Weisskopf pointed out, if measurements more precise than the uncertainty relations were possible, they would allow the preparation of “super-pure” ensembles, which quantum mechanics explicitly forbids. All measurements are in fact ideal preparations that project into an eigenstate of the relevant particle. That would mean one could prepare an ensemble of systems (via repeated application of such a “super-precise” measurement) in states sharper than allowed — i.e. what Pauli, Popper, and others called a “super-pure ensemble”. One might wish to deny that measurement and preparation coincide, as Margenau suggested at about the same time. However, even in this case the difficulty remains.

%\s{to all kind of measruemnts}, the \IR obly applyt to preparations. However, 


 



%Once again that measuremetn without a prepatiaon that. Howver, that ever more deeply that he alrady preparead the particles in sharp position. Thus, the momentum already spread from the outset.


%In 1959 Popper claimed that the EPR argument was weaker, and in 1967 he again emphasized this point. In his interpretation, the EPR argument was, indeed, \s{an argument against the uncertainty relations} (16). However, even this view was contested, as seen in the correspondence addressed to Popper. Once again, in the 1980s Popper tried to bypass the link between preparation and measurement, using this time the EPR correlations. That knwloge of the position $A$, whodl cause a scattering of the momentum of a partice $B$. Since this shold not be possb, that one can  allowed to determne the $q_y$ of the partucle in the opposite di $p_y$ of $B$ increase, once we nonw the positon of $A$: (a) This papersn action at distance  (b) this wil not appen and I now position and momentum $B$ beating the undetainty reations. The mistake is again the same has been prepared  Since the particles, have been preparead from the begiing in precise positi, so both $A$ and $B$ are compltely scattered.

%here again he sought to have measurement independent of preparation by using the EPR correlation instead of a conservation law. But ultimately the issue remained the same. If one prepares with exact position, the momentum would be completely spread on the other side.

%%to circumvent from behind ... the preparation-liknage. Indeed, that the measurement on the particle $B$ depend on what you ahve measured on the particle $A$

% by measuring one property of particle $A$, and combining it with the correlation, one infers the predictive measurement  property of particle $B$ \textit{without a direct preparation}. The hope was that such indirect determinations would bypass the usual preparation–measurement link constrained by the \IR. That the question of completness is a red herring\footnote{Popper misunderstood completens, with unitlmate, the end-of-the-road thesis. This is hoeve, not how the term was used in the debate}. However, the use of the correlation to obtaine a predictive measurement bypass the \s{preparation} fails. This is what Einstein's ETP argument showed.   

%However, the failure here lies in the assumption that correlation can circumvent the preparation–measurement coupling, since one ultimately prepares from the outset in the correlation step. 



%It does not apply to all measurements.


%When we consider a pair of particles, the more precise the correlations in position, the more indeterminate are the momentum values of the two partners, whether or not a process of measurement is carried out. In the extreme case Einstein considered (which had quite different ends in view from what Popper proposed) the complete correlation in position implies that each of the two particles presents a complete scattering, or dispersion of the momentum values (even though the pair of

%Therefore, the conclusion described at ( 1) is erroneous, if, as the author assumes, the state of the composite system is really such as to imply perfect correlations; for such a state, all the counters %When he returned to quantum mechanics after the war, he simply  replaced conservation laws with the EPR correlations in an attempt to beat the  \IR. 


%%However, \cop{the standard interpretation implies that the measurement on the  right does not change the momentum distribution on the left.}
%%{The particles  has been \emph{preparead} in state of precise positon cause a dispesed momentu. Howeer, no such preparation as be performed for the particle on the right. }. 
%
%%\cop{When we \emph{prepare} a pair of particles with precise correlations in position, the more indeterminate are the momentum values of both partners, \q{whether or not a process of \emph{measurement} is carried out}. In the extreme case Einstein considered (which had quite different ends in view from what Popper proposed) the complete correlation in position implies that each of the two particles presents a complete scattering, or dispersion of the momentum values}. \cop{In the extreme case Einstein considered (which had quite different ends in view from what Popper proposed) the complete correlation in position implies that each of the two particles presents a complete scattering, or dispersion of the momentum values} \cop{for such a state, all the counters have an appreciable probability of being activated}, even before narrowing the slit. \q}But this would also bring about, for each of these functions, a greater dispersion of momentum and thereby the possibility of activating the nonshaded counters, even in the absence of any measurement.}. That thwo particles have beeb prepared in perfect position, than the momentum is spread ... \q{even before the measurement is conducted}. Once again POpper tired to by-basse the prepeation-meauremnet coupling. Are prefeclty correltatd but their momentum is not spread. Once again that of positon followed by a mearuem mof mentum. Taht sperad the momentu. But still thnks that the moemtum on the side unspread, which is not the case.  Once gain popper tied to have determinatio (via correlation) that does not presuppsoed a state prpeatation, thereby beating the \IR. (presice position byut no pread of me). But once again he failed, since 
%
%
%
% That one need the correlation EPR that \s{preapre} the both pervelcty precise position. that one pass to right large slid. One slid is close, that the momentum is spread: (1) on the side eighet nothing happes, thus we have a pefect more precse of mometym and positi. However, the premise if yoy they hav spread momentum from the outset. Thus from the outset. Once Popper tried to use a correction to bypass. Indeed, the owuld been detterm without narrown th sild. Nad once again he faild. Indeed, that with perctly ... their momentu is compltel srpead form the outse, even if this not meuare, nadeven befor nallwor the ild. The trick that Popper to measurement uncoubpled from a preparation, thus beating the \IR, ithis time using the EPR correlation to bypass the preparation. This 



%\cop{if the particles at the left whose position has been indirectly measured show an increased scatter?}. 


%Popper in the communit, that his argument was taken more serious Given the renew famed of Popper, that look at his more simpathatically. He was not anymore a bearly know teahcer, bit considered one greates taken much mores sirels. Jammer made the plausible conjecture that Popper’s argument served as a mediation between the 1931 ETP argument and the later EPR argument. Indeed, Popper’s argument corresponds reforat ETP, replacing time–energy with position–momentum, as the EPR argument required. However, Popper's influence must be excluded for purely chronological reasons. Popper’s is indeed a version of the ETP argument, in which he exploits precisely the same backdoor that Einstein and his collaborators had considered possible, but had already closed. The reason is ultimately connected with Popper’s introduction of the distinction between preparation and measurement, that \citep{Margenau1974} credited for having introduced. 



%The latter produces a new state. A measuring arrangement, such as we have just finished describing in the previous section, can be used to "filter" through objects possessing a prescribed value of a certain physical quantity. It is enough to make a suitable slit through the screen on which the atoms fall to transform the Stern-Gerlach apparatus into a source of identically oriented atoms. It thus becomes a good set-up for producing pure cases.  Nevertheless, among the ensemble there exist instances (electons that happened to pass through) to which we can assign the pure-state wave function $\psi$ corresponding to the transmitted component. But this assignment is \emph{anonymous}: we cannot tell which electron has that property without performing an additional act of observation. Hence, the \s{pure cases} exist only in a collective, non-individualized sense. So when Popper claims that \s{the electron that passed through the filter} had the same momentum already before the filtering, he treats the ensemble property as an individual attribute — as if the filter revealed a pre-existing feature rather than statistically preparing a subset. That the position is also changed, 


%In quantum mechanics, a filter (say, a slit, velocity selector, or momentum filter) does not assign a definite property (momentum, position, polarization, etc.) to any individual particle. It only prepares an ensemble of systems that have, as a group, the property corresponding to a certain pure state. But crucially, before an act of observation, we cannot say which particular particle has that property. The “purity” of the prepared state is statistical and anonymous — it applies to the ensemble as a whole, not to an individually identified object. 



%If correlations allow us to ascribe both position and momentum to $B$, then the quantum description (via the wavefunction and the \IR) is incomplete, not reality itself.  This is prpoabnly this passage that conviced that the argumetn was not primaty and argeumetn agist compet, but against the \IR.

%Thus, here Einstein that the uncertainty principle cannot be an ontological statement about reality. It is possible that the second part of the argument was later conceived by Popper as a replacement for his own.

%Popper devised the ETP argument, but he misunderstood one of its most fundamental concepts—namely, the distinction between selection and measurement. Thus, Popper’s trick fails. The point is that Popper misunderstood the notion of \s{selection}, which is not really akin to that of preparation. 


%	2.	Case 1 — Position then momentum:
%	•	Measuring position of the electron first projects the total state onto one of the position-conditioned branches.
%	•	This projection acts as a preparation: it defines a new ensemble of electrons localized around the measured position.
%	•	When you subsequently measure momentum, you are not probing the “same” pre-collision electron state, but a newly prepared one — sharply defined in position, and therefore spread in momentum according to the uncertainty relation.
%	3.	Case 2 — Momentum then position:
%	•	The first momentum measurement now defines a different kind of ensemble — one sharply defined in momentum and delocalized in space.
%	•	The following position measurement probes this new ensemble, producing a very different distribution of results (broader in position, sharper in momentum).

%Popper advanced his ETP-like argument, he was attempting to bypass the conceptual link between preparation and measurement. 
%Jammer hypothesis was that Popper's argument ETP argument from time-energy to position-momentum. 

%Without a supplementary check by an observer, it is not possible to guarantee whether a given atom has gone through the filter

%After the collision, the momentum selection via wathever filetr creates a new ensemble in which position is completely spread out. Indeed, contrary to Popper’s view, preparation and measurement are conceptually coupled: the preparation provides the state on which the measurement acts. Thre is no way to reconstruct before this selection. One can at most, revonstruct the path between the prerapti and positon . However, if one wopud have a position meaure, folwed  y one would a complete different path. The \CH (\KH) is ultamtely corerrect, every predicive meaured is liled to a prepattion. Althogut Popper, he did not seem to have recongnisd this lesson.



%But this attribuonly works out, so to speak, at the expense of the individuality of the object, one does not know in advance which are the atoms that have the property in on.  It is necessary however, to be more the filter never puts any individual object whatsoever into a new pure ... Thus, it does not make sense, to claim that the same electron has preserved the same momentum after passing through the field. \q{We can easily attribute to the objects that get through the slit the $\psi$ function of the pure case, but we cannot say which object, that is, which variable is the argument of this $\psi$ function}. Thus, the same electron or photon goes through momentum \s{unchanged} is a meaningless claim; for the new ensemble position is completely spread One of a preparation a new state \citep{London1939}. If one cannot reconstrut back to the soruce, the backdoor closes, and one cannot triangulate away the preparation, to get a precrtive maeusra ...  

%Consider a monochromatic beam passes through a narrow slit at a given $y$–coordinate. While the exact direction of each particle after the slit is not predetermined, once we focus on those that emerge in a specific direction, we can compute their momentum components precisely. These particles form, in effect, a conceptual sub-beam, whose trajectories can be predicted and experimentally tested. A variant of this situation occurs when selection is made along the flight direction: due to momentum dispersion the packet spreads out, but by recording impacts with a moving film strip, one can correlate position and momentum at each instant. \todo{To Bohr might be attributed yet a third stance: retrodiction is in fact what happens in all measurements, but it is noticeably limited by complementarity only in quantum mechanical cases.}


%So to assign a \s{path} to one particle before specifying the on the other is meaningless — the state of each subsystem is not defined independently. 

% This total state is fixed and evolves deterministically, but neither particle possesses an individual state. Each has only relative states with respect to possible measurement contexts on the other. 


%The intuition behind the ETP-like strategies was to find a way  around this coupling, in order to obtain a measurement without a previous preparation that would not be constrained by the usual uncertainty relations. it defines a new ensemble of electrons localized around the measured position. When you subsequently measure momentum, you are not probing the “same” pre-collision electron state, but a newly prepared one — sharply defined in position, and therefore spread in momentum according to the uncertainty relation. The first momentum measurement now defines a different kind of ensemble — one sharply defined in momentum and delocalized in space. The first non-predicte measuremet acts as a preatio. Before the preparation (filter, slit, velocity selector), there is only an entangled total state. After the preparation, the system is engeistate of operator. No state of a subsystem exists prior to the specification of the prepeation preoceure. Therefore, any attempt to reconstruct a path before preparation





%The  retrodiction of the particle’s past trajectory, combined with non-statistical correlations such as conservation laws, seemed to allow arbitrary sharp predictions independent  of a previous preparations, thereby bypassing the \IR. Yet this trick fails, as Einstein showed, since one can only reconstruct within the context set by a preparation and a subsequent measurement\footnote{NBalletn very simpaththt tp Popper make this poit quait clar/}. If a moral can be trhgot by Poper's mistake the reconstructed past itself depends on the kind of  preparation chosen. In this sense, the preparation–measurement coupling is, indeed, central to \qm. Popper eventually recognized that his original argument was flawed, but he never fully understood the reason why. When he later attempted to substitute the original ETP-like experiment was replaced an EPR-like one \citep{DelSanto201}, new set-up was plagued by the  same difficulties, even more profoundly, the conservation could be repalded by EPR correlations.

%The path pf the particle before the momentum selectio cannot be reconstruce. Indeed, as Hermann pointed out, 




%According to the uncertainty principle, narrowing the momentum spread $\Delta p$ should increase the uncertainty in position $\Delta q$ (since $\Delta x \Delta p \gtrsim h$). But in Popper’s reasoning, a good narrow-band filter has the opposite effect: it reduces position blurring, so $\Delta q$ gets smaller as $\Delta p$ does.

%Supporting this view (in addition to other considerations) was the analogy with emission and mean lifetime: one must assume that an atom also requires a finite time for absorption. Absorption becomes very improbable, or even arbitrarily improbable, if the duration of the wave packet is much shorter than this mean lifetime. Thus, the average residence time of the relevant atom implies that very short wave trains will no longer be \s{filtered,} but will pass through largely unaffected. Popper admits this is a delicate point, but he tries to show that, at least in principle, one could have type II filters that transmit photons cleanly without producing the kind of unavoidable spatial blurring demanded by Heisenberg’s interpretation. If such filters exist, then his proposed “filter experiment” might still circumvent the uncertainty principle.



%Anticipating objections, Popper considered two possibilities. Either no such “non-blurring” filters exist, because even transmitted quanta suffer uncontrollable delays; or they exist, but only imperfectly, since absorption might lead to re-emission or incomplete suppression of unwanted wavelengths. Popper dismissed these concerns, arguing that re-emission could be made negligible and that suitably combined filters could in principle produce sharp cut-offs. On this basis he maintained that type II filters might exist, and that his proposed filter experiment could still evade the uncertainty principle.


%The question (1), whether filters of these two types exist, is itself perfectly meaningful within the current (Heisenbergian) interpretation; the further question (2), whether a filter of type II blurs position or not, would at first sight appear “meaningless” within the current interpretation, and for the following reason: a stationary state (infinitely long wave train) is eo ipso “smeared out in position.” For this reason, the question of where a light quantum is located, whose position has been registered only upon the “breaking up” of the wave train $n$ at the point counter, makes no sense. Only if one could determine such a position for the wave train before (and after) the filter, would the question of whether the filter causes spatial blurring have meaning. However, to establish a position determination before the filter, one would need an immediate shutter, and then a filter of type II would, by definition, be ineffective. It is noteworthy that the arrangement used in my thought experiment makes this previously “meaningless” question (2) appear as meaningful (I prefer, as you may recall, to avoid using the term “meaningless”). For if the experiment proceeds as I envision it (with coincident detections at $X$ and $Y$), then question (2) is decided positively; otherwise negatively. In any case, it is then decidable.




%\footnote{Popper’s second mistake was to think that because he was considering rays of photons and electrons, the information gathered through the measurement would tell us that there was a particle in the other ray with such and such properties, but it would not allow us to tell which particle that was, ergo it would not allow us to select a sub-ensemble of particles from the ray that was more informative than a quantum mechanical pure case. That Popper's msitak that without a preap. Howece, I dnot thing that this was Popper mistak. Poper, mistake was Of the idnicitaly, of particel. Since it does not a previosu preparation ... }.
 



%Popper’s error in treating the uncertainty relations as applying \emph{strictly} only to ensembles, thereby allowing (for a suitably chosen single case) sharper knowledge than the uncertainty bounds.   %You can talk about the \s{past} of a quantum system, but only in relation to a specific experimental setup.

%The "relative" character of the quantum mechanical description, which she regarded as the "decisive achievement of this remarkable theory."


%\q{Since every physical description and explanation of processes is valid
%only relative to its respective observational context, so the calculation of, say, a corpuscle
%trajectory combining the variables of different observational contexts into one representation remains physically vacuous precisely to the extent that it exceeds the uncertainty relations: it is uncheckable and provides no grounds for future prediction}
%As Hermann warns her readers, such ‘backward deductions’ or ‘retrodictions’ cannot, as Popper had hoped, be extended earlier than the photon–electron inter- action. One cannot legitimately describe what happens between two consecutive position measurements on an electron or between a momentum measurement on the particle followed by a position measurement because such narratives fail to take into account both the duality of the quantum object and the fact that there are two different ‘observational contexts at play’. Physically speaking, these descriptions of past paths are meaningless, ‘empty’ %Bohr’s Slit and Hermann’s Microscope

%precise non-predictive measurements have far-reaching implications, Popper found it necessary to examine the legitimacy of this assumption in more detail, a task he undertakes in Appendix VI, where he himself ultimately questions the premise. That the position is not disturbed is a key premise of Popper’s, as we shall see.

%\subsection{The Experimental Set-Up}


%Against this, she insisted that the duality experiments and the uncertainty relations constrain \emph{each} elementary process, and that wave functions legitimately describe \emph{individual} systems as well.\footnote{Paraphrasing Grete Hermann’s March 1935 critique; cf. her discussion of Popper’s \textit{Logik der Forschung} (Vienna 1935).}. However, Hermann was also critial of Heseiber one claim that is a qustion of state. Indeed, is a qestion of context. It epdens on what you desice to reconstruc.



%\cop{Such constructions actively disregard the essential fact that each specification of a variable only has meaning relative to an observational context. To treat a trajectory as if it had physical meaning is to ignore this constitutive feature of quantum mechanics. Hence, it is not just “a matter of taste” but physically meaningless.} \cop{If you measure position at successive times (e.g., two position measurements in a row), you can try to reconstruct a kind of trajectory in terms of positions. But this trajectory is already tied to the fact that you chose a position–position context. If instead you measure momentum and then position, the “reconstruction” will look different, because you’ve combined two different contexts (momentum first, then position). The uncertainty relations mean you cannot combine these into one single consistent picture of the past; they yield different, mutually incompatible reconstructions.}. Hermann stresses that it is meaningless to pretend these reconstructions can be fused into a single classical-like trajectory that would hold \s{independently} of the measurement context. That’s what she means when she says that mixing variables from different contexts into one representation “exceeds the uncertainty relations” and becomes physically vacuous.



%For this reason, it is impossible to reconstruct the path prior to the momentum  selection. Popper’s claim in the second case represents a fundamental  misunderstanding of the very notion of preparation.

%\begin{itemize}
%  \item Preparation in a position eigenstate followed by a momentum measurement 
%  yields one kind of reconstructed trajectory (sharp initial position, later momentum).
%  \item Preparation in a momentum eigenstate followed by a position measurement 
%  yields a different, incompatible reconstruction (sharp initial momentum, later position).
%\end{itemize} Since these reconstructions belong to different observational contexts, they 
%cannot be fused into a single context-free trajectory. Any attempt to combine 
%them into one representation exceeds the uncertainty relations and becomes 
%physically meaningless.


%That a way non-rseonca are once agin.  
%
%
%Grete Hermann distinguishes between two types of inference. On the one hand are posterior reconstructions of an electron’s trajectory, such as when one tries to calculate exact positions and momenta from successive measurements. These combine variables from different observational contexts into a single description and thereby appear to exceed the limits set by the uncertainty relations. Hermann argues that such reconstructions are physically meaningless, since they ignore the wave–particle duality and the relativity of quantum descriptions.  
%
%On the other hand are causal inferences that belong to the interpretation of a measurement itself. These must respect the dual character of atomic processes, where each observation situates wave and particle aspects in relation to one another, depending on the experimental set-up. For example, a microscope observation may be interpreted either in terms of sharp position with unsharp momentum, or vice versa, depending on the observational context. In this way the uncertainty relations are preserved.  
%
%Hermann concludes that calculations of trajectories which mix observational contexts provide no valid predictions, while inferences grounded in the duality of measurement contexts remain meaningful and testable.



%That measruemetn tepedndes on the prepartion ... ondeed, if woy preparaete on a sharp moemthim thatn and tuatn pptoi, but if meusre of positon and then momebtum Onwe cowuld have found a different result physical reality.
%
%
%%histroical reality ... measruemnt should be repatable. Margenau Cohen Bunge 

%

%The Compton-Simon and Bothe-Geiger experiments proved strict energy and momentum conservation in individual atomic events (Compton and Simon 1925a, 1925b, Bothe and Geiger 1924, 1925)


%: \q{But the questioner can choose between them AFTER the projectile is once and for all on its way}. Thus, \q{Only one measurement – of the time or the color – can be carried out precisely, and in fact according to EINSTEIN one can still decide after the departure of the light ray which of the two predictions one wants to choose.}. 


%dated 5 April 1932, Einstein seems to have eladey in terms of position and momentum.  


%Between 1931 and 1933, while the discussion on the \s{past of an electron} was ongoing, Popper worked on the manuscript of the \emph{Die Beiden Grundprobleme der Erkenntnistheorie}. A large part of the manuscript is devoted to refuting Schlick's claim that the laws of physics are pseudo-propositions, as presented in his 1931 paper on causality, which Schlick knew very well. 


%\footnote{In the foolwing \cop{the symbol $\delta$ to indicate the uncertainty in an individual measurement, while $\Delta$ refers to the standard deviation of an ensemble of similar measurements)}}


%In retrospect, Popper himself would later describe these efforts as \q{a gross mistake for which I have been deeply sorry and ashamed ever since} \citep[15\hide]{Popper1982}.  For this reason, recent literature has referred to this episode only in passing, ultimately dismissing it as a youthful blunder, and concetrate on his work 1980s. 


%The \cop{observation of the position will alter the particle's momentum by an unknown and undeterminable amount such that after carrying out the experiment our knowledge of the electronic motion is restricted by the uncertainty relation' (Heisenberg, 1930, p. 20). This way of putting things seems to suggest that before the measurement has taken place the particle does indeed have a well-defined position and momentum, even if we cannot know them.}.



%We will get “spin up” or “spin
%down” with equal probabilities, but we have no means of predicting which
%particular outcome will be obtained. Thus, the cnnot cor




%But this attribution only works out, so to speak, at the expense of the individuality of the object, as one does not know in advance which are the atoms that have the property in question. We can easily attribute to the objects that get through the slit the $\psi$








%The filter prepares the state of the survivors, but it does not measure a pre-existing momentum for each individual.


%	•	→ On this reading, Einstein critiques feasibility, but misses that Popper wanted to deny that predictive measurements imply preparation.
	







%measuement is alywas relative to a preparatio, .... (mascropic apparatus?) is always relative to a preparation. That get depedns on the prepration you deside to make.  ... prepare ... system in eigenstate in some direction ... relational to the apparatus ... Unsharpness is ... always less the one ...  projector ... Ludwig ... preprationelle Teile and effectuelle Teil ... Ozawa ...Intergral one ...  

%Thus, Einstein regarded Popper’s filtering method as subject to the same trade-off underlying the uncertainty relations. 

%\section{Conclusion}







