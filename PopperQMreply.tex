\documentclass[11pt]{article}
\usepackage[utf8]{inputenc}
\usepackage{els}
\usepackage{rebuttal}
\newcommand{\RT}{Reviewer~\#2\xspace} 
\newcommand{\RO}{Reviewer~\#1\xspace}


\begin{document}
\section*{Response to reviewers' comments}

\reviewersection

%I would like to start by thanking the author for a very interesting and insightful manuscript. Given that, as the author correctly points out, there has been until now relatively little attention for Popper's thought experiment and his work on quantum physics in the literature on Young Poppper, this manuscript definitely is a significant contribution to this field. Moreover, by discussing the development of Popper's thought experiment through his interactions with several quantum physicists, the author clearly shows that this topic is of interest both to historians of philosophy of science and to historians of quantum mechanics. I think that the manuscript is already very welldeveloped, and the only remarks I have concern little details about specific aspects of the paper's structure.

\begin{point}
The distinction between predictive and non-predictive measurements is introduced without really specifying what these terms mean precisely. Moreover, near the end of the paper there is also talk of non-prognostic measurements, which seem to mean the same thing as non-predictive, but this is not specified.
\end{point}

\begin{reply}
Non prognostic and non-predictive are indeed the same. I have forgot to update an early transation progniostic is more simular to the oriciala genra,.
\end{reply}



\begin{point}
- The paper argues that both the claims by Jammer and by Margenau "fail to withstand critical scrutiny". In the case of Jammer, I see why this is the case: the chronology does not allow for Popper's thought experiments to have inspired EPR. In Margenau's case, however, it is not completely clear to me why Popper's distinction between physical selections and measurements could not have anticipated Margenau's distinction between preparation and measurement. While it is indeed the case, as the author shows, that there are some differences and that Popper's views were confused, there are also similarities, as the author points out (p. 10). Hence, one could very well claim, as Margenau did, that Popper's distinction did anticipate (aspects of) Margenau's.
- At the start of section 3 (p. 9), the author writes "Popper starts from the well-known fact" without stating explicitly where Popper does so. It would be good to specify this, given that there are quite a few texts by Popper discussed.
\end{point}

\begin{reply}
My point is that philsoca selection is  smilar but ultaimatle a mistandestadin. Bit I can ideed make the more clear. 
\end{reply}


\begin{point}
This allows us to rephrase the two different interpretations (a) and (b) of equation (1) (given on p. 9) as follows:
(a) the indeterminacy relations limit the degree of statistical homogeneity that can be achieved in the physical selection of an ensemble of identical systems;
(b) the indeterminacy relations limit the precision of the measurement performed on a single system.
This is very confusing, because by now we have already three different (a)-(b) enumerations that, moreover, seem not completely consistent: the last (a) listed here seems to equal (b) on p. 9 of the manuscript, and the last (b) listed here seems to equal (a) on p. 9.
Moreover, the author then formulates Popper's program on the basis of these (a)-(b) as follows: "I (a) follows directly from the formalism; |I (b) does not logically follow from (a); III (b) follows from (a) only if one adds an additional hypothesis (c); IV the combined system (a) + (c) turns out to be contradictory" (p. 10). It is very difficult to disentangle what is going on here precisely. I would urge the author to rewrite this as a separate list, rather than as an enumeration in the text. I would also like to ask the author to be more consistent in their choice of enumeration symbols, and use distinct ones when required: e.g. (a)-(b), (i)-(ii), etc. Finally, if I am not mistaken, the author's I-IV distinguished here also appeared earlier (on p. 8, in the discussion of Weisskopf's reply to Popper). There, however, the author used (1), (1a), (1b), and (2). I think it would improve readability if the author were to use the same enumeration symbols throughout.
\end{point}

\begin{reply}

\end{reply}


\begin{point}
- On p. 11, the author argues that Heisenberg claimed that while the indeterminacy relations forbid predictive measurements of a particle's future position-momentum trajectory, they do allow nonpredictive measurements of position and momentum by combining two predictive measurements. Given that it is left unspecified how we are to understand predictive and non-predictive measurements precisely, it is not completely clear how Heisenberg arrived at this second claim (regarding the possibility of non-predictive measurements by combining two predictive measurements).
\end{point}

\begin{reply}

\end{reply}


- On p. 11, the author suggests in passing that Popper's interest in retrodictive measurements was connected to his more general philosophical views: "for Popper [...] the point is that if sharp past trajectory reconstructions were not possible, the frequency predictions of quantum mechanics could not be falsified, and the theory would therefore belong to the realm of 'metaphysics'. Without the possibility of reconstructing the past paths of particles, one could not subject the theory to empirical control". This is a very interesting point, especially given that, as the author points out, up until now Popper's thought experiment has been barely discussed in the literature about Young Popper. I would therefore like to ask the author to highlight this point more in the introduction of the paper. In that way, it will be clear that the paper's audience is not merely historians of quantum physics, but equally well historians of philosophy of science.
- On p. 17, the author states that von Weiszacker argued that Popper did not see how non-predictive measurements "escape the indeterminacy relations since they are not physical measurements at all". This point has not been made before, and it would be good to include this in an explanation of how we are to see non-predictive (non-prognostic?) measurements.
- On p. 26, the author writes that "Popper was not misled by Heisenberg's notion of 'measurement', but by the notion of 'physical selection' he adopted". The 'he' here in this sentence is ambiguous, and could refer to both Popper or Heisenberg. Also, if the 'he' used here refers to Popper, the term 'adopted' seems quite ambiguous as well, given that it was Popper who, according to the author, introduced the term (whereas 'adopted' suggests that he took it over from someone else). I would like to ask the author to clarify this sentence.





\reviewersection

%The paper claims to identify and fill a lacuna in the scholarship on early Popper, namely, his participation in the early 1930s in debates over Heisenberg’s uncertainty relations with leading physicists from Heisenberg himself to Einstein. The debate focused on a series of thought experiments culminating in the famous EPR one. The actual lacuna isn’t any lack of attention to the episode, but the missing depth and detail of the analyses it has yielded, the narrowness of their focus and the dearth of their sources. 

\begin{point}
While ignoring more detailed analyses such as Redhead’s, the new analysis seeks to redress claims such as Jammer’s about the relevance of Popper’s arguments and examples to EPR, 
\end{point}

\begin{reply}

\end{reply}




%Margenau’s about Popper’s selection-measurement distinction anticipating Margenau’s own preparation-measurement one, and Popper’s own later sorrow over a mistake.


%Along the way the paper presents the most detailed and insightful account yet of the young Popper’s participation both in the discussion of his proposals and of his exchanges with the different physicists with some insight into underlying assumptions, even into Popper’s character. The paper persuasively demonstrates that Popper’s thought-experiment setups and arguments probed and challenged the interpretation of Heisenberg’s quantum mechanics only by misunderstanding the distinction between quantum preparation, identified with classical physical selection, and quantum measurement and, crucially, the central role their inseparability plays in quantum mechanics –displaying its non-classical character. Another central distinction at work in Heisenberg’s interpretative claims is one between predictive and non-predictive measurements. The paper’s analysis sheds further critical light, albeit in cursory remarks, on Popper’s later assessment and proposals.

%I endorse the publication of the paper with a couple of suggestions.


\begin{point}
As a matter of presentation, the reader and the analysis could benefit from more clarity and information on the distinction between predictive and non-predictive measurements adopted by Heisenberg and, after him, Popper and the distinction between measurement and preparation, of physical selection, proposed by Margenau in the immediate aftermath of the episode.
\end{point}

\begin{reply}

\end{reply}

\begin{point}
One relative weakness of the paper, despite its considerable length, is the missed opportunity to present in more detail (in a page or two) how the different interpretive positions relate to the actors’ more philosophical interests and commitments (sometimes evolving) implicitly or explicitly noted in the more technical debate, especially Schlick’s, Heisenberg’s, Einstein’s and above all Popper’s. About Popper’s, the paper mentions in passing on p.11 the metaphysical status of past trajectories, or rather of the concept, in reaction to Heisenberg’s reference to matters of taste and Schlick’s to their meaninglessness due to the impossibility of their verification. But no reference is made to Heisenberg’s positivist epistemological program, noted for instance in Cassidy’s biography, contrary to Popper’s. 
\end{point}

\begin{reply}

\end{reply}

\begin{point}
The same applies to Einstein’s evolution from determinism to realism and non-positivistic epistemology. The paper describes the exchanges with Heisenberg et al as ‘series of experimental conjectures and refutations’. But there’s no explicit discussion of the use of thought experiments within the framework of Popper’s realism and falsificationism in LSD. Nor is there any consideration of the place within Poppe’s philosophy of science of his philosophy of physics and engagement of physicists. Not does the paper take notice of the epistemological response to Heisenberg, Schlick and Bohr in chap IX, sect. 73, of failing to carry out their programs and assuming an inconsistent dual interpretation and, in the new, 1959, appendix XI, adopting a Kantian metaphysical epistemology.
\end{point}

\begin{reply}

\end{reply}







\end{document}