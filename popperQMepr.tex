

%Popper anticiate preparatio and measrem, was amognt htef  \q{the uncertainty principle restricts the degree of statistical homogeneity which it is possible to achieve in an ensemble of similarly prepared systems}, bit  past \p{from the data of both state preparation and measurement in the time interval between these two operations.} \citep[367]{Ballentine1970}. Howve,r before the perato  the coipl before the be avaeded. that o cannot before the preaptaiom. Teh couplin bewteen cannot be vaded
%
%%, $\Delta q_y \Delta p_y \approx 0$..... o that their relative position is fixed ($x_1 - x_2 = x_0$), while the total momentum is also fixed ($p_1 + p_2 = 0$
%
%%to obtain a predictive measurement that does not depend on a prior preparation, by exploiting correlations between two systems.
%
%%In both experiments, Popper failed to notice that the system he started with was already prepared — not in an individual eigenstate, but in an entangled state that defines its statistical properties.
%
%

\begin{figure}
\centering
 \includegraphics[scale=0.5, trim = 0mm 0mm 0mm 0mm, clip]{1982PopperExperiment}
\label{fig:1982popperexperiment}
\end{figure}


Popper famously imagined pairs of particles emitted in opposite directions from a  common $S$ source along the $x$-axis towards two slits $A$ and $B$, beyond which a semicircle of Geiger counters was placed (\cref{1982popperexperiment}). If one \emph{knows} the position $\Delta q_y \approx 0$ of particle $A$, one can also determine, by symmetry, the position $\Delta q_y \approx 0$ of particle $B$\footnote{If particle $A$ lands at $y = 1 \,\text{mm}$, then particle B will also be found at $y = -1 \,\text{mm}$}\todo{minus sign?}. Popper then proposed closing the slit for $A$, thereby causing diffraction, i.e. a very broad momentum distribution $\Delta p_y$ (all the counters on the right would be activated). Invoking the EPR reasoning, Popper argued that two alternatives were possible: (a) the Cophenhangen view: $B$’s momentum distribution must also spread, even though $B$ never encountered a slit (all the counters on the right would be activated). This would imply an action at a distance; (b) Popper’s view: $B$’s momentum does not spread, so that one could in principle predict both $B$’s position and momentum with arbitrary precision. This would amount to a violation of the \IR.
%
Once again, the (b) alternative favored by Popper is a \s{measurement} that does not require a previous \s{preparation}, thereby beating the \IR understood as \s{scatter relations}. However, Popper failed to recognize that in his setup the particle pair had been previously \emph{prepared} in a state with precisely correlated positions. Because of this, while the total momentum is sharp, their momenta are necessarily broadly spread from the outset.\footnote{It is hard not also to conclude that also in this case Popper committed \s{gross mistake}. Indeed, Popper's setup is nothing but a special case of the EPR paper example, where particles are in a state with perfectly equal positions: $\psi\left(x_A, x_B\right)=\delta\left(x_A-x_B\right)$. As in EPR, Fourier-transforming to the momentum representation gives $\tilde{\psi}\left(p_A, p_B\right)=\frac{1}{2 \pi \hbar} \int d x_A d x_B\, \delta(x_A - x_B)\, e^{-\frac{i}{\hbar}\left(p_A x_A+p_B x_B\right)} = \delta(p_A + p_B)$. The total momentum is sharp $\left(p_A+p_B=0\right)$, but each particle's momentum is completely undetermined from the outset}. Contrary to Popper's claim, orthodox quantum mechanics predicts a broad distribution of detection events on both sides of the apparatus, and thus the activation of all counters, \emph{even without closing the slit } \citep[sec.~11.3]{Ghirardi2007}. Narrowing the $A$'s position via a slit broadens $A$’s momentum distribution, but $B$’s momentum distribution is already broad, so it does not change (all the counters on the right would be activated, anyway). Only the joint correlations are modified, but this can only be established post-facto using classical signals\footnote{By observing the distribution of $B$ alone. You cannot tell if $A$ had one slit, two slits, or no slit just by looking locally. The difference appears only in the correlations: some positions of $B$ are more (or less) likely to coincide with certain positions of $A$, depending on whether there was one slit or a double slit at $A$. To extract this pattern, you must match $B$’s clicks with $A$’s clicks via classical communication}. This is exactly why entanglement does not allow faster-than-light influence.
%\end{comment}
%
%
%
%%(e.g., Redhead, Selleri, Sudbery, Kim & Shih) 
%
%
%%\delta\left(x_A-x_B\right) \propto \delta\left(p_A+p_B\right)$
%
%
%%https://arxiv.org/pdf/1507.02010

