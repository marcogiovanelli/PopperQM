% !TEX encoding = UTF-8 Unicode

\documentclass[final,12pt,a4paper]{article}
\usepackage{wrapfig}
\usepackage[font=footnotesize]{caption}
\usepackage{fullpage}
\usepackage{els}
%\usepackage{cassirer}
\usepackage{ii}

%\newcommand{\TE}{thought experiment\xspace}
\newcommand{\IR}{indeterminacy relations\xspace}


\DeclareSourcemap{
  \maps[datatype=bibtex]{
    \map{
      \step[fieldsource=entrykey, match=DM,fieldset=shorthand, null]
      \step[fieldsource=entrykey, match=ASWB,fieldset=shorthand, null]
      \step[fieldsource=entrykey, match=ESC,fieldset=shorthand, null]
      \step[fieldsource=options,match={skipbib=true},replace={skipbib=false}]
    }
  }
}



\newcommand{\TE}{thought experiment\xspace}
\newcommand{\GE}{\german{Gedankenexperiment}}

\hypersetup{draft}


\setlength{\textfloatsep}{10pt plus 1.0pt minus 2.0pt}


\title{The Past of an Electron: Young Popper's \GE\ Against the Indeterminacy Relations}


\begin{document}
\maketitle

\begin{abstract}
Between 1934 and 1936, Popper wrote at least eight manuscripts on the interpretation of quantum mechanics, proposing a  \GE\ based on electron--photon collision in which reconstructing the past trajectory of an electron would allow one, via momentum conservation, to predict both position and momentum of a photon with arbitrary precision, thus apparently \s{beating} the \IR. Popper corresponded about it with leading physicists such as Weisskopf, Heisenberg, von Weizsäcker, and Einstein, who pointed out the flaws in his reasoning. In retrospect, Popper described these efforts as a \q{gross mistake}. Yet Jammer (1974) suggested that this mistake may have influenced the 1935 Einstein–Podolsky–Rosen argument, while Margenau (1974) credited Popper with anticipating the later distinction between preparation and measurement. Drawing on unpublished material, this paper argues that both claims are doubtful. Popper's thought experiment is more closely related to the 1931 Einstein–Tolman–Podolsky setup, though he reached the opposite conclusion precisely because he misunderstood the link between preparation and measurement. The same \s{mistake}, the paper concludes, also undermines the EPR-like \TE\ Popper proposed at the turn of the 1980s.
\end{abstract}

\begin{keywords}
Karl R. Popper \sep quantum mechanics \sep uncertainty relations \sep thought experiments \sep preparation vs.\ measurement
\end{keywords}

\vspace{1cm}

\begin{center}
Data availability statement: not applicable.
\end{center}



\thispagestyle{empty}

\newpage

\section*{Ethical Statement}

\begin{itemize}
\item Funding: None
\item Conflict of Interest: None declared
\item  Ethical approval: Not required
\item Informed consent: Not required
\item Author contribution: I am the sole author of this paper

\end{itemize}


\end{document}