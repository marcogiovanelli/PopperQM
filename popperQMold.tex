\documentclass[submitted]{article}

\usepackage{els}

\title{The Past of the Electron: Young Popper's Interpretation of Quantum Mechanics}



\begin{document}

\maketitle



 
% 1. Heisenerg
% 2. Schlick 
% 3. Knowledge of Past and Future in Quantum Mechanics∗
% 4.
% 4. Einstein 1932
% 5. Popper 1934
% 6. Popper 1935
% 7. Weizacker 
% 8. Hermann
% 5. ABL as soltuion?
 
%Ortsbestimmung eines Elektrons durch ein Mikroskop. In: Zeitschrift für Physik 70 (1931), S. 114-130.


Between the summer of 1934—just a few months before the publication of his *Logik der Forschung*—and the autumn of 1936, Popper wrote at least nine manuscripts on the interpretation of quantum mechanics. This was accompanied by an active correspondence. Two of these essays must be considered lost: a “Reply to the Criticism of Heisenberg and Weizsäcker” and a paper discussing a “measurement setup.” The latter is indicated by a letter of 27 January 1935, in which Popper writes: \q{I enclose a discussion of this measurement setup ... I have discussed and calculated this arrangement in detail with Käthe Schiff.}. Howwver, has neve most that focue, that prpanl. Wher Howeve, that since at centar incerat proc. for history of physc an hsot ohuslp hf sciecne.

Indeed, but fundalta lesson.

%(i) A measurement of q causes an unpredictable and uncontrollable disturbance of p, and vice versa. [This was first proposed by Heisenberg (1927) and is widely repeated in text books]. (ii) The position and momentum of a particle do not even exist with simultaneously and perfectly well defined (though perhaps unknown) values (Bohm, 1951, p. 100) .

 
 
 
 
\section{Heisenberg}

Heisenberg's (1930, p. 21) famous microscope for measuring the position of an electron. If the angular aperture of the microscope is $\epsilon$ and the wavelength of light used is $\lambda$, then the accuracy of the position measurement will be limited to

$$
\delta x=\lambda / \sin \epsilon,
$$

by the resolving power of the instrument. (We use the symbol $\delta$ to indicate the uncertainty or error in an individual measurement, while $\Delta$ refers to the standard deviation of an ensemble of similar measurements). Because the direction of the scattered photon is unknown within a cone of angle $\epsilon$, the $x$ component of momentum of the electron will be changed by some unknown increment in the range $\pm \sin \epsilon(h / \lambda)$. Note that this experiment is not an example of simultaneous measurement of $x$ and $p_x$. Only $x$ is measured here.

%The uncertainty principle refers to the degree of indeterminateness in the possible present knowledge of the simultaneous values of various quantities with which the quantum theory deals; it does not restrict, for example, the exactness of a position measurement alone or a velocity measurement alone. Thus suppose that the velocity of a free electron is precisely known, while the position is completely unknown. Then the principle states that every subsequent observation of the position will alter the momentum by an unknown and undeterminable amount such that after carrying out the experiment our knowledge of the electronic motion is restricted by the uncertainty relation. This may be expressed in concise and general terms by saying that every experiment destroys some of the knowledge of the system which was obtained by
previous experiments. 

\q{This formulation makes it clear that the uncertainty relation does not refer to the past; if the velocity of the electron is at first known and the position then exactly measured, the position for times previous to the measurement may be calculated. Then for these past times Apoq is smaller than the usuâl limiting value, but this knowledge of the past is of a purely speculative character, since it can never (because of the unknown change in momentum caused by the position measurement) be used as an initial condition in any calculation of the future progress of the electron and thus cannot be subjected to experimental verification. It is a matter of personal belief whether such a calculation concerning the past history of the electron can be ascribed any physical reality or not.}

The classic example is of two successive position measurements taken at definite times:  A particle with known initial momentum $p$ passes through a narrow slit in a rigid massive screen. After passing through the hole, the momentum of the particle will be changed due to diffraction effects, but its energy will remain unchanged. When the particle strikes one of the distant detectors, its $y$ coordinate is thereby measured with an error $\delta y$. Simultaneously this same event serves to measure the $y$ component of momentum, $p_y=p \sin \theta$, with an error $\delta p_y$ which may be made arbitrarily small by making the distance $L$.

the (average) momentum of a particle can be reconstructed for the past, while the second position measurement makes the momentum of the particle uncertain for the purpose of future measurements. But this is not the instantaneous momentum at any time. It is the momentum the particle would have had if it had moved uniformly between the two positions. arbitrarily large. Clearly the product of the errors $\delta p_y$ need not have any lower bound, and so the common statement of the uncertainty principle given above cannot be literally true. One may raise the objection that $p_y$ has not been measured, but only defined in the above equation. However this method of measuring momentum by means of geometrical inference from a position measurement is universally employed in scattering experiments. It rests upon the assumption of linear motion in a field-free region (Newton's First Law), which remains valid in quantum mechanics (at least for $L$ much greater than a de Broglie wavelength). 

There is here, of that I should konwo the tile of fliht of the elctron, if the avrate the (average) momentum of a particle can be reconstructed for the past, while the second position measurement makes the momentum of the particle uncertain for the purpose of future measurements. But this is not the instantaneous momentum at any time. It is the momentum the particle would have had if it had moved uniformly between the two positions. If would that teter, of cours distinc. Os that can be reconsturcte, but the trscot, cannot be verificau. If one to belive, is a matter of tate.

\section{Schlick}

Im Nachlaß befindet sich eine Druckfahne,23 datiert mit 17.Dezember 1930 (DF). 

 Let us suppose, for example, that the velocity of an electron were to be exactly measured, and then its position exactly observed; the equations of quantum theory then do allow us, indeed, to calculate exactly the earlier positions of the electron, but in actuality this statement of position is physically meaningless, since its correctness is essentially untestable, it being impossible in principle to verify afterwards whether, at the time stated, the electron was located at the calculated point. Had it been observed at this point, however, it would certainly not have reached those positions at which it was later found, since its path, of course, would have been disturbed by the observation in a manner impossible to calculate. Heisenberg16 thinks that "whether any physical reality should be assigned to the calculation concern-ing the past of the electron, is purely a question of taste." But I should prefer to put it more strongly, in complete agreement with what I take to be the incontestable basic viewpoint of Bohr and Heisenberg themselves. If a state- ment about an electron's position is not verifiable within atomic dimensions,
we can attach no meaning to it; it becomes impossible to speak of the 'path'
of a particle between two points at which it has been observed. (This is not,
of course, true of bodies of molar dimensions. If a projectile is 1)-0W here, and
a second later is a dozen yards away, it must have p;issed during this second
through the intervening points in space, even if no one has perceived it; for
it is in principle possible to verify afterwards that it was located at the inter-
vening points.) This can be regarded as the sharpened form of a principle of
the general theory of relativity: just as no physical meaning can there be
attached to those transformations which leave all point-coincidences - or
intersection-points of world-lines - unchanged, so here we may say that it has
no meaning whatever to attribute physical
 
\section{Einstein}

In the winter of 1930–31, during his sojourn at Caltech, Einstein collaborated with Richard Tolman and Boris Podolsky on a short but remarkable paper. Barely two pages long, it contained a subtle thought experiment that, in retrospect, foreshadows many of the themes that would surface later in the Einstein–Bohr debates. The three physicists imagined a sealed container, a box $B$ filled with particles in thermal agitation. One face of the box was equipped with a shutter $S$, capable of opening for a very brief instant. At the moment of opening, particles might escape in different directions.

The figure illustrates the essential elements of the Einstein–Tolman–Podolsky thought experiment of 1931. A box $B$, filled with particles in thermal motion, is equipped with a small shutter $S$. At a chosen moment, this shutter may be briefly opened, allowing particles to escape. Imagine that, during such an opening, two particles are emitted. One travels directly from the shutter $S$ to the observation point $O$, following the straight line $SO$. The other takes a longer route: it first strikes the curved mirror $R$ and, after being reflected, arrives at the same point $O$. The geometry is such that both particles, despite the different paths, can be detected at the same place.

From a classical standpoint, the observer at $O$ would seem to have all the necessary data to reconstruct the precise moment at which the shutter opened. By measuring the arrival times of the two particles, and by determining their momenta at the point of detection, the observer could, in principle, work backward. Since the distances along $SO$ and $SRO$ are known in advance, the time taken for each trajectory could be calculated. Subtracting this time of flight from the recorded arrival times should reveal the instant of emission, thus fixing the shutter’s opening as a determinate event in the past.

Einstein, Tolman, and Podolsky, however, showed that this reconstruction fails within the framework of quantum mechanics. The difficulty arises at the crucial step of measuring momentum. To determine the momentum of the particle arriving along the direct path $SO$, one might employ the Doppler effect, scattering a light quantum from the particle. But such a measurement does not disclose the precise moment of scattering. It provides only statistical information about the particle’s momentum before and after the interaction. As a result, the trajectory cannot be traced back unambiguously, and the supposed determination of the shutter’s opening time collapses.

The lesson of the experiment is that quantum mechanics imposes strict limits not only on prediction but also on retrodiction. One cannot assume that the past is laid down in full determinacy, waiting to be reconstructed from sufficiently detailed present data. Instead, the very attempt to obtain the necessary information about momentum undermines the possibility of recovering the exact history of the particle. The time of emission from the shutter, though it appears to be a simple macroscopic fact, thus falls under the scope of quantum indeterminacy. The diagram captures this tension with elegant simplicity: two paths converging on the observer, promising the recovery of a definite past, only to reveal the quantum mechanical impossibility of such certainty.

Classically, this looks like a neat trick: the different path lengths give you two independent time-of-flight measurements. With these, plus the momentum information, you ought to be able to triangulate the common emission time of both particles with great precision.

The reflected particle ($SRO$) has a longer path, so it arrives later at $O$.
	•	The direct particle ($SO$) arrives first.

Now, if you measure the arrival times of both, and you know the difference in their path lengths, you can infer the flight time of the direct particle — without measuring its momentum directly.

2. Why this matters

Classically, that means:
	•	Use the geometry ($L_{SO}$ and $L_{SRO}$) and the timing difference $\Delta t$ between arrivals.
	•	From the reflected particle’s data, deduce when the shutter must have opened.
	•	This automatically fixes the direct particle’s time of flight.
	•	Knowing its displacement and its flight time gives you its momentum (via $p = mL/\Delta t$).

So in principle, you can reconstruct the momentum of the first particle indirectly, using the timing provided by the second one. \cop{So, by comparing arrival times, you can infer when both particles must have left the shutter.}

%(1) determining the value of a quantity at a time immediately before the measurement; and (2) preparing a state at a time immediately after the measurement. The standard reading of the uncertainty relations – and the one ETP presuppose – is as a limitation on simultaneous state preparation: there are no measurements such that the prepared state
%allows one to exactly predict the results of further measurements of either of two
%conjugate quantities, despite the fact that scenarios like the photon box suggest that
%a system simultaneously possesses values for such quantities However, what was
%generally supposed – an assumption ETP aimed to correct – was that retrodictions
%of values of conjugate quantities were possible, even though physically meaning-
%less because devoid of predictive power. This is the line taken by Heisenberg in
%the Chicago lectures, for instance (Heisenberg 1930b, p. 20)


Berlin, 4 November 1931.

\q{Imagine a box with a shutter that opens and closes automatically, which is further equipped with a clock; upon opening of the shutter, let a monochromatic light ray be emitted of about 100 wave trains, and let it be reflected by a mirror placed at a known distance (many light years) and return to an observation point. The energy (colour) of the emitted light ray can be determined through weighing before and after the emission of the ray, the time of emission through the clock. EINSTEIN shows in this very ingenious thought experiment that it is not possible to predict with the aid of measurements both the
colour as well as the time of arrival of a light ray at the observation point. Only
one measurement – of the time or the colour – can be carried out precisely, and
in fact according to EINSTEIN one can still decide after the departure of the light
ray which of the two predictions one wants to choose. – The American physicist
TOLMAN has further extended this thought experiment in such a way that one can
additionally show that also for the past one can make only one of the two statements
precisely.}

it shows that if we were able to reconstruct the past of one
particle in a way that violates the constraints of uncertainty, we would be able
to make simultaneous predictions about the other particle that also violate uncer-
tainty. This is the new loophole ETP find – and then close off – using their thought
experiment. The paper does not really side with Bohr (nor with Hermann), in that
ETP tacitly assume throughout (e.g. by talking about disturbing the momentum of
the particle) that the limitations imposed by uncertainty – and also on the possi-
bility of retrodicting the energy and time of release of the particles – are purely
epistemic limitations


In the summer of 1933, Einstein visited Belgium on his way to the United States (fig. 3). During his stay, he delivered three lectures in Brussels on his own version of the spinor, the “semi-vector” (van Dongen Reference van Dongen2004), in front of the elite of Belgian physics. When asked which members of his audience could understand the content of his lectures, Einstein supposedly answered: “Perhaps professor De Donder … Lemaître certainly, and the others I do not think so” (cited in Mawhin Reference Mawhin2012, 54-55).  In 1933 Rosenfeld gave a lecture in Brussels based on a joint paper with Bohr, and Einstein was in attendance. After the lecture, Einstein described to Rosenfeld a thought experiment involving no boxes, only two interacting particles – that is, an experiment nearly identical to that of EPR. Here is Rosenfeld’s recollection of
Einstein’s words on that occasion (Rosenfeld 1967, pp. 127–128):

What would you say of the following situation? Suppose two particles are set in motion towards each other with the same, very large, momentum, and that they interact with each other for a very short time when they pass at known positions. Consider now an observer who gets hold of one of the particles, far away from the region of interaction, and measures its momentum; then from the conditions of the experiment, he will obviously be able to deduce the momentum of the other particle. If, however, he chooses to measure the position of the first particle, he will be able to tell where the other particle is. This is a perfectly correct and straightforward deduction from the principles of quantum mechanics; but is it not very paradoxical? How can the final state of the second particle be influenced by a measurement performed on the first, after all physical interaction has ceased between them?

When Einstein arrived at Princeton in October 1933, 

%%\cop{In 1933 Rosenfeld gave a lecture in Brussels based on a joint paper with Bohr,
%%and Einstein was in attendance. After the lecture, Einstein described to Rosenfeld
%%a thought experiment involving no boxes, only two interacting particles – that is,
%%an experiment nearly identical to that of EPR. Here is Rosenfeld’s recollection of
%%Einstein’s words on that occasion}, Hower, that 1934 paper by Ppper.


\section{1934-1945}

That Eiisnste, since the past paricel, that one can use two paril two riag, tir; Howeve,r intersti, even straw man the ncertaintu reatliosn. 1933 proab 1934, that to his unce, on probabilti, and one on quantum mechanics. For the sake of philosophy, Popper took remarkable risks. By Springer’s deadline of March 1, 1934, he was still wrestling with the problem of probability and had managed only to settle on a new title for the book, *Logik der Forschung*. His chapter on quantum physics existed only in a rudimentary form. When he submitted the manuscript on April 9, leaving a copy at Schlick’s apartment, he was far from ready. He was still working out the consequences of his new concept of probability for falsifiability and testability.  

Fortunately, Schlick, then recovering from influenza in Italy, was away. Ten days later Popper retrieved the manuscript, only to return it on May 9. This was not the end of the process. Springer kept the manuscript for some time, while Popper pressed on with his work on quantum physics, consulting Franz Urbach (1902–1969), Urbach’s colleague in Zurich, Viktor Weisskopf (b. 1908), and his cousin Käthe Schiff (b. 1909). By June or early July, Springer returned the manuscript with the demand that it be reduced by one-third.  

This gave Popper the chance to revise not only the chapter on physics but the entire book—though by then he was utterly exhausted. He later reported that his uncle Walter Schiff undertook much of the editing. A glance at the contents shows where the major cuts were made. The technical chapters endured virtually no abbreviation—indeed, Popper even added new material. The longest chapters, on probability and quantum physics, received his most concentrated effort, raising the question of when he found time to revise the earlier sections. The prolix arguments of *Grundprobleme* were now recast into concise, fast-moving, sharply argued passages. The pressure of space shaped this new style, but more importantly, it marked the moment Popper truly discovered his voice.

This communication contains a discussion of the same thought experiment that is treated in more detail in *The Logic of Scientific Discovery*, section 71.7. It is dated 27 August 1934 and was published on 30 November 1934. Before publication, Dr. Arnold Berliner (1862–1942), the editor of *Die Naturwissenschaften*, sent a galley proof to Werner Heisenberg (1901–1976) in Leipzig. After publication, Popper undoubtedly sent out some offprints. A manuscript of this first communication could not be located, but it was included in the book.

\todo{Siehe >Popper-Archiv< (z.B. Fasz. 292,12: >Albert Einstein<; Fasz. 293,29: >Hans Euler•; Fasz. 305,32: >Werner Heisenberg•; Fasz. 360,21: >Viktor Weisskopf<; und Fasz. 360,22: •Carl Friedrich von Weizsäcker [und Werner Heisenberg]•). }


\subsection{Oppoer crqiuete }

Viennese interpation (Frank von Mises).

In Popper’s reading, the so-called Heisenberg uncertainty relations are nothing more than statistical dispersion relations. They describe the spread of values in an ensemble of similarly prepared systems, rather than fundamental restrictions on the properties of individual systems. As such, they do not impose any ultimate limit on the precision of measurements of a given quantity in a single physical system. The common belief that they do so stems from an auxiliary hypothesis that is not derived from the formalism of quantum theory itself. In fact, one can imagine—and Popper explicitly did—thought experiments that undermine this additional assumption.

This perspective leads directly to a critical distinction. Heisenberg’s original version of the uncertainty principle referred to the errors in simultaneous measurements of position $q$ and momentum $p$ on an individual system. In this formulation, measuring one observable inevitably disturbs the accuracy of the other. Popper’s alternative version, by contrast, referred instead to the statistical dispersions observed when one considers many measurements performed on an ensemble of systems prepared in the same way.

The upshot of Popper’s analysis is that the current, Heisenberg-style interpretation of the uncertainty principle cannot be strictly derived from the quantum formalism itself. Rather, if the principle is to be given a foundation, it must be seen as a statement about the statistical properties of ensembles, not as a universal law restricting what can be measured in an individual quantum system.

\q{predictive measurements and physical selections are inseparably linked.}. To show, that are predictie meusare that are not prepartions 

§76 auxiliary hypothesis: connection between preparation (selection) and measurement  
\begin{itemize}
\item  item preparation refers to the future, that is, to any procedure that will produce a statistically reproducible set of systems (filtering).   
\item measurement refers to the past and involves the detection of a particular system by a measuring instrument (recording).  
preparation $\neq$ measurement.
\end{itemize}


Popper: by virtue of the conservation laws, from a non-predictive measurement of the path of a particle (particle A) one obtains a predictive measurement of the path of its partner (particle B), with which it had collided.


\section{Weizacker}


Von Weizsäcker argues that Popper’s proposed experiment does not allow one to circumvent the uncertainty principle. The key point is that from measurements made at location X, one cannot determine both the position and the exact time of the collision at S with arbitrary precision. To see this, one must look closely at the actual measurement process in X.

Suppose one first measures the momentum of particle A, and then later its position at a chosen location.

Measuring the position destroys knowledge of its momentum after the measurement. However, from the momentum measured before the position check, one can reconstruct a “trajectory” of the particle leading up to the collision.

This is likely what Popper meant by a “non-predictive measurement” (i.e. one that tells us about the past, not the future).

But, von Weizsäcker stresses, this reconstructed trajectory is in principle uncontrollable: it only holds during the short interval between the end of the momentum measurement and the beginning of the position measurement, where the particle is assumed not to interact with anything else. It cannot be extended backwards before the momentum measurement, because that very measurement has disturbed the particle’s position according to the uncertainty relation.



Furthermore, if the momentum is measured by a scattering process (e.g. via the Doppler effect of light quanta or electrons), the measurement itself takes a finite time. During this time:

The frequency of the scattered photon must be determined, which requires a non-zero measurement duration.

Consequently, the exact time of the scattering cannot be sharply fixed.

Because the collision changes the particle’s velocity, the average velocity over this measurement period cannot be precisely known.



Thus, even if one determines the particle’s position accurately after the momentum measurement, one cannot reconstruct its position before the measurement more precisely than within the bound set by the uncertainty relation, namely an error of about \hbar / (4\pi \Delta p).



The conclusion: our knowledge of the trajectory of particle A before the momentum measurement—and hence of particle B’s trajectory after the collision at S—remains perfectly consistent with the uncertainty relations.



Finally, von Weizsäcker generalizes: this proof only covers one specific measurement setup, but there is no reason to think other apparatuses would fare differently. Popper introduces a contradiction with quantum mechanics because he fails to distinguish clearly between (1) his so-called “non-predictive measurements,” which are not physically realizable measurements, and (2) testable, empirical inferences about the past. The uncertainty principle does not apply to the former simply because they do not describe possible physical measurements. But for any testable inference about the past, the same accuracy limits hold as for predictions about the future, since quantum laws are symmetric with respect to time.



The experiment proposed by Mr. Popper does not, upon correct discussion, allow any violation of the accuracy limit given in the uncertainty relation. It is not possible to determine exactly from the measurements in $X$ the location and time of the collision in S. To see this, however, it is necessary to make precise statements about the type of measurements in X. For example, suppose first the momentum of particle A is measured, and then the moment of its arrival at a given location. The position measurement indeed destroys the knowledge of the momentum after the collision; nevertheless, from the momentum before the collision and the position one can construct a “trajectory” of the particle prior to the collision. This fact is presumably what Mr. Popper intends to express with the notion of a \s{non-predictive measurement.} Yet this trajectory is in principle uncontrollable. 

It applies only to the time interval between the end of the momentum measurement and the beginning of the position measurement, during which the particle has no interaction at all with its environment, and it cannot be extended into the period before the momentum measurement, since this measurement itself destroys knowledge of the position in accordance with the uncertainty relation. If the momentum, for example, is to be measured by a collision process (Doppler effect) — a Geiger counter is not suitable for momentum measurements — the measurement requires a finite time, otherwise the frequency of the colliding photon or electron cannot be exactly defined, and thus the momentum transferred in the collision remains partly indeterminate. Consequently, the time of the collision cannot be precisely determined, which means, because of the change in velocity during the collision, that the mean velocity during the duration of the momentum measurement is not exactly known. 

If, therefore, the position after the momentum measurement is exactly known, one still cannot infer the position before the momentum measurement more precisely than with an error of order $\hbar/ \Delta p$. This means that our knowledge of the trajectory of particle A in X before the momentum measurement, and thus also the trajectory of particle B after the collision in S, is consistent with the uncertainty relation. This demonstration, of course, applies only to a particular measuring arrangement. But we have no reason to doubt that any other combination of measuring devices in $X$ would yield the same result. For Mr. Popper, in his general discussion of the experiment, introduces a contradiction with quantum mechanics by failing to distinguish between his so-called “non-predictive measurements” and verifiable measurements about the past. The uncertainty relations cannot be applied to “non-predictive measurements” only because the statements in which their results are reported contain no claims about physically possible measurements at all; verifiable inferences about the past, on the other hand, are subject — owing to the symmetry of the quantum-mechanical laws with respect to the direction of time — to the same accuracy limitations as verifiable inferences about the future.


The author is arguing:
	•	To reconstruct a trajectory, you need both momentum (before collision) and time of flight.
	•	Measuring momentum precisely requires a finite time (e.g. via Doppler scattering). That time window makes the exact collision moment fuzzy.
	•	If you try to sharpen the time of flight, the energy (hence momentum) becomes uncertain — this is the time–energy uncertainty trade-off.
	•	So you can’t have both simultaneously:
	•	precise momentum and
	•	precise collision time / time of flight

without violating the uncertainty relation.

One way to measure the momentum of a particle is to scatter light (or another probe, like electrons) off it and look at the frequency shift of the scattered radiation — that’s the Doppler effect.
	•	The amount of frequency shift encodes the particle’s velocity (and hence momentum).
So, in principle, you can learn about the particle’s momentum from the Doppler shift of a scattered photon.

To get accurate momentum (via Doppler frequency), you need long measurement time, which smears the collision time.
	•	To get accurate time of flight, you need a sharply localized collision time, which means poor frequency (momentum) resolution.


Would you like me to also phrase this in more physical terms with a simple analogy (e.g. listening to a violin note: the shorter you listen, the less precisely you know its pitch → same logic for particle’s Doppler shift)?

That’s the heart of the argument: momentum precision (via frequency) and temporal precision (event time) can’t both be sharp, which is just the time–energy uncertainty principle in disguise.


The present \s{supplement} was apparently written after Popper learned that the first communication had been accepted for publication by Dr. Arnold Berliner, the editor of the journal *Die Naturwissenschaften*. Since it makes no reference to any criticisms of his thought experiment, it seems to have been written before Popper became aware of the objections raised by Werner Heisenberg and Carl Friedrich von Weizsäcker.   It is possible that this “supplement” was merely intended as an enclosure to accompany the offprints of the first communication; but it is also possible that Popper hoped *Die Naturwissenschaften* would publish it as well. In any case, Popper seems to have forgotten the text: he never mentions it in any of his other surviving essays on quantum mechanics, nor is there any trace of it in the preserved correspondence. After learning of the criticism of his thought experiment, Popper concentrated on preparing a reply, and in the process he likely forgot about the “supplement.”

\section{Reply From the Leibiz g Rpg}


\section{Hermann}



\section{Einstein and Popper}

%Philipp Frank to ()tto Neurath, rnid- to late July 1935~ t"-·o lcttcTS, ,'\.'eumt/1 J'11tt/Jl,~/J.
%Viktor Wei..,.._kopf to J>opp~r. 21 janu,try 1935, Popper Archives (360, 21). Later, he rene\ved their discussiom but was critical (24 Octobt."r 19.H).

%Draft of response to Wdzülkker, dated 6 December 1934, Popper Archives (305, 32). 


Euler to Popper, 4 February 1935 (291, 29). Popper drafted, but probably did not send, a furious response on February 14, then another on March 12. Heisenberg sent a final letter on 19 March 1935 (305, 32).

Popper to Carnap. 10 Junt." 1935.
Carnap Collins; l'oppt"r to Richard \'Oil Miscs, 26 June 1935, Popper Archiws 1329· 4j.)

He sent "Zur Kritik der Ungenauigkeitsrelationen" to Carnap, Ehrenhaft, Heisenberg, and
Schrödinger, hoping that Ehrenhaft would forward it to Einstein. (Popper to Carnap, 10 June 1935, Carnap Collection; Popper to Richard von Mises, 26 June 1935, Popper Archives 1329-4.)


%Submission date: March 25, 1935
%Publication date: May 15, 1935



Einstein to Popper, 15** June 1935, Popper Archives (292, 12). 
Kaufmann to Popper, 12 July 1935, Kaufmann Nachlass.
Popper to Einstein, 18 July 1935, Popper Archives (292, 12).

%Am 18. Juli 1935 antwortete Popper Einstein mit einem sehr langen (8 Seiten), maschinengeschriebenen Brief, gegen dessen Ende hin (auf Seite 6) er schrieb: »Um diesen Brief nicht zu lang werden zu lassen, möchte ich die Diskussion ... hier abschließen. Ich erlaube mir aber, Ihnen zur Ergänzung noch eine kleine Schrift (Manuskript) >Zur Kritik der Ungenauigkeitsrelationen< zu schicken, die ich vor etwa zwei Monaten auf Anregung von Professor Ehrenhaft für Sie, verehr- ter Herr Professor, geschrieben habe; diese stellt die Frage in einem anderen Zusammenhang und, wie mir scheint, unter Vermeidung der Fehler meines Buches dar. Ich habe bei diesen Zusendungen freilich ein schlechtes Gewissen; vielleicht nehme ich Ihre Zeit für nichts und wieder nichts in Anspruch!«

%Popper to Einstein, 29 August 1935 (from Galtür in Tyrol) (292, 12). 

Hochverehrter Herr Professor!
In meiner Antwort auf Ihren Brief vom 15.6. kündigte ich Ihnen die Über-
sendung eines Manuskripts an (•Zur Kritik der Ungenauigkeitsrelationen<),
das ich über Veranlassung von Prof. Ehrenhaft schon vor Monaten mit der
Absicht, es an Sie zu schicken, verfaßt habe.
Vor der Absendung kamen mir nun allerlei Bedenken, und ich dachte
schon, daß es mir gelingen werde, meine Auffassung selbst zu widerlegen.
Dann hatte ich in diesem Monat Gelegenheit, mit dem Assistenten von Prof.
Pauli, Viktor Weisskopf26
, zusammenzukommen, und ich bat ihn, mir bei
dieser Widerlegung behilflich zu sein, - unter Hinweis darauf, daß ich Ihnen
nicht gern etwas schicken möchte, was ich sozusagen im eigenen Wirkungs-
kreis erledigen kann. Wcisskopf ist natürlich von vornherein ziemlich über-
zeugt gewesen, daß die Sache nicht stimmen kann; trotzdem sind wir in
langen und ausführlichen Diskussionen zu keiner stichhältigen Widerlegung
gekommen (wir setzen die Diskussion noch schriftlich fort).
Da ich Ihnen nun das Manuskript schon so lange angekündigt habe, bleibt
mir wohl nichts anderes übrig, als es Ihnen trotz der ungeklärten Situation zu schicken, denn ich habe die Hoffnung auf eine b.1ldigc und \'ollst:inJige
Klärung ziemlich aufgegeben.
Weisskopf erzählte mir, daß in letzter Zeit von Ihnen, hoch\'crchner l lerr
Professor, eine Arbeit über ein verwandtes Thcm.1 erschienen ist. kh lube
mir leider bis jetzt diese Arbeit nicht verschaffen können und mul~ Sie deshalb
ganz besonders um Entschuldigung bitten, denn es ist wohl l'inc Zumutung,
sich an jemanden mit einer frage zu wenden, ohne sich vorher vergewissert
zu haben, daß dieser die Antwort nicht schon längst gegeben hat.
Ich bitte Sie, mir deshalb und wegen der Verzögerung der Zusendung nicht
böse zu sein!
Ich danke Ihnen nochmals sehr für Ihren lieben Brief.



\subsection{Popper NEw Thag}

Imagine a device that cannot resolve short light pulses, but only reacts to infinitely long, monochromatic wave trains. Such devices should exist, because time and frequency behave just like space and momentum. We already know of devices like gratings that sort momentum more sharply when the beam is spatially broad. By symmetry, there should be devices that sort frequency more sharply when the wave train is temporally long.”

\subsection{**}


Am 11. September 1935 schreibt Albere Einstein2
c an Popper, daß er seine Abhandlung angesehen hat und daß er weitgehend überein-
stimmt. Diese Abhandlung wird hier zum ersten Mal ver()ffcntlicht -
und zwar in der Fassung, die Popper am 29. August an Einstein
geschickt hat.

\q{I have read your article, and I largely agree with you, X. Only, I do not believe in the possibility of producing a \s{super pure case} that would allow us to predict the position and the momentum (color) of a photon with \s{inadmissible} precision. I consider the means you propose (a screen with a fast shutter combined with a selective set of glass filters) to be ineffective in principle, for the reason that I firmly believe that such a filter would act in such a way as to \s{smear} the position, just as a spectroscopic grating would do.  \\

My reasoning is as follows. Consider a short light signal (precise position). In order to see more easily the effects of an absorbing filter, I assume that the signal is analyzed into a very large number of quasi-monochromatic wave trains. Suppose that the set of absorbing filters cuts out all the colours $W_n$ except one, $W_1$. Now, this group of waves will have a considerable spatial extension (\s{smearing} of its position) because it is quasi-monochromatic; and this means that the filter will necessarily \s{smear} the position.}



\section{1936}

\q{Ich hatte die ganze Zeit seit dem Erscheinen meines Buches schriftliche und mündliche Diskussionen mit verschiedenen Physikern über die Interpretation der Quantenmechanik, vor allem mit Heisenberg und seinen Leuten, aber auch mit Einstein. Unter dem Eindruck dieser Diskussionen habe ich meine Auffassung korrigiert, - wobei jedoch einige wichtige Punkte meiner früheren Auffassung erhalten bleiben (rein statistische Interpretation der •Hcisenbergschcn Formeln<, die Ungenauigkeitsbehauptung hat nicht die Form eines physikalischen Satzes, bzw. eines Naturgesetzes). Diese modifizierte Auffassung durfte ich in Kopenhagen Bohr vortragen und dieser hat zu meiner großen Freude ohne Einschränkung zugestimmt. Bohr hat einen ganz großen Eindruck auf mich gemacht. Er drückt sich manchmal nichts weniger als klar aus - aber bei einiger Mühe ist es doch mög- lich, sich zu verständigen. Und es ist mehr als diese Mühe wert!}



%%The real reason for this error, apparent only from the more detailed discussions in Popper’s book Logik der Forschung (Vienna 1935 [sic]) [Popper, 1934], lies in a misjudgment of the duality experiments and their consequences. Popper is misled by the probability interpretation of the wave functions to apply these quantum mechanical state descriptions, and the uncertainty relations given with them, only strictly speaking to ensembles of physical systems, and for a single appropriately chosen system to assume instead no restriction through the uncertainty relations. In this he misunderstands that because of the duality experiments the applicability of the classical conceptions is lim- ited according to the uncertainty relations already for every single elementary process, and that accordingly wave functions can in fact be used for describing the state of individual systems. 
%
%
% 
%
%\section{Ciao}
%
%
%
%%EPR propose a pair of particles (say, A and B) that interact and then separate. They are described by a joint wave function with perfect correlations:
%%
%%Position correlation: If you measure the position of particle A, you can immediately infer the position of particle B with certainty.
%%
%%Momentum correlation: If instead you measure the momentum of particle A, you can immediately infer the momentum of particle B with certainty.
%%
%%
%%Now, suppose the particles are far apart and no longer interacting:
%%	•	If I measure position of A, I can predict position of B with certainty.
%%	•	If I measure momentum of A, I can predict momentum of B with certainty.
%%
%%By EPR’s criterion of reality:
%%
%%If, without disturbing a system, we can predict with certainty the value of a physical quantity, then there exists an element of physical reality corresponding to this quantity.
%%
%%So both position and momentum must simultaneously be elements of reality for particle B.
%
%
%
%%
%%
%%%%If you try to interpret predictive measurements as “mere” observations, you miss that they inevitably function as state preparations. That means they change the system, and they cannot be separated from the quantum formalism (uncertainty relations, Fourier trade-offs, etc.).
%%%
%%%
%%%%~ccording to Professor Nathan Rosen, it might have been possible that Popper influenced
%%%%Einstein (private communication, April 22, 1967); but according to Mrs. Polly Podolsky, who
%%%%kept in close touch with her (in 1966 deceased) husband's work, it was unlikely that Popper's
%%%%work could have reached Einstein before the first draft of the Einstein-Podolsky-Rosen paper
%%%%was written (letter from Mrs. Polly Podolsky, dated August 1, 1967)
%%%
%%%
%%%Shortly after November 30, 1934, when the paper was published, Popper,
%%%encouraged by Einstein's friend the violinist Adolf Busch whom Popper
%%%knew through Busch's son-in-law, Rudolf Serkin, sent an offprint of the
%%%article together with a copy of his just published book, Logik der For-
%%%schung, to Einstein for comments.
%%
%%\section{1935 Manuscritp}
%
%
%Old Lyme, i i settembre '35 

%\section{Einstein}


 \section{Conclusion}
 
1. preparation and measurement: When carefully considered, his predictive measurements turn out not to be measurements at all but preparations of states, for they
lack the one element which makes an operation a measurement: the emergence of a number, always seletion. Thus, a preparation is always a matter of choosing or filtering outcomes from a larger ensemble. he act of selecting an outcome prepares the system in the eigenstate associated with that outcome. But, crucially, the preparation always involves selection — whether by blocking unwanted outcomes, or by conditioning on a particular measurement result. distinguishes “preparations” (procedures that yield ensembles in reproducible states) from “tests” (procedures that yield outcomes).

Thus, the same physical device can serve either as a preparation (via selection) or as a measurement (via detection), depending on how it is used.

2. instead of blocking, you put a detector at the end of each channel to record how many went up and how many went down. That’s measurement. There are no predictive measurements in the strict sense — what we call “predictive measurements” are really preparations. Every predictive act is a preparation, and all preparations are selections.

%Every preparation implies predictions.
%	•	But predictions are only tested (never “completed”) by measurement on an ensemble.

2. hyptheis of linkange is true: there are no predictive measuremtn that are not preparations;  



\q{For example, when electrons are made to pass into a magnetic field, a new state with
respect to electron spin has been produced, but the spin has not
been measured } \q{The correct statement therefore is that the uncertainty relations
reflect the interaction between system and apparatus when the state
is being prepared, not when it is measured.}



%retroactive measurements might not be preparations, but all predictive measurements are preparations
  
  
%The time-asymmetry in quantum mechanics beween retrodiction and prediction comes from the asymmetry between state preparation and measurement.



\citetrackerfalse

\printshorthands
\printbibliography

\end{document}