\documentclass[12pt,draft]{article}

\newcommand{\LdF}{\bt{Logik der Forschung}\xspace}
\newcommand{\Lo}{\bt{Logik}\xspace}
\newcommand{\KPA}[1]{KPA #1}
\usepackage[font=scriptsize]{caption}

\usepackage{els}
\usepackage{notate}
\usepackage{ii}
\usepackage{quantum}

\newcommand{\TE}{thought experiment\xspace}
\newcommand{\IR}{indeterminacy relations\xspace}
\newcommand{\UR}{uncertainty relations\xspace}
\newcommand{\HR}{Heisenberg's relations\xspace}
\newcommand{\Weis}{Weisskopf\xspace}
\newcommand{\vW}{von Weizsäcker\xspace}
\newcommand{\VW}{Von Weizsäcker\xspace}
\newcommand{\CH}{coupling-hypothesis\xspace}

\newcommand{\NW}{\jt{Die Naturwissenschaften}\xspace}
\newcommand{\ERG}{\jt{Erg\"anzung}\xspace}


\newcommand{\KH}{\german{Kopplungshypothese}\xspace}
\newcommand{\GE}{\german{Gedankenexperiment}}
\newcommand{\LF}[1]{\citep[#1]{Popper1935}}

\renewcommand{\pos}{\ensuremath{x}\xspace}
\renewcommand{\mom}{\ensuremath{p_x}\xspace}
\newcommand{\pam}{\ensuremath{x} and \ensuremath{p_x}\xspace}


\begin{document}

\title{The Past of an Electron: Young Popper's \GE\ Against the Indeterminacy Relations}
\maketitle

\begin{abstract}

Between 1934 and 1936, Popper composed at least eight manuscripts on the interpretation of quantum mechanics, proposing a \german{Gedankenexperiment} in which reconstructing the past trajectory of an electron would, by invoking momentum conservation, permit the prediction of both position and momentum with arbitrary sharpness—thus apparently \s{beating} the \IR. During this period, Popper engaged in intense correspondence with leading physicists such as Weisskopf, Heisenberg, von Weizsäcker, and Einstein, who sought to expose the flaws in his reasoning. In retrospect, Popper would later describe these efforts as \q{a gross mistake}. Still \citet{Jammer1974} suggested that Popper’s mistake may have been instrumental in the formulation of the 1935 Einstein--Podolsky--Rosen argument, while \citet{Margenau1974} credited him with anticipating the later distinction between preparations and measurements. Drawing on unpublished material, I argue that both claims are questionable. Popper’s thought experiment is more closely related to the 1931 Einstein--Tolman--Podolsky setup, though he reached the opposite conclusion precisely because he misunderstood the relation between preparation and measurement.
\end{abstract}

%The same misunderstanding also plagues Popper's post-war work on the topic \citepp{Howard2012}{Shields2012}{Freire2014}{DelSanto2017}{DelSanto2019}. 






%I argue that these manuscripts are better understood as an early attempt to formulate a \emph{coupling hypothesis}---namely, that preparations and measurements are conceptually inseparable, and that every measurement must be seen as relative to a prior preparation.%Recent scholarship has largely dismissed this episode as a youthful blunder, focusing instead on Popper's post-war work on quantum mechanics \citepp{Howard2012}{Shields2012}{Freire2014}{DelSanto2017}{DelSanto2019}. 


\intro

Between 1934 and 1936, Karl R.~Popper composed at least eight manuscripts revolving around a \german{Gedankenexperiment} to challenge the widespread interpretation of the so-called indeterminacy or uncertainty relations, which, in quantum mechanics, were thought to limit the joint \emph{measurement} of conjugate quantities such as position and momentum, or time and energy:

\begin{enumerate}[itemsep=0pt, parsep=0pt, topsep=4pt, partopsep=0pt]
  \item \textit{\ul{Zur Kritik der Ungenauigkeitsrelationen}} --- August 1934 \citep{Popper1934}
  \item \textit{Ergänzung zu der vorstehenden kurzen Mitteilung} --- before November 1934 \citep{Popper1934a}
  \item \textit{\ul{Bemerkungen zur Quantenmechanik}} (chapter~2.VII of \cite{Popper1935}) + Appendix~V,~VI,~VII --- Fall 1934
  \item \textit{Zur Kritik der Ungenauigkeitsrelationen, 2.~Mitteilung [A]} --- December 1934 \citep{Popper1934b}
  \item \textit{Zur Kritik der Ungenauigkeitsrelationen, 2.~Mitteilung [B]} --- December 1934 \citep{Popper1934c}
\item \textit{\sout{Erwiderung auf die Kritik Heisenberg--Weizsäcker}} --- late 1934 to early 1935
  \item \textit{\sout{Meßanordnung}} --- February 1935
  \item \textit{Zur Kritik der Ungenauigkeitsrelationen [1935]} --- August 1935 \citep{Popper1935b}
 \item \textit{Bemerkung zum Komplementaritätsproblem der Quantenmechanik} --- 1936 \citep{Popper1936a}
\end{enumerate}

Only 1 and 3 were published during Popper's lifetime. Numbers 6 and 7 have been lost, although 6 may in fact be identical with 5. The remaining material was published only recently in a collection of Popper’s early writings \citep{Popper2006}. While drafting these texts, young Popper—at that time a schoolteacher with only a single short publication to his name \citep{Popper1932/1933}—engaged in extensive correspondence with leading quantum physicists: (1) Victor Weisskopf (then assistant to Wolfgang Pauli in Zürich), (2) Heisenberg and his assistants Carl Friedrich von Weizsäcker and Hans Euler, and (3) Einstein. Some of this correspondence is preserved at the KPA and remains unpublished. Moreover, Popper discussed his ideas in person with Viennese physicists he could get a hold of: Franz Urbach, his cousin, the crystallographer Käthe Schiff; Hans Thirring\footnote{According to Popper's recollection, Thirring invited Popper in 1934 (possibly 1935) to his seminar; see \citet[1125]{Schilpp1974}} and his assistant Eugene Guth, as well as Felix Ehrenhaft. Naturally, \latin{verba volant}: we have little insight into the concrete content of these exchanges.

The late 1920s and early 1930s saw a proliferation of \german{Gedankenexperimente} as tools to probe conceptual issues \emph{outside} the quantum-mechanical formalism---to show its consistency or to disprove it. \citet{Heisenberg1927a} formulated his famous $\gamma$-ray microscope \TE\ to show that the \UR\ reflect the mutual disturbance that arises from the experimental determination of position and momentum. By contrast, Einstein was regarded by his contemporaries as having devised \german{Gedankenexperimente}, such as the light quantum box \TE, to challenge the \UR. Young Popper, on the contrary, accepted the mathematical \emph{derivation} of the \UR\ from the quantum formalism; he devised a \GE\ intended to expose the tension between two different \emph{interpretations} of the \UR. In Popper’s view, the \IR\ impose a limitation on the \emph{physical selection} of an ensemble of systems in which both position and momentum are sharply defined; however, \emph{pace} Heisenberg, they do not impose a limitation on the accuracy of \emph{measurements} of position and momentum on a single system. 

Popper could exploit the fact that quantum physicists generally conceded that \emph{non-predictive measurements} of arbitrary sharpness are permitted by the \UR; nevertheless, they typically denied that \emph{predictive measurements} were. According to Popper, the reason is that they tacitly assumed what he called the \emph{\CH} \origg{\KH}\footnote{\citet[238\psq]{Popper1959} later adopted the translation \s{hypothesis  of linkage,} possibly because \s{coupling} in English typically implies a physical interaction. However, I prefer to retain a rendering closer to the original German, which better reflects the resonance of Popper's terminology with contemporary physical discourse} between \s{physical selections} and \s{predictive measurements}. Popper, however, thought he could show that the \CH\ was inconsistent with the formalism of \qm. He devised a \TE\ based on a particle collision in which, starting from a \emph{non-predictive measurement} reconstructing the past trajectory of a particle, one could, by invoking momentum conservation, calculate arbitrarily sharp \emph{predictive measurements} on another particle after the collision. In retrospect, Popper himself would later describe his \TE\ as \q{a gross mistake for which I have been deeply sorry and ashamed ever since} \citep[15\hide]{Popper1982}.

\s{Young Popper} has long been a well-established \s{scholarly field} \citep{Hacohen2004}. Nevertheless, the historical literature on Popper's interpretation of \qm\ \citepp{Howard2012}{Shields2012}{Freire2014}[sec.~6]{Maxwell2016}{DelSanto2017}{DelSanto2019} mentions this episode only in passing, ultimately dismissing it as a youthful blunder and focusing instead on Popper’s renewed engagement with the theory after the war \citepp{Popper1967,Popper1982}. Yet early commentators writing during the height of Popper’s fame gave considerably more weight to his engagement with \qm\. Max~\citet{Jammer1974} conjectured that Popper's mistaken thought-experiment was at least instrumental in inspiring the 1935 Einstein--Podolsky--Rosen (EPR) thought experiment \citep{Einstein1935}. Indeed, according to Jammer, Popper’s setup bore a striking resemblance to the EPR setup, pointing to an experiment based on correlated systems. Henry~\citet{Margenau1974} credited Popper with anticipating the distinction between \s{preparation} (that Popper called physical selections) and \s{measurement} he proposed at around the same time \citep{Margenau1936}{Margenau1937} and which later became standard. In particular, Popper was one of the first philosophers to appreciate that \IR expressed a \s{preparation uncertainty} rather than a \s{measurement uncertainty} \citep[see][72\psq]{Popper1974}.

This paper, based on little-known and unpublished material from the Karl Popper Archives (KPA), demonstrates that, upon closer examination, both claims fail to withstand critical scrutiny. Popper himself was cautious \q{gross mistake made by a nobody (like myself)} could have influenced Einstein, and only suggested that \q{from a purely temporal standpoint} it cannot be ruled out (\letter{Popper}{Jammer}{13}{4}{1967}, \citep[178\fn{30}]{Jammer1974}). However, it is precisely the chronology that seems to exclude any influence of Popper’s thought experiment on the emergence of the 1935 EPR argument. What can be said is that Popper independently formulated a variant of the lesser-known 1931 Einstein--Tolman--Podolsky (ETP) thought experiment \citep{Einstein1931-03}, though he arrived at the opposite conclusion. The paper argues that the reason was ultimately that Popper deeply misunderstood the relation between \s{preparation} and \s{measurement} that he allegedly anticipated \citepp[see][sec.~6]{Maxwell2016}. Popper's notion of \s{physical selection} is akin, but still not on par with the modern notion of \s{preparation}.  In particular, the \emph{coupling hypothesis} between preparations and measurements is not an \s{additional hypothesis}, ultimately contradictory with rest of the formalism, as he argued, but a central feature of \qm\ that distinguishes it from classical physics.

\todo{Redhead1995}

Against the background of the 1930s debate on the past trajectory of particles in \qm\ (\cref{sec:past}), the paper reconstructs the emergence of Popper's \TE\ (\cref{sec:popperquantun,sec:interpreation,sec:TE}) and the criticisms of Heisenberg and his Leipzig group (\cref{sec:leipizg}) as well as Einstein (\cref{sec:einstein}). Popper ultimately, but only after considerable resistance, conceded that his \TE\ was a \s{gross mistake}. Although Popper fully understood the fundamental flaw his experimental arrangement, the paper argues that he did fully appreciated the conceptual reason behind it. The failure of the \TE\ should have convinced Popper of the validity of the \CH\ between preparations and measurements he aimed to challenge. On the contrary, Popper concluded that it was the particular kind of ETP-like setup he had chosen to challenge the \CH\ that was at fault. Thus, \textcites{Popper1982}{Popper1982b}{Popper1985}{Popper1986}, Popper attempted to replace the ETP-like experiment with an EPR-like one. 

%However, this setup too was once ultimately again based on a \s{gross mistake}.


\section{The Past of an Electron}
\label{sec:past}

\subsection{The Heisenberg-Schlick Dispute}
\label{sub:heisenbergschlick}

In the spring of 1929, Heisenberg delivered a series of lectures at the University of Chicago on quantum mechanics. A written account of the lectures was completed in March 1930 and appeared later that summer in print, in both English and German editions \citepp{Heisenberg1930}{Heisenberg1930a}. In presenting the \IR, Heisenberg takes the opportunity to address, although rather in passing, the vexed question of the retrodictability of sharp joint values of position and momentum variables of, say, an electron. As already pointed out in \citets{Bohr1928} Como lecture, if one performs two position measurements at subsequent times, then from the time-of-flight one can in principle reconstruct the electron momentum with arbitrary sharpness between these two measurements\footnote{See also \letter{Ehrenfest}{Goudsmit, Uhlenbeck, and Dieke}{3}{11}{1927}}%
%
\footnoteh{In a famous 1927 letter to his students, \Ehr\ refers to this question: \q{The uncertainty principle does not forbid calculating momentum from past data. It limits the precision with which you can simultaneously measure position and momentum in a single act of observation. Furthermore, one also notices that these position measurements at 3 o'clock do NOT allow an exact evaluation of the momentum BEFORE 0 o'clock and AFTER 3 o'clock, but only within the uncertainty associated with Compton recoil} \letterp{Ehrenfest}{Goudsmit, Uhlenbeck, and Dieke}{3}{11}{1927}}. %
%
Heisenberg discussed a similar \TE in his book, but replaced the first position measurement with a momentum measurement:

\q{This may be expressed in concise and general terms by saying that every experiment destroys some of the knowledge of the system which was obtained by previous experiments. Beforehand, however, it should be noted that the uncertainty relations apparently do not apply to the past. For if the electron’s \emph{velocity} is initially known and then its \emph{position} is measured precisely, one can also calculate the electron’s positions for the \emph{time before the position measurement} with precision; for this past, $\Delta p \Delta q$ is then smaller than the usual limit. Yet this \emph{knowledge of the past} is purely speculative, since (because of the \emph{change of momentum during the position measurement}) it in no way enters as an initial condition into any calculation concerning the electron’s future and does not appear in any physical experiment at all. Whether one should ascribe any physical reality to the aforementioned calculation about the electron’s past is therefore purely a \emph{matter of taste} (\german{Geschmacksfrage})}[\citep[15]{Heisenberg1930}]
%  
Heisenberg's choice to use a momentum followed by a position measurement, as we shall see, will be of considerable importance. Heisenberg might have in mind measuring procedure he uses a few pages later (see Fig.~\ref{fig:heisenberg-doppler}). Electrons move along the $+x$ direction with a known longitudinal momentum $p_x$, while light quanta enter along the $-x$ axis. The Doppler shift of a scattered light quantum at $t_0$ provides, assuming energy and momentum conservation, a measure of the electron’s transverse momentum $\delta p_y$; the arrival point of the electron at a later time $t_1$ on the detection screen fixes its position $\delta y$. In this way, one seems able to reconstruct the \emph{past path} of the electron with arbitrary precision $\delta y \, \delta p_y \approx 0$. Notice, that Heisenberg refers here somewhat ambiguously to the \emph{time before the position measurement}. As it turned out this remark was open to misunderstanding, since it seems to imply the possibility of reconstructing the path of the electron also \emph{before} the momentum measurement.

\begin{figure}
\centering
\includegraphics[scale=0.1]{1930heisenbergdoppler.png}
\caption{Heisenberg’s Doppler-effect thought experiment for the determination of an electron’s past transverse path. Adapted from \citet{Heisenberg1930}.}
\label{fig:heisenberg-doppler}
\end{figure}
%
Heisenberg immediately pointed that the reconstruction of past of the electron it has no predictive significance for its future behavior, and therefore does not constitutes a violation of the \IR. However, the question of whether such reconstructions possess any \s{physical meaning}---that Heisenberg dealt with somewhat rather quickly--- constitutes, as one might expect, an interesting philosophical puzzle. Indeed, the issue of  was immediately taken up by Moritz Schlick. Schlick started to draft a manuscript on the problem of causality in quantum physics in the summer of 1930 and finished it at the beginning of October. At the end of November, Arnold Berliner confirmed receipt of \art{Die Kausalität in der gegenwärtigen Physik}, which soon circulated among physicists, prompting correspondence with Einstein, Heisenberg, and Pauli. The paper identifies causality with predictability; thus, Schlick accepts that Heisenberg’s relation concerns an indeterminacy of prediction, thereby limiting the applicability of causality on the microscopic scale. Schlick was aware that such indeterminacy does not apply to retrodictions. Nevertheless, he remained unconvinced by Heisenberg’s agnostic stance toward the physical reality of past trajectories:

\q{Heisenberg thinks that whether any physical reality should be assigned to the calculation concerning the past of the electron is purely a question of taste. But I should prefer to put it more strongly, in complete agreement with what I take to be the incontestable basic viewpoint of Bohr and Heisenberg themselves. If a statement about an electron’s position is not \emph{verifiable} within atomic dimensions, \emph{we can attach no meaning to it}; it becomes impossible to speak of the ‘path’ of a particle between two points at which it has been observed. (This is not, of course, true of bodies of molar dimensions, for it is in principle possible to verify afterwards that the projectile was located at the intervening points.)}[\citep[159]{Schlick1931}]
%
Schlick argues that although quantum theory’s equations allow one to calculate an electron’s past positions after exactly measuring its velocity and then its position, such results are physically meaningless, since they are \s{unverifiable} in principle: one cannot verify whether the electron was actually moving with that velocity at the calculated point, since any observation would disturb its path in an uncontrollable way. For Schlick, therefore, the very idea of a sharply defined past path of the electron lacked physical meaning, just like the notion of this future path.

%The discussion thus moved the debate from Heisenberg’s cautious suggestion of a \emph{Geschmacksfrage} to a more radical claim about the semantic limits of physical concepts. The very notion of \s{path} or trajectory postion-cum-momentum is meaningeless in \qm.


\subsection{ETP Experiment}
\label{sub:ETP}

It is not surprising that Einstein, in his correspondence with Schlick in the late 1930s, characterized the latter’s stance as \s{too positivistic} \letteraea{Einstein}{Schlick}{28}{11}{1930}[21-603]. Concluding his letter, Einstein mentioned that he would spend the coming winter in Pasadena. Just a fortnight after \posscite{Schlick1931} paper appeared on \datef{13}{2}{1931}, Einstein, in collaboration with Richard Tolman and Boris Podolsky (ETP), submitted to the \emph{Physical Review} a short letter reporting the results of his stay in California. The paper was published on \datedm{15}{3}{1931} \citep{Einstein1931-03}. Neither Heisenberg nor Schlick are mentioned; however, Einstein and his collaborators address the same question of the past path of particles in \qm. They argue that \textit{if} sharp \emph{retrodictions} about one particle were possible, one could, with the help of conservation laws, infer \emph{predictions} about the future path of a second particle that would otherwise be excluded by the uncertainty relations. Thus, ETP conclude that retrodictions should \emph{not} be allowed, and that the \IR apply equally to the past and the future.

The proposed arrangement (see Fig.~\ref{fig:ETP-setup}) resembles the one Einstein reportedly used in the debate with Bohr in Brussels in 1930 \citep[1.4.3]{BacciagaluppiCrull2024}. A box $B$ contains identical particles. At time $t_0$, a shutter $S$ releases two particles, $a$ and $b$. By weighing the box before and after the emission, the observer can determine the total energy loss, $E_{\text{tot}} = E_a + E_b$. At $O$, particle $a$ may undergo different measurements. Its momentum $p_a$ (and energy $E_a$) can be determined, for example, by a Doppler-effect measurement at $O_1$, while its arrival position and time $t_1$ can be recorded at $O_2$. From the measured position of $a$ at $t_1$, together with its past momentum $p_a$, the shutter opening time $t_0$ can, in principle, be reconstructed. If $E_a$ has been measured, then the energy of particle $b$ can be inferred from $E_b = E_{\text{tot}} - E_a$. Once $t_0$ has been determined, the arrival time $t_b$ at $O_2$ can be inferred from the geometrical arrangement of the setup, that is, from $SRO$.


\begin{figure}
\centering
\includegraphics[scale=0.15]{1931ETPmod}
\caption{Adapted from \citet{Einstein1931-03}.}
\label{fig:ETP-setup}
\end{figure}

At this stage, a backdoor appears to open, seemingly offering a way around the energy--time uncertainty relation $\delta E, \delta t \leq \frac{h}{4\pi}.$ Yet ETP quickly close it. One might attempt to measure momentum via the Doppler (or Compton) shift of low-energy light quanta, that is, long-wavelength radiation (e.g., in the infrared range). Assuming conservation of momentum and energy, this method, in principle, allows one to infer the momentum of a particle before and after scattering from the frequency shift of the scattered photon. However, to determine the frequency with high precision, many oscillation periods must be observed, which blurs the exact time of the scattering event. As a result the precise moment at which the shutter $S$ was opened cannot be reconstructed. Conversely, if one tries to determine the scattering time very precisely, one must use high-frequency (short-wavelength) radiation, thereby losing information about the momentum before the collision. The authors summarized their conclusion as follows:

\q{Thus in our example, although the velocity of the first particle could be determined both before and after interaction with the infrared light, it would not be possible to determine the exact position along the path $SO$ at which the change in velocity occurred as would be necessary to obtain the exact time at which the shutter was open. \\ It is hence to be concluded that the principles of the quantum mechanics must involve an uncertainty in the description of \emph{past} events which is \emph{analogous} to the uncertainty in the prediction of \emph{future} events. It is also to be noted that although it is possible to measure \emph{the momentum of a particle and follow this with a measurement of position}, this will not give sufficient information \emph{for a complete reconstruction of its past path}, since it has been shown that there can be \emph{no method for measuring the momentum of a particle without changing its value}}[\citep[781]{Einstein1931-03}]
%
Thus, ETP point out the flaw in this kind of experiment: one can reconstruct the path of a particle \emph{between} the momentum and the position measurements, but not \emph{before} the momentum measurement in $O_1$. The uncertainty principle therefore not only restricts predictions about the future but also imposes limitations on the reconstruction of the past: the path reconstruction before the momentum measurement should be physically forbidden \emph{if} the \IR\ are to hold.

%Finally, it is of special interest to emphasize the remarkable conclusion that the principles of quantum mechanics would actually impose limitations on the localization in time of a macroscopic phenomenon such as the opening and closing of a shutter.

\todo{a \pemo against the \CH hypothesis}

As Ehrenfest’s famously \datef{9}{7}{1931} pointed out to Bohr, Einstein, contrary to common perception, was clearly not attempting to devise \q{new \s{perpetua mobilia} = machines against the uncertainty relation} (\letter{\Ehr}{Bohr}{9}{7}{1931}, AHQP/BSC 18). Rather, Einstein’s point was that, \emph{because of} the \UR, the experimental \emph{choice} between determining time or momentum of $a$ appeared to retroactively determine whether the emitted light quantum $b$ possesses a definite energy or a definite emission time \emph{after} it had already left the box. After a talk in Leiden in November 1931, Einstein might have been stimulated by \Ehr\ to replace the energy--time with the position--momentum uncertainty \letteraeap{Einstein}{\Ehr}{5}{4}{1932}[10-231]. Indeed, as recalled by Léon Rosenfeld, Einstein, after a seminar in Brussels in the spring of 1933, apparently reformulated the argument without the aid of the light quantum box. Instead, he considered two particles emitted simultaneously: a measurement of the momentum (or position) of the first particle allowed an inference of the corresponding momentum (or position) of the second. If Rosenfeld's testimony is accurate, Einstein possessed, in embryonic form, the gist of the EPR paper before permanently moving to the United States on \datef{17}{10}{1933}.

%\citep[]{BacciagaluppiCrull2024}.

\section{Popper’s Turn to Quantum Theory. The Correspondence with Weisskopf}
\label{sec:popperquantun}

At the turn of 1933, Popper was reworking the manuscript of the \bt{Die Beiden Grundprobleme der Erkenntnistheorie} \citep{Popper1930-1933} into what would become the \emph{Logik der Forschung} \citep{Popper1935}. Popper submitted to Schlick, the series's editor, a first version of the manuscript in the spring of 1934. In particular, at request of the publisher Springer the manuscript had to be shorted considerably \citep[235--244]{Hacohen2010-06-11}. While Popper was working on adding two technical chapters on probability and one on \qm, much of the editing was carried out with the help of his uncle, Walter Schiff. Between May and June 1934, Popper probably started to devoted special attention to the quantum physics chapter, discussing with Viennese physicists like Urbach, Urbach's friend Weisskopf and his cousin Käthe \citep[237]{Hacohen2010-06-11}. At that time, Weisskopf was in Zurich as Pauli's assistant, so their exchange had to take place by mail, leaving us written evidence of Popper’s thinking during that period. Although Popper’s original letter has not survived, much of his position can be inferred from Weisskopf’s reply.

Weisskopf seems to concede to Popper that the uncertainty relations do not exclude that position and momentum of a single electron can be \s{known} (\ie, measured) with arbitrary accuracy. If one fixes the position of electrons with a very narrow slit, the emerging beam no longer consists of electrons with a well-defined momentum or energy. The diffraction pattern on a screen behind the slit reflects this momentum (or angular) distribution, which could also be inferred from the energy spread: when the electrons hit a target, some atoms become excited even though their excitation requires more energy than the mean kinetic energy of the electrons (the Fourier tail). Since the position is fixed by the slit measurement, we can reconstruct the path of the electron with arbitrary precision. As Weisskopf put it, \qt{[t]his consideration is of course not changed if, by some measurements, I determine which energies are present in this accumulation of electrons \emph{without altering their spatial arrangement}. In this, I agree with you and admit my mistake}{Diese Überlegung wird natürlich nicht geändert, wenn ich durch irgendwelche Messungen feststelle, welche Energien in dieser Elektronenanhäufung vorhanden sind, ohne deren räumliche Anordnung zu verändern. Darin gebe ich Ihnen Recht und bekenne meinen Fehler} \letterKPAp{Weisskopf}{Popper}{26}{6}{1934}[360-21].

However, Weisskopf immediately raises a counter-objection: \qt{But now comes the point: if I can measure these energies without disturbing the spatial arrangement, then I have devices with which I can also \emph{produce} \origins{herstellen} beams (aggregations of electrons) that are 1.) spatially confined and 2.) possess a sharp energy (= momentum of free particles)}{Aber nun kommt der Punkt: Wenn ich diese Energien ohne Störung der räumlichen Anordnung messen kann, so habe ich Apparate, mit denen ich auch Strahlen (Elektronenanhäufungen) \emph{herstellen} kann, die 1.) räumlich begrenzt, 2.) eine scharfe Energie (= Impuls bis freien Teilchen) haben} \letterKPAp{Weisskopf}{Popper}{26}{6}{1934}[361-21]. The possibility of producing such aggregates of electrons would indeed violate the \IR, since a wave localized in space must contain infinitely many frequencies, and thus momenta and energies. Popper probably had replied that instruments allowing \emph{measurements} of electron sharp energy and position might exist without thereby enabling the \emph{production} of an \emph{aggregation} of such electron beams. However, Weisskopf doubted this, arguing that if we can \emph{measure} electron energy, we should also be able to \emph{separate} them according to predetermined values, that is to produce an aggregate of electrons all with the same energy and the same position:

\qt{(1) Apparatuses that can \emph{measure} both position and energy at the same time are not necessarily able to \emph{produce} electron aggregates with sharp position and energy. I consider this conclusion not entirely correct, since -- in the case that one has the possibility of determining the energy of electrons at sharp position -- it must surely also be technically possible somehow to remove them, for example, if they have too much energy.  \\ (2) You can probably anticipate my line of reasoning, which leads you to say that quantum mechanics is only capable of dealing with that world in which such \emph{apparatuses} are not applied. But I would already regard that as a breakdown. After all, apparatuses are also part of nature and should therefore be valid within it. \\ There would, however, have to exist accumulations of electrons whose position and energy are sharply determined, even though this is impossible according to wave theory. Thus the statement that electrons can in fact always be represented by waves would not always be correct, and in this lies perhaps the end of quantum mechanics}{\begin{enumerate} \item Apparate, die Ort und Energie zugleich messen k\"onnen, m\"ussen deswegen noch nicht im Stande sein, Elektronen- anh\"aufungen mit scharfem Ort und Energie herzustellen. Diesen Schluss halte ich f\"ur nicht ganz richtig, da es -- im Falle, da{\ss} man \"uberhaupt die M\"oglichkeit hat, die Elektronen bei scharfem Ort bzgl. ihrer Energie zu erkennen, -- sicher auch m\"oglich rein technisch m\"oglich
sein muss, sie irgendwie z.B. wegzuschaffen, wenn sie eine zu gro{\ss}e Energie haben. \item Sie sehen meinen Schlussweise wohl kommen, der Sie sagen, da{\ss} die Quantenmechanik eben nur im Stande ist jene Welt zu behandeln, in der solche Apparate nicht angewendet werden. Das w\"urde ich aber schon als einen Zusammenbruch bezeichnen. Die Apparate sind ja auch ein St\"uck Natur und sollten in solcher gelten -- so m\"ussen \"ahnliche Erscheinungen auch irgendwo sp\"urbar auftreten. \end{enumerate} Es m\"ussten aber Elektronenanh\"aufungen existieren, deren Ort und Energie scharf bestimmt ist, ob es nach der Wellentheorie unm\"oglich ist. Somit w\"are es, sog. Elektronen lassen sich tats\"achlich durch Wellen darstellen nicht immer richtig und darin ist wohl das Ende der Qu.-M.}[\letterKPA{Weisskopf}{Popper}{26}{6}{1934}[360.21]]
%
Here the two fundamental issues at stake are laid out quite clearly: (1) the distinction between (1a) \emph{producing} an aggregate of electrons with arbitrarily sharp position and momentum, and (1b) \emph{measuring} the position and momentum of a single electron with arbitrary precision; (2) the \s{coupling} between the two operations, that is, the claim that the possibility of (1b) necessarily implies that of (1a), which in turn would certainly constitute a violation of the \IR. Popper also asked \Weis about the use of the term \emph{pure case} to indicate a particle ensemble in which either position or momentum is sharp; probably around that time he coined the expression \s{\emph{super-pure case}} to refer to one in which both momentum and position are sharp.

Weisskopf asked Popper to let him know his opinion on these ideas. However, concerning Popper's request to cite their discussions in his forthcoming book, he asked not to be named personally. In the following weeks, Popper might have come up with a clever trick to counter Weisskopf’s objection: \GE\ in which an apparatus was laid out that could \emph{measure} with arbitrary precision the position and momentum of a single particle without allowing the \emph{production} of a super-pure case more homogeneous than the \IR would permit. Already,P on \datef{27}{8}{1934}, Popper submitted a short note, \art{Zur Kritik der Ungenauigkeitsrelationen} to \jt{Die Naturwissenschaften}, one the most prestigious scientific journal of that time. As we shall see, Poppe  independently envisaged the same backdoor to the \IR as that which the ETP had already suggested, only using momentum and position instead of energy and time. 

Popper seems to have also soon realized that this backdoor could indeed be closed in the same way it was suggested by ETP. However, contrary to Einstein, Popper wanted to leave the backdoor open. Thus, he wrote a short supplement, \bt{Ergänzung zu der vorstehenden kurzen Mitteilung} to clarify the matter. By the autumn of 1934, Popper had completed the \emph{Logik der Forschung}. Thus, by that time, chapter IX, titled \emph{Bemerkungen zur Quantenmechanik}, together with the three appendices V, VI, and VII devoted to the same topic, can be regarded as complete. The manuscript of the \bt{Ergänzung} was probably written after the paper had already been submitted to \jt{Die Naturwissenschaften}, but before it was published. It must also have been written prior to Appendix VI and VII, which appears to be Popper's attempt to address problems raised in the \bt{Ergänzung}.

\section{Popper's Interpretation of the Indeterminacy Relations}
\label{sec:interpreation}

From the formalism of quantum mechanics one can derive the so-called \IR \origg{Unbestimmtheitsrelation}, that is the inequality:

\begin{equation}
\label{eq:ur}
\Delta p_x \Delta x > h / 4 \pi
\end{equation}
%
where \pos is the space coordinate of a point mass and the \mom the momentum component along the same direction. Popper’s aim was not to challenge Heisenberg's mathematical \emph{derivation} of formula, but rather his physical \emph{interpretation} of it. Heisenberg presented the \IR as (a) \s{imprecision relations,} \origg{Ungenauigkeitsrelationen}, which set a limit on the accuracy of simultaneous measurements of $\pos$ and $\mom$ for a \emph{single} particle (b) \s{dispersion relations,} \origg{Streuungsrelationen}, expressing the limit on the standard deviations of repeated \pos and \mom measurements on an \emph{ensemble} of particles. It is undeniable that Popper was among the first to make this distinction, at least in the philosophical literature,\footnote{In the physics literature this distinction was at least implied in \citet{Kennard1927} and \citet[67\psq]{Weyl1928}. Popper was familiar with the latter.} although similar views were defended around the same time by Viennese scholars, e.g., by \citepp{Mises1930}{Mises1934}. It is unclear whether Popper was aware of these works or whether he arrived at a similar conclusion independently, starting from von~\citets{Mises1928} frequency interpretation of probability, which he defended in ch.~VI of the book.

%https://arxiv.org/pdf/1904.06139

According to Popper, Heisenberg’s \emph{philosophical justification} for \cref{eq:ur} was based on the idea that measurement itself disturbs physical systems: precision in one variable (e.g., position) can be achieved only at the expense of precision in its conjugate (e.g., momentum), with \cref{eq:ur} establishing the bound. Against this view, Popper argued that the \emph{mathematical derivation} of \cref{eq:ur} follows from Born’s interpretation of the $\psi$-function as a probability amplitude, with $|\psi(x)|^2 dx$ giving the probability of finding a particle along the $x$-axis within the infinitesimal interval between $x$ and $x + dx$. Using Fourier analysis, one can show that the position and momentum probability distributions are mathematically related in such a way that a narrower distribution in position entails a broader distribution in momentum, and vice versa. In this reading, \cref{eq:ur} regulates the dispersion relation between these distributions rather than limits the precision of measurement. 

%As Popper emphasizes, as not often been observed that \q{DaB der mathematischen Ableitbarkeit der HEISENBERG-Formeln aus den grundlegenden quantenmechanischen Gleichun-gen auch eine Ableitbarkeit der Interpretation jener Formeln aus der Interpretation dieser Grundgleichungen genau entsprechen muB}.

%\footnote{Indeed, \citet{Heisenberg1927a} derived the \IR from $S(q, p)=\frac{1}{\sqrt{2 \pi \hbar}} e^{i p q / \hbar}$, where the transformation function $S(q,p)$ expresses the amplitude that a state with definite momentum $p$ has the coordinate $q$}



For Popper the confusion between these two \emph{interpretations} is ultimately the consequence of physicists not clearly distinguishing between \begin{inparaenum}[(1)] \item \emph{physical\footnote{As opposed to merely \s{mental} or \s{imagined} selection, focusing on a certain section of a particle beam without physically separating it from the rest \LF{163}} selection} \origg{Aussonderung}, which occurs when, for example, a beam of particles is filtered so that only those passing through a narrow slit (a region $\Delta x$) remain, thereby selecting an \s{aggregate} of particles with all the same position, a so called \s{pure case}\footnote{An aggregate \origg{Menge} corresponds to what we now call an \s{ensemble}. As \citet[225\fn{$^\ast$1}]{Popper1959} later recognized, at that time, he did not clearly distinguish between an aggregate of co-present particles, such as a beam (spatial ensemble), and an aggregate of repetitions of an experiment performed with one particle or one system of particles (temporal ensemble)} \item a \emph{measurement} \origg{Messung} involves the detection of a particular system by means of a measuring instrument. Unlike separation, measurement produces an actual registration of a value—such as a spot on a screen or a pointer deflection---an irreversible change in a piece of macroscopic apparatus which constitutes a publicly observable record \LF{163} \end{inparaenum} 

%Where, in this paragraph, I speak of ‘an aggregate of particles’ I should now speak of ‘an aggregate—or of a sequence—of repetitions of an experiment undertaken with one particle (or one system of particles)’. Similarly, in the following paragraphs; for example, the ‘ray’ of particles should be re-interpreted as consisting of repeated experi- ments with (one or a few) particles—selected by screening off, or by shutting out, particles which are not wanted.

Young Popper indeed seems to have articulated a distinction similar to the one between \s{preparation} and \s{measurement} formulated by \textcites{Margenau1936}{Margenau1937}, at around the same time\footnote{However, below I argue that Popper's notion of \s{physical selection} is not identical to the modern notion of \s{preparation}}. Popper argues that (a) not every physical selection amounts to a measurement: for instance, when electrons are passed through a velocity filter, the procedure produces an ensemble of electron with definite momentum (a pure case described by a, monochromatic plane wave), but no value of the momentum is actually recorded. Conversely, Popper also makes the possibly more troublesome claim, (b) that not every measurement can be regarded as a physical selection: electrons belonging to a monochromatic beam may be registered by a Geiger counter at a given location, yet in this case the electrons are not physically separated according to their position \LF{163\psq}. As we have seen, \Weis challenged point~(b), arguing that once a measurement on single particles sharper than Heisenberg's \s{imprecision relations} would allow is possible, the same setup could in principle be used to produce a super-pure ensemble of particles, thereby violating Heisenberg's \s{variance relations} as well.

The two possible \emph{interpretations} of \cref{eq:ur} can be the rephrased as follows: \begin{inparaenum}[(a)] \item the \IR limits the degree of \emph{statistical homogeneity} that can be achieved in the \emph{physical selection} of an \emph{ensemble} of identical systems; \item the \IR limits the \emph{precision} of the \emph{measurement} performed on a \emph{single} system \end{inparaenum} \LF{164}. According to (a), given an ensemble of particles with, sah, sharp momenta (pure case), we are not allowed to select a sub-ensemble that also has a sharp position (a super-pure case); (b) refers to the simultaneous measurement of position and momentum on a single particle, in which the two measurements hinder each other. Popper's program can be then articulated as follows \begin{inparaenum}[(I)] \item that (a) follows directly from the formalism \item (b) does not \emph{logically} follow from (a). \item (b) follows from (a) only if one adds an \emph{additional hypothesis} (c); \item The combined system (a) + (c) turns out to be \emph{contradictory} \LF{154\psq}.
\end{inparaenum}


Point (I) and (II) could be dealt with relatively easily. Popper notes that it has rarely been observed that the mathematical derivability of the Heisenberg formulas from the fundamental equations of quantum mechanics must also be matched by a corresponding derivability of the interpretation of those formulas from the interpretation of the fundamental equations. (a) Some, like Arthur!\citet[1\psq, 57]{March1931} regard the statistical dispersions as a mere consequence of measurement inaccuracies; (b) others, like \citet[68]{Weyl1928}, seem at least to accept the reverse, namely that the limitations on single measurements follow from the statistical view. Popper rejects both options. Quantum formulas make \emph{frequency predictions} \origg{Haufigkeitsprognosen}. The latter cannot imply prohibitions on individual measurements (apart from the trivial cases of probability~1 or~0) \LF{165\psq}.  The \UR\ implies that if 1000 position measurements show the electron to be at the same place, then it may be expected that 1000 momentum measurements would yield widely divergent results. A \s{sharp} measurement of both \pam---that is, an exceptionally precise outcome for both variables---is not impossible but highly improbable. It represents an event analogous to a statistical fluctuation within the ensemble distribution.

%just as there is no contradiction between \s{the throw of a die $k$ yields a five with probability $1/6$} and either \s{$k$ is a five} or \s{$k$ is not a five}


Thus, from a strictly logical point of view, according to Popper, all alleged \s{proofs} that exact position and momentum \emph{measurements} contradict the formalism of quantum mechanics are flawed. The \IR\ do not \emph{forbid} exact position--momentum measurements; they merely maintain that exact position--momentum \emph{predictions} cannot be \emph{derived} from them \LF{166}. Heisenberg himself could not conceal the tension between these two alternatives: (I) he argues that, since the \IR\ forbid selecting an ensemble of systems with exact values of both position and momentum, \emph{predictive measurements} of the future position--momentum trajectory of a particle with arbitrary precision are \emph{forbidden}. (II) He nevertheless concedes, almost in passing, that the \IR\ allow for arbitrary \emph{non-predictive measurements} of position and momentum. These reconstructions can, in principle, be obtained by combining two predictive measurements: (a) position followed by position, (b) momentum followed by position, or (c) position followed by momentum \LF{156\psq, 167}. 

Because predictive position--momentum measurements cannot be derived from \qm, physicists generally conclude that the very notion of a \emph{future} trajectory has lost physical meaning in \qm \citep[??]{March1933}. Yet the question of the physical meaning of the \emph{past} trajectory between two measurements remains ambiguous. Popper could then enter the Heisenberg--Schlick debate we have presented in \cref{sub:heisenbergschlick}. As we have seen, (a) Heisenberg dismissed the issue as a \s{question of taste}, and (b) Schlick regarded it as a \s{meaningless question}. Both ultimately agreed that, since such reconstructions cannot be \emph{verified}—that is, since no testable prediction can be derived from them—they pertain to the realm of \s{metaphysics}. As one might expect, however, for Popper the question in this form is ill-posed. The point is that if sharp past trajectory reconstructions were not possible, the frequency predicitons of \qm\ could not be \emph{falsified}, and the theory would therefore belong to the realm of \s{metaphysics}. Without the possibility of reconstructing the past paths of particles one could subject the theory to empirical control \origg{Nachprüfung}.

In, say, a diffraction experiment, the position is physically selected at $x_1$ by means of a screen containing a slit. The theory predicts that the particles’ energy—and hence their speed—remains constant, while their direction, and thus the transverse component of momentum $p_y$, is spread. This can be subject to empirical control by placing a photographic plate at $x_2$ and observing the distribution of spots. Each spot, corresponding to a single particle detected at position $\delta y$, allows one to infer the deflection angle $\delta \varphi$ and the corresponding past momentum component $\delta p_y$. The observed distribution of these reconstructed paths (obtained through a position selection followed by a momentum measurement) must conform to the predicted distribution as the diffraction experiment is repeated many times; otherwise, the theory would have to be rejected.\todo{cut with Weisskopf letter?}

%(I) Heisenberg’s formulas \cref{eq:ur} should receive a statistical interpretation as \s{spread limitations}, and (II) their interpretation as \s{accuracy limitations} is not a logical consequence of quantum mechanics. As a consequence,
%\citep{Pechenkin2002}

\subsection{The Coupling-Hypothesis}
\label{sub:couplinghypothesis}

As one can see, Popper articulated an early version of what would might be called an objective statistical interpretation of \qm \citep[50.2]{Pechenkin2022}, which dispenses with the collapse of the wave function and treats measurement as the selection of a sub-ensemble rather than a physical reduction. When a measurement is performed, one selects an ensemble of systems for which, say, the momentum is sharply defined \LF{171}. In this interpretation, the \IR\ imply that one cannot further select a sub-ensemble whose members also possess a precisely defined position. By June 1934, Popper had managed to convince Weisskopf that the \IR\ \emph{forbid} physical selections of an ensemble more homogeneous than a pure case, but still \emph{allow} arbitrarily precise measurements on single systems. However, Weisskopf may have prompted Popper to take a further step. As we have seen, Weisskopf countered that if it were possible to \emph{measure} with arbitrary sharpness, then the same measuring instruments could be used to \emph{produce} particle aggregates whose position and momentum spread would be smaller than that allowed by the \IR.  In this sense, the \IR, as a prohibition of a super-pure case, also imply the prohibition of arbitrarily sharp predictive measurements. 

In \LdF, Popper addressed an objection that an unnamed physicist might have raised—without any doubt Weisskopf. Indeed, on pages 173--174, Popper reproduces several passages from Weisskopf’s letter almost verbatim, though without explicitly naming him as the source, as Weisskopf had requested. Popper now offers a more articulated reply to \Weis’s objection. According to Popper, \Weis’s argument does not \emph{prove} that sharp predictions contradict quantum mechanics. Rather, it tacitly introduces an \emph{additional hypothesis}, which Popper calls the \emph{\CH} \origg{Kopplungshypothese}. The latter maintains that predictive measurements and physical selections are inevitably linked, so that any predictive measurement is also a physical selection—or, operationally speaking, that any instrument capable of measuring position and momentum with arbitrary precision could also be used to select a corresponding super-pure case, at least in principle:

\qt{The statement (which corresponds to Heisenberg’s view) that exact single predictions are impossible turns out to be equivalent to the hypothesis that predictive measurements and physical selections are inseparably coupled. With this new theoretical system—the conjunction of quantum theory with this auxiliary \emph{coupling-hypothesis}—my conception must indeed clash. What remains to be shown is that the system, consisting of statistically interpreted quantum mechanics (together with the laws of momentum and energy conservation), combined with the coupling hypothesis, is contradictory}{{Zu diesen Überlegungen bemerken wir zunächst daß sie vielleicht ganz plausibel sind ein strenger \emph{Beweis} dafür daß wenn eine prognostizierende Messung möglich ist auch eine entsprechende physikalische Aussonderung möglich sein müßte, kann jedoch (wie wir gleich sehen werden aus guten Gründen) nicht erbracht werden: infolgedessen beweisen diese Überlegungen nicht daß exakte Prognosen der Quantenmechanik widersprechen würden sondern sie führen eine \emph{zusätzliche Hypothese} ein; der (Heisenbergs Auffassung entsprechende) Satz, daß genaue Einzelprognosen unmöglich sind, erweist sich als äquivalent mit der Hypothese, da $\beta$ prognostizierende Messungen und physikalische Aussonderungen gekoppelt ${ }^1$ sind. Und dem theoretischen System: Quantenmechanik plus Koppelungshypothese muß unsere Auffassung natürlich widersprechen}}[\LF{174}]
%
In footnote~1 on p.~346, Popper remarks that the additional hypothesis could take other forms, but he addresses this one in particular because the objection—that measurement and physical separation are coupled—had been raised against his view in both oral and written discussions, once again alluding to Weisskopf. Popper had now realized that the most deep-seated prejudice of Heisenberg's view lay in the unexamined belief that physical selections and measurements are necessarily coupled. With this, he considered point~(3) of his program accomplished. What remained to be addressed was point~(4): to demonstrate that the system—statistically interpreted quantum mechanics, together with the conservation laws of momentum and energy, when combined with the coupling hypothesis—leads to a contradiction.



\section{Popper’s ETP-like Thought Experiment}
\label{sec:TE}

As we have seen, according to Pooper, physicists grudgingly conceded that (a) \IR do not limit the accuracy of non-predictive measurements. However, they could still argue (b) the \IR do limit the accuracy of predictive ones. Popper’s goal was to show that also (b) should be rejected. On the basis of non-predictive measurements, one can make a predictive measurement that could not serve a physical selection producing a super-pure case \LF{\S77}. As we have seen there three possible combination of predictive-measurement, that  of non-predictive measurements: (a) two successive position measurements, (b) a position measurement preceded by a momentum selection, and (c) a position measurement followed by a momentum measurement. The case considered by Popper—momentum selection followed by position measurement—belongs to type (b). The choice is not casual: It \qt{just that case which, according to Heisenberg \textelp{}, permits \s{a calculation about the past of the electron}}{Es ist das wohl gerade jener Fall, der nach Heisenberg eine \s{Rechnung liber die Vergangenheit des Elektrons} gestattet} \LF{177}\hide{241f.}.

\subsection{The Experimental Setup}
\label{sub:setup}

As we have seen (see above \ref{sub:heisenbergschlick}), \citet[15]{Heisenberg1930} on p.~15 of his book uses the a sequence of momentum then position measurement (b) to argue for the possibility of a \s{calculation of the past} of the electron \emph{before} the position measurement. Popper seems to read this claim as implying that while (b) allows one to calculate the past trajectory also \emph{before} the momentum measurement, (a) and (c) allow it only \emph{between} the two measurements\footnote{In my view, there is no evidence that this is the case. Heisenberg's remark is simply vague}. As we shall see, this is the key but shaky premise on which Popper's argument hinges. On this basis, Popper proposed a thought experiment involving a particle collision. After the collision, by performing a \emph{non-predictive} measurement on one particle (momentum followed by position), with help of conservation of momentum, one could in principle determine an arbitrarily sharp \emph{predictive measurement} of position and momentum for another particle. As one can see, Popper arrived independently at a \TE\ analogous to the ETP scenario (see above \cref{sub:ETP}), except that energy and time are replaced by position and momentum. Of course, Popper's goal was the opposite of ETP's: to show that the asymmetry between past and future does indeed allow one to overcome the \IR\ when interpreted as \s{inaccuracy relations}.

\begin{figure}
\centering
 \includegraphics[scale=0.22, trim = 0mm 0mm 0mm 0mm, clip]{popperexperimentGmod.png}
 \caption{Slightly modified, by adding the labels $X_1$ and $X_2$}
 \label{fig:setup}
 \end{figure}


The idealized experimental setup consists (\cref{fig:setup}) of two (extremely low-density) beams intersecting at $S$ \origg{Schnittpunkt}. Each beam is treated as if it were a \s{pure case}. The electron beam $A$ is monochromatic, carrying a sharp known momentum $\mathfrak{a}_1$, while the light quantum beam $B$ is slit-selected, with a sharp position at $Bl$, but with momentum of unsharp direction and known magnitude $\mathfrak{b}_1$. One can then perform a \emph{post hoc} psychological selection focusing on the partial rays $[A]$ and $[B]$ that collide at $S$, in a manner similar to the Compton--Simon and Bothe--Geiger experiments. An electron from beam $A$ emerges at detector $X$ with momentum $\mathfrak{a}_2$, while a light quantum from beam $B$ emerges at detector $Y$ with momentum $\mathfrak{b}_2$. The process is governed by momentum conservation, such that $\mathfrak{a}_1 + \mathfrak{b}_1 = \mathfrak{a}_2 + \mathfrak{b}_2$. Since the light quantum has a broad momentum-direction distribution, in some runs of the experiment the electron \s{keeps} its original momentum, while in others it \s{inherits} the light quantum’s momentum. If one considers only the electrons, disregarding the light quanta with which they interacted, the electron beam is a non-pure case after the collision.

Popper suggests placing an apparatus at $X$ to measure the momentum $[\mathfrak{a}_2]$ of the electron after the collision and its position at the time $t_A$. In the Note for \NW, as well as in the main body of the \LdF, Popper proposes using a moving photographic film \origg{Filmanstreifen}. When a particle strikes, it leaves a mark at a specific spatial coordinate (position) on the strip, and because the strip is moving, this also indicates a time coordinate (moment of impact). In this setup, the photographic film is also supposed to measure the energy of the electron indirectly through the energy transferred to the atoms of its sensitive material. From the energy one can calculate the electron’s momentum $[\mathfrak{a}_2]$ in the $SX$ direction. Popper assumes that \qt{[t]he \emph{accuracy} of this calculation is, for the $SX$ direction (with a suitable arrangement), subject to no fundamental limitation of the kind implied by the uncertainty relations. It is assumed here that the magnitude of momentum and the time of an incoming particle—\s{non-predictive} measurements—can be measured with arbitrary precision}{Die \emph{Genauigkeit} dieser Berechnung ist für die $SX$-Richtung (bei geeigneter Anordnung) keiner grundsätzlichen Beschränkung von der Art der Ungenauigkeitsrelationen unterworfen, dabei wird vorausgesetzt, daß Impulsbetrag und Zeitpunkt eines einfallenden Teilchens \s{nichtprognostische} Messung beliebig genau meßbar ist} \citep[807]{Popper1934}.

Popper does not discuss this premise at this stage; he only argues that, \emph{if} this is the case, from $t_A$ and the momentum $[\mathfrak{a}_2]$ the time of collision $t_S$ can be reconstructed. Via momentum conservation, one can infer the light quantum’s momentum after the collision from the electron’s momentum, $\mathfrak{b}_2 = (\mathfrak{a}_1 + \mathfrak{b}_1) - \mathfrak{a}_2$. Consequently, from the time of the collision and the light quantum’s momentum $\mathfrak{b}_2$, one can make a predictive measurement sharper than allowed by the \IR: the light quantum from beam $B$ will reach $Y$ at time $t_B$ with momentum $\mathfrak{b}_2$. This prediction can, in principle, be tested empirically. Reconstructing the \emph{past of an electron} via an arbitrarily sharp non-predictive measurement together with a conservation law appears to allow an arbitrarily predictive measurement of the \emph{future of a light quantum}. 

%In other terms, reconstructing the \emph{past of an electron} via an arbitrarily sharp non-predictive measurement together with a conservation law appears to allow an arbitrarily predictive measurement of the \emph{future of a light quantum}. 

%In other terms, one can infer something about $B$ once one measures $A$, but one cannot control or reproduce that same condition for $B$ in the next trial without measuring $A$ again: 


Popper clarifies that the apparatus \qt{does \emph{not} permit the assignment of a definite position and a definite momentum arbitrarily to particular particles}{ ist charakteristiseh, dab es uns nicht gestattet, bestimmten TeiIehen einen bestimmten Ort und Versuchsschema. einen bestimmten hnpuIs willkfirlieh zu erteilen} \citep[807]{Popper1934}. Repeating the experiment many times would yield a statistical distribution of $AB$ particle pairs, all with \emph{different} but correlated positions and momenta, not an ensemble of $B$ particles all having the \emph{same} position and momentum: \qt{The experiment therefore provides no indication for the production of a collection of particles that is more homogeneous than a \emph{pure case}}{Das Experiment gibt also keinen Anhaltspunkt zur Herstellung einer Teilchenmenge, die homogener ist als ein reiner Fall} \citep[807]{Popper1934}. 

Popper can then conclude that the \TE\ shows that one can determine a predictive measurement of the position and momentum of a single $B$-particle without allowing for the selection of an ensemble of $B$-particles having both sharp momentum and sharp position. While fully complying with Heisenberg’s inequalities understood as statistical dispersion relations \origg{Streuungsrelationen}, Popper believed that one could nevertheless produce a case that violates Heisenberg’s inequalities understood as limits of measurement accuracy \origg{Ungenauigkeitsrelationen}. Against the indeterministic metaphysics he saw as prevailing, Popper could then invite his reader to conclude that each particle in the ensemble possesses a \emph{pre-assigned, yet unknown} momentum and position, thus opening the possibility that deterministic \s{precision laws} might underlie the \s{quantum frequency} laws \LF{\S78}.

%Teconstructing the \emph{past of an electron} via an arbitrarily sharp non-predictive measurement together with a conservation law appears to allow an arbitrarily predictive measurement of the \emph{future of a light quantum}. In other terms, reconstructing the \emph{past of an electron} via an arbitrarily sharp non-predictive measurement together with a conservation law appears to allow an arbitrarily predictive measurement of the \emph{future of a light quantum}. 


\subsection{The issue of the Momentum Measurement}
\label{sub:momentum}

As already noted, Popper was aware that the device underlying this \TE works only if the momentum of $A$ remains unaffected by the selection at $X$, allowing the entire path of $A$ to be reconstructed up to the collision at $S$. Popper was probably initially confident of this, relying on \posscite[15]{Heisenberg1930} somewhat elliptical remark (see above \cref{sec:past}). An unpublished \emph{Ergänzung} to the Note, written sometime between summer and fall 1934, shows that Popper soon realized a legitimate objection could be raised against this premise: \qt{One could, namely, take the following view: Non-predictive measurements allow us to make exact calculations only for certain intervals, namely for the interval \emph{between} two measurements. For the time \emph{before} the first or \emph{after} the second measurement, however, the uncertainty relation holds, whose validity is therefore (contrary to Heisenberg’s remark) symmetrical for past and future}{Man könnte nämlich folgende Auffassung vertreten: Nichtprognostische Messungen gestatten uns eine genaue Berechnung nur für bestimmte Intervalle, nämlich für die Intervalle zwischen zwei Messungen, etwa zwischen zwei Ortsmessungen oder zwischen einer Orts- und darauffolgenden Impulsmessung. Für den Zeitpunkt vor der ersten oder nach der zweiten Messung gilt die Ungenauigkeitsrelation, deren Geltung also (im Gegensatz zu Heisenbergs Bemerkung) für Vergangenheit und Zukunft symmetrisch wäre.} \citep[397\psq]{Popper1934a}. 

%Popper neded at least ot tecncal qualsation abotu the measurement of momentum $\mathfrak{a_2}$. (1) precise measurement in the $SX$-direction is possible if both the magnitude of momentum and the time of arrival at $X$ can be measured accurately unaffected by the diffraction of the particles resulting from the restriction of the $SX$-direction; (2) a technical discussion of the instruments involved in the measuring of the momentum, that also efore the mome, that leave the position ifactec.  

As the reader may recall, this was precisely the conclusion reached in the ETP paper (see above, \cref{sub:ETP}), which Popper, without any doubt, did not know. Yet Popper sought to reach the opposite conclusion. He therefore had to offer at least a technical clarification concerning the measurement of momentum:\begin{inparaenum}[(a)] \item one can make the interaction region $S$ arbitrarily narrow ($\Delta \mathfrak{S} \to 0$) and still deduce $\mathfrak{b}_2$ exactly from momentum conservation—so that $\Delta \mathfrak{b}_2 \to 0$, once the magnitude of the momentum $\mathfrak{a}_2$ at $S$ is reconstructed; this analysis was expanded in Appendix~VII. \item The path of the $A$-particles can be reconstructed even \emph{before} the first momentum measurement, up to $S$. This discussion was later incorporated into Appendix~VI. \end{inparaenum} I will consider here only point~(2). Indeed, with some good will, one might concede (1), since $\Delta S$ and $\Delta \mathfrak{b}_2$ are not conjugate variables. However, (2) proves to be the true Achilles’ heel of Popper’s setup.

\todo{assumed that is pre asssigned rather than proving it}

As we have mentioned, Popper initially proposed to place the film strip at $X$ to measure not only position and time, but also the energy and therefore momentum of the incoming particle. Popper soon realized that this apparatus would obviously violate the \IR \citep[400]{Popper1934a}\footnote{See also Einstein's objection, \cref{sub:einsteinLdF}}. Thus, in the \ERG, Popper proposed a revised setup, that was then included in Appendix VI of \LdF. It can be describe as follows: \begin{inparaenum}[(1)] \item At $X_1$ one might place an electrostatic analyzer, a filter made by parallel plates with electric field $E$ normal to the beam direction\footnote{See below \cref{fig:weizsaeckerelectrical}}. The latter is supposed to filter a sub-ensemble of electrons with the same, \emph{predetermined} momentum $\mathfrak{a}_2$ \emph{without changing it} \LF{220\psq}. Since, in Popper's view, a filtering of the momentum leaves the momentum of the electron unchanged, it must also leave its position unchanged, even if unknown \item the unknown position may later be disclosed by a second measurement at $X_2$, a Geiger counter (or moving film strip) records position $x$ of the $B$-particles and arrival time $t$. Since the first measurement left the state of the electron unchanged, one might think that it is possible to reconstruct the past of the electron not only \emph{between} the two measurements---momentum followed by position---but also \emph{before} the first measurement at~$X_1$\end{inparaenum}.

Popper realized that two possible objections can be raised against the possibility of a momentum measurement as described:

\begin{itemize}
\item \emph{The filter leaves momentum unchanged but changes the position}. Popper argues that, unlike a position selection (a narrow slit) which spreads the trajectory, a momentum selection (an electron spectrometer) lets particles of a certain momentum pass along the same path while blocking the others. Assuming that momentum selection unpredictably disturbs position would imply that the particle jumps discontinuously (with superluminal speed) to another point along its path. This, however, contradicts quantum mechanics, which allows discontinuous jumps only for bound, not for free, particles. Popper refers to any theory that introduces such position disturbances as an \s{imprecision theory} \origg{Ungenauigkeitstheorie}; it is logically possible, but ultimately empirically indistinguishable from standard \qm. Popper concluded that according to \IR interpreted statistically, the momentum selection leaves particle positions \emph{unknown} but unchanged \citep[401]{Popper1934}. 

\item \emph{If the filter left momentum unchanged one can use it to produce a \s{super-pure case}}. By reversing the order of selection, one could indeed first localize the particle very precisely narrow slit or short \s{momentary shutter} (\german{Momentverschluß}), then select its momentum with a filter\label{shutter}. Since the latter seems to leave both position and momentum unchanged, one might expect, after many runs, to produce an ensemble with sharp~$\pos$ and~$\mom$. Popper counters that, contrary to the momentum selection that leave the position unchanged, the position selection spread the momentum: the sharper the initial position, the larger the diffraction the fewer particles pass the filter. Only rare, random detections occur, revealing merely the \emph{statistical distribution} of position and momentum of an \s{anonymous} ensemble, not sharp individual trajectories of this or that particle 
\end{itemize}

Popper’s reasoning rests on an alleged asymmetry between position and momentum selection. He treats momentum selection as non-disturbing and non-anonymizing (since he needs to reconstruct the path of a particular $A$ particle); by contrast position measurement is disturbing and anonymizing (to avoid the possibility of a super-pure case). As we shall see, this asymmetry is, at best, only apparent. What is relevant here, is that Popper clearly sees that the question of the momentum measurement decides the destiny of his \TE\: \q{[a] Assuming that the $x$-coordinates of the particles are not disturbed by the momentum measurement, then the exact determination of position and momentum also extends to the time \emph{before} the momentum selection [at $X_1$]. [b] Assuming that the momentum selection disturbs the $x$-coordinates, then we can calculate the trajectory exactly only for the time \emph{between} the two measurements [at $X_1$ and $X_2$]} \citep[220]{Popper1935}. As we shall see, Popper was compelled to respond to several interlocutors---some of the greatest physicists of all time---who challenged the legitimacy of assumption~[a] and argued that~[b] was the case. He eventually gave it up, but not without a fight.

\section{The Leipzig Group Reaction}
\label{sec:leipizg}

\LdF\ was completed and sent to the publisher in the fall and printed at the end of December 1934 (despite the colophon bearing the date 1935). In the intervening months, Berliner, the editor of the \textit{Naturwissenschaften}, had sent Heisenberg the galley proofs of the short note that Popper had submitted in August. Heisenberg wrote to Popper on \datef{23}{11}{1934}, explaining that in response to Berliner’s request, he discussed the issue raised by Popper with his assistant \vW. They subsequently sent Berliner a brief note written by the latter outlining their critique, asking him to forward it to Popper. \q{You will then see what kind of criticism it is. Essentially, it concerns the concept of the \s{non-prognostic measurement,} which in our view appears to be misunderstood in your work} \letterKPAp{Heisenberg}{Popper}{23}{11}{1934}[305-32]. Let us not forget that Popper was virtually unknown at the time, and it is quite remarkable that Heisenberg (newly appointed to a chair at Leipzig) and his group chose to engage in a detailed refutation of various variants of his \TE rather than dismiss him as a crank.


\subsection{\VW's Objection and Popper's Reply}
\label{sub:weizobjection}

%Diese Bahn ist jedoeh prinzipiell unkontrolIierbar. 

In his brief response, \vW, as Heisenberg did, suggest to use a scattering method, that is the Doppler effect, to measure the momentum of electron at $X_1$ and then a Geiger counter measure the position at $X_2$. With this setup, \vW arrived precisely at the conclusion that Popper wanted to avoid. The electron \s{trajectory} [A] can be reconstructed only for the interval \emph{between} $X_1$ (momentum determination) and $X_2$ (position recording), but it \emph{cannot} be extended \myemph{before} $X_1$. \vW concedes that it is possible to know the momentum before and after the measurement exactly; however, in order to measure the momentum with precision via Doppler effect on needs a  precise frequency determination ($\Delta \nu$) of the scattered radiation requires a finite observation time ($\Delta t$); improving one worsens the other. Thus one does not know at what moment within this time interval the impact between the measuring apparatus and the object of measurement occurred—that is, for how long the object had the \s{momentum before the impact} and for how long it had the \s{momentum after the impact}. This is sufficient to blur the knowledge of position $X$ after the impact back to the time before the impact. Hence, the time of the collision at $S$ remains undetermined, and the trajectory [B] cannot be predicted. \vW concluded: 


% Es ist zwar möglich, den Impuls vor und nach der Messung genau zu kennen, aber eine Impulsmessung nimmt, je genauer sie ist, umso längere Zeit in Anspruch, und man weiss dann nicht, wann innerhalb dieses Zeitintervall[s] der Stoss zwischen Messapparat und Messobjekt stattgefunden hat, d. h. wie lange das Messobjekt den „Impuls vor dem Stoss“ und wie lange es den „Impuls nach dem Stoss“ hatte; dies genügt gerade, um eine Ortskenntnis nach dem Stoss, die man durch den Messprozess hindurch bis in die Zeit vor dem Stoss zurückverfolgen will, hinreichend zu verwischen.

%Diese Bahn ist jedoeh prinzipiell unkon-
%trolIierbar. Sie gilt n~imlich nur ffir das Zeitintervali zwischen
%dem Ende der Impulsmessung und denl Beginn der Orts-
%messung, in dem das Teilchen iiberhaupt keine Weehsel-
%wirkung mit seiner Umgebung hat, und EiBt sich nieht in
%den Zeitraum vet der Impulsmessung fortsetzen, da diese
%ihrerseits die Kenntnis des Orts gem/ifl der Ungenauigkeits-
%relation zerstSrt.


\qt{This trajectory, however, is \emph{in principle} uncontrollable. It is valid only for the time interval between the end of the momentum measurement and the beginning of the position measurement, during which the particle has no interaction whatsoever with its surroundings, and it cannot be extended into the period before the momentum measurement, since the latter itself destroys the knowledge of the position in accordance with the uncertainty relation. \textelp{} Thus, even if the position [at $X_2$] after the momentum measurement is exactly known, one still cannot infer the position \emph{before} the momentum measurement [at $X_1$] with an accuracy greater than the error $h / 4 \pi \Delta p$. That is, our knowledge of the trajectory of particle $A$ before the momentum measurement at $X$, and thus also of the trajectory of particle $B$ after the collision at $S$, is in accordance with the uncertainty relation}{Diese Bahn ist jedoeh prinzipiell unkontrolIierbar. Sie gilt n~imlich nur ffir das Zeitintervali zwischen dem Ende der Impulsmessung und denl Beginn der Orts-
messung, in dem das Teilchen iiberhaupt keine Weehselwirkung mit seiner Umgebung hat, und EiBt sich nieht in den Zeitraum vet der Impulsmessung fortsetzen, da diese ihrerseits die Kenntnis des Orts gem/ifl der Ungenauigkeitsrelation zerstSrt. }[\vW in \cite[808]{Popper1934}]
%
\vW concedes that the proof applies only to the \emph{specific} experimental setup he described. However, he no reason to doubt that \emph{any} other combination of measuring devices at $X$ would yield the same result. The problem is that Popper does not distinguish \s{non-predictive measurements} and verifiable measurements concerning the past. The former escape the \IR since are not physical measurement at all, the latter on the contrary do satisfy the \IR which are symmetric for the past and future.

On \datef{26}{11}{1934}, Popper replied by sending the galley proofs Appendices VI and VII of the \Lo to Heisenberg \letterKPAp{Popper}{Heisenberg}{26}{11}{1934}[305-52]. Popper, somewhat prone to paranoia, felt he was not taken seriously by \vW. He emphasized two points: \begin{inparaenum} \item in the first note he did not \emph{prove} the legitimacy of \emph{calculating the past of the electron}; he \emph{assumed} it, referring \emph{Heisenberg}’s own remark in his book as warrant. \item von \emph{Weizsäcker} refuted only one \emph{method} of \emph{momentum measurement} of individual electron via scattering method. Popper maintained that \emph{momentum selection} of an aggregate of electrons \origg{Teilchenmenge} via filters or spectrometers can determine momentum \mom \emph{without} disturbing the \pos-coordinate, thus allowing for a non-predictive retrodiction before $X_1$. \end{inparaenum} Popper wrote to Heisenberg:

\qt{You will now ask why I did not elaborate further in the note on what I am writing here. The reason is that I assumed that the so-called \s{non-prognostic measurement} (momentum selection followed by a position measurement) and the possibility of calculating the \s{past} prior to the momentum measurement were already known. This assumption was based on a remark of yours (Phys. Principles, p. 15) concerning case (b) and the \s{calculation of the past of the electron}. In my note, however, I had to be brief and restrict myself to what was essential, that is, above all, to what was new. Only from Mr. Weizsäcker’s note did I realize that at least he was not familiar with this case}{Sie werden nun fragen, warum ich des, was ich hier schreibe, in der Note nicht näher ausgeführt habe. Der Grund ist, der, dass ich annahm, dass die fragliche \s{nichtprognostische Messung} (Impulsaussonderung mit nachfolgender Ortsmessung) und die Möglichkeit, die \s{Vergangenheit} vor der Impulsmessung zu berechnen, bekannt ist. Diese Annahme gründete sich auf eine Bemerkung von Ihnen (Phys. Prinzipien, S. 15) über den Fall (b) und die \s{Rechnung über die Vergangenheit des Elektrons}. In meiner Note musste ich mich aber kurz fassen und nur das Wichtigste bringen, also vor allem das Neue. Erst aus Herrn Weizsäckers Note bemerkte ich, dass zumindest ihm dieser Fall nicht bekannt ist, und ich muss nun annehmen, dass er entweder auch Ihnen nicht bekannt war (und ich aus der zitierten Stelle Ihres Buches zuviel herausgelesen habe), oder dass Sie, was ich sehr begreiflich finden würde, meine Note nicht recht ernst genommen haben, so dass Ihnen dieser Zusammenhang entgangen ist}[\letterKPAp{Popper}{Heisenberg}{26}{11}{1934}[305-52]] 
%
Since he mentions p.~15 of the German edition of Heisenberg's 1930 book, Popper probably refers to Heisenberg's claim that the \s{calculation of the past of the electron} can be reconstructed \emph{before the position measurement}---without specifying whether it can also be reconstructed before the momentum measurement. Only after reading von~Weizsäcker’s note did he realize that at least that the latter \q{was not familiar with this case} \letterKPAp{Popper}{Heisenberg}{26}{11}{1934}[305-32]. He now had to assume either that it was also unknown to Heisenberg (and that he had read too much into the cited passage) or that Heisenberg, as he could well understand, had not taken his note very seriously, so that this connection had escaped his attention.

%(406)
On \datef{30}{11}{1934}, the Note appeared in volume 22 of \NW, together with \vW's objection. The time was too short for Popper to prepare a rejoinder. On \datef{6}{10}{1934}, Popper sent Heisenberg a second communication \citep{Popper1934b} in response to \vW, which he hoped to publish. Popper also attached a copy of the \LdF that was published but still not distributed. Popper's counterargument is revealing of his way of thinking at that time\footnote{As we shall see, his reconstruction of the event twenty years later seems to me quite different}. Popper observed that \vW\ had used momentum \emph{measurement} at $X_1$ via the Doppler effect, that is by using light quantum collision. The latter, however, has the shortcoming that it \qt{disturbs the particle’s momentum (recoil) and that, since the exact instant of the disturbance is not known}{den Impuls des Teilchens stört (Rückstoß) und daß, da der genaue Zeitpunkt der Störung nicht bekannt ist} (406). Popper concedes that that no reconstruction the momentum before the scattering event is possible in this way. However, following the same argument he had developed in Appendix VI, Popper suggests that one can resort to \q{a momentum \emph{selection} (electron spectrometer, light filter)}, which \qt{unlike the momentum measurement of individual particles, for example by means of the Doppler effect---has the property of not affecting the momenta or momentum components of the selected particles}{eine Impuls\emph{aussonderung} (Elektronenspektralapparat, Lichtfilter) hat nämlich-- im Gegensatz zu der Impulsmessung individueller Teilchen, etwa mit Hilfe des Dopplereffektes -- die Eigenschaft, die Impulse bzw. die Impulskomponenten der ausgesonderten Teilchen nicht zu beeinflussen}. Thus, if $[A]$ is an electron beam, \qt{only electrons with a certain \myemph{predetermined} amount of momentum enter the send counter}{nur Elektronen mit einem gewissen vorgegebenen Impulsbetrag in den Spitzenzähler einfallen}, and if $[A]$ is a light beam, the filter lets through only those light quanta that \emph{already} possessed the required momentum before entering it \citep{Popper1934b}.

One might object, as Popper himself anticipates, that we do not really know in detail what a light filter does, and that its operation might itself disturb the light quantum's state. However, Popper argues, a filter produces not only light but also sharp images \origg{Bilder}—thus, it does not, in any case, disturb the directions of the quanta. If the filter did disturb the light quanta, one would have to assume some mechanism that restores their direction after passage: either (1) the quanta pass through the filter but are unpredictably displaced along their paths, with their previous direction suddenly reestablished \origg{Zurückversetzung auf der Bahn}, or (2) the filtering process involves absorption and directed re-emission \origg{gerichtete Emission} in the same direction. Yet both options appears problematic: the first contradicts the continuity required by quantum theory; the second assumes a kind of \s{directed fluorescence} not known to occur. Popper therefore concluded that the most plausible hypothesis is that, also in the case of the filter, the quanta with the \s{right} momentum are allowed to pass through unchanged. thus the position.

%Once again, if the momentum remains unchanged, then the position, although it remains unknown cannot have been disturbed; otherwise, as we have seen, some kind of action at a distance would have to be introduced. If this is the case, the path before the momentum measurement at $X_1$ can be reconstructed once the final position is determined via the Geiger counter at $X_2$.

\subsection{Heisenberg’s and \vW’s Counter-Replies}
\label{sub:heisobjection}

\begin{figure}
\centering
 \includegraphics[scale=0.3, trim = 0mm 0mm 0mm 0mm, clip]{1934WeizsaeckerElectrical.png}
 \caption{- D: The slit system (diaphragm) of width $d$. This restricts the transverse position of the incoming particle beam.
- A and B: The plates of a capacitor (or deflecting field) that bend the particle trajectory.
- C: The curved trajectory of the particle inside the field.
- E-F: The vertical displacement $y$ at the detector plane (the "screen" at the right).
- $L$ : The free flight distance $L$ (the length of the field region).}
\label{fig:weizsaeckerelectrical}
\end{figure}

On \datemy{6}{12}{1934}, the same day of Popper's letter, \vW drafted a detailed response which Heisenberg forwarded to Popper on \datemy{10}{12}{1934}. \VW\ considered the setup suggested by Popper in Appendix VI, an electrostatic analyzer, (see Fig.~\ref{fig:weizsaeckerelectrical}), and showed the very  geometry of the deflection experiment already contains the limits expressed by  the uncertainty principle\footnoteh{(See appendix \cref{app:1} for more details)}. To determine the momentum of the particle from its impact point on the screen $EF$, the beam must pass through diaphragms that fix its path length precisely. But narrowing the slit to improve accuracy inevitably  produces diffraction, so that the transverse spread $\Delta y=EF$ of the beam at the detector cannot be made smaller than a definite bound. In other words, the position on the screen, which is supposed to measure momentum, is blurred by the  same diffraction effects introduced by the diaphragms.  

Since the vertical displacement $y$ depends on the flight time $t=L/v$, this 
transverse uncertainty also propagates into the longitudinal motion. An 
uncertainty in $y$ translates into an uncertainty in the velocity $v$, and thus 
into both the longitudinal momentum $p_x$ and the coordinate $x=vt$. When these  effects are consistently taken into account, one finds that the product 
$\Delta p_x \Delta x$ is likewise bounded below by Planck’s constant. \cref{fig:weizsaeckerelectrical} illustrates the intuition: the more tightly the beam is collimated at $D$, the  more it spreads out when it reaches the screen at $E\!-\!F$, so that no  arrangement of diaphragms or fields can evade the uncertainty relation. The situation was not dissimilar to the one described by \citet[21\psq]{Heisenberg1930} in his book, where he used a homogenous magnetic field instead of an electric field to measure the momentum.

In his reply of \datedm{10}{12}{1934}, Heisenberg enclosed \vW's criticism of the electrostatic analyzer, and also addressed Popper’s suggestion of using a light filter instead of an electron spectrometer \letterKPAp{Heisenberg}{Popper}{10}{12}{1934}[305-32]. Heisenberg suggests that one might use reflective resonance filters, such as sodium or mercury vapor, following Wood’s classical residual-ray experiments \origg{Reststrahlenverfahren}. When light strikes such a medium, radiation at the resonance frequencies is strongly absorbed and re-emitted (or reflected), while other frequencies pass through unaffected. Thus, a broad-spectrum beam incident on sodium vapor yields only the reflected sodium D-lines, making the medium act as a natural frequency filter. Heisenberg invoked Wood’s experiment to show that resonance scattering always involves finite line widths, not infinitely sharp spectral of a single frequency $\nu$. When an atom absorbs a light quantum at its resonance frequency, it is excited to a higher state and, after a short lifetime $t$, decays by re-emitting a light quantum at the same frequency. Because the excited state exists only for a finite time, its energy cannot be perfectly sharp in compliance time--energy uncertainty relation.

Even when light is \s{filtered} into sharp momentum (frequency) states, the finite lifetime of the excited atoms ensures a residual spread $\Delta \nu$ (and hence $\Delta p$). This constitutes a momentum selection, since light quantum momentum is $p = h/\lambda$, but its precision is limited by the finite resonance linewidth, determined by the lifetime of the excited state, $\Delta t = 1/\Delta \nu$. Thus, light quanta cannot be prepared with arbitrarily sharp momentum. Even if the light quantum’s later position is known precisely, its position before reflection cannot be reconstructed, because the time spent in the reflecting atom is indeterminate. Hence there is an uncertainty $\Delta x$ such that $\Delta x \cdot \Delta p \sim h$. The filter selects not an exact frequency, but a narrow band, whose spread is intrinsic to quantum mechanics through the lifetime--linewidth relation.

In his reply of \datedm{16}{12}{1934} Popper attached a revised version of the second communication summarizing the main point of the debate \citep{Popper1934c}. He conceded that \vW's argument against the electrostatic selector was correct. Nevertheless, he defended his idea of using a light quantum filter against Heisenberg’s objections. He clarified that Heisenberg’s remarks on Wood’s experiment concerned only the absorbed light at the resonance line, whereas \q{the light of other frequencies \myemph{passes through}, and it is only to this transmitted light that my considerations apply} \letterKPAp{Popper}{Heisenberg}{16}{12}{1934}[305-32]. In other terms, Popper stressed that his argument did not involve absorbed and re-emitted light quanta, but only the transmitted ones. Only if one assumes that the transmitted light also underwent absorption and re-emission, would the experiment lose its validity. However, it is the common view that filters simply transmit, not re-emit, light. He also dismissed the claim that the transmitted light was too inhomogeneous: \q{It is in fact possible, by means of a set of filters, or even with a green filter alone, to isolate a finite range of transmitted frequencies $\Delta \nu$, i.e.\ a finite $\Delta p$ interval} \letterKPAp{Popper}{Heisenberg}{16}{12}{1934}[305-32]. Once the light quantum passed through the filter, its position could be later measured with arbitrary accuracy $\Delta x \to 0$, so that, in principle, \q{the \emph{product} of the two uncertainties could approach zero, even for imperfect filters} \letterKPAp{Popper}{Heisenberg}{16}{12}{1934}[305-32]

Popper did not relent. He proposed a new idea for an apparatus, which Heisenberg left to his assistant Hans Euler to examine \citep[293-29]{KPA}. On \datef{4}{2}{1935}, Euler sent Popper a detailed refutation. On \datemy{12}{2}{1935}, Popper drafted a somewhat impatient response that was never sent, and another on \datef{14}{4}{1935}. I will not go into this exchage, since Popper’s original proposal has been lost, and he seems to have later abandoned the idea. At this point however, the Leipzig group was starting to become mildly annoyed by Popper's insistence. In a mid-March letter to Grete Hermann, \vW, at her request of clarification, reconstructs the entire series of experimental conjectures and refutations, concluding somewhat humorously that it has become a \q{a parlor game for us in Leipzig, to refute Popper’s setups} (\letter{\vW}{Hermann}{13}{3}{1935}, in \cite[Brief 15]{Herrmann2019})\footnote{The correspondence between \vW and Hermann, and \posscites{Hermann1935}{Hermann1935a} take on Popper's interpretation of \qm would require a separate investigation}. It was Heisenberg, who ultimately decided that it was time to bring the game to an end a few days later. 

In his final letter to Popper, on \datedm{19}{3}{1935}, Heisenberg explained that all the considerations of his previous letter apply to transmitted light as well. \letterKPAp{Heisenberg}{Popper}{19}{3}{1935}[305-32]. He considered a substance opaque over nearly the entire spectrum due to strong absorption bands, yet transparent within a narrow region between two band heads \origg{Bandk\"opfe}. Light quanta with frequencies inside this window pass quickly, while those near the band heads are absorbed and re-emitted after a delay comparable to the lifetime of the excited states. As the interval between the band heads narrows, more quanta experience delayed re-emission, lengthening and blurring the passage time. In the limit of an infinitesimally narrow band, the mean delay grows without bound, consistently with the Fourier relation $\Delta \nu \Delta t$: a sharp frequency (or momentum) definition entails an indeterminacy in time (or position) \letterKPAp{Heisenberg}{Popper}{19}{3}{1935}[305-32]. At this stage, Popper might have sought support from Weisskopf. Yet the latter confessed that he had met with Heisenberg and had come to side with him and \vW \letterKPAp{\Weis}{Popper}{21}{1}{1934}.  The game had run its course—but Popper still had one card up his sleeve.


%https://cqi.inf.usi.ch/qic/grete.pdf

\section{The Popper--Einstein Correspondence}
\label{sec:einstein}

On \datef{25}{3}{1935}, \citet{Einstein1935} submitted the EPR paper the \emph{Physical Review}; it was published in \datemy{1}{5}{1935}. In the intervening weeks, Popper managed to send Einstein a copy of \LdF. Popper’s friend the pianist Rudolf Serkin, performed with the Busch Quartet and had recently married the daughter of Frieda and Adolf Busch, who knew Einstein. Frieda sent him a copy of Popper's book on \datef{28}{4}{1935} \letteraeap{Busch}{Einstein}{28}{4}{1935}[34--338]. As one can see, Einstein received Popper’s work after the EPR paper had already been submitted. \citets{Jammer1974} hypothesis that the Popper's paper might have served as intermediary between the ETP and the EPR paper is intriguing. As we have mentioned, Popper's experiment is nothing but a version of the ETP experiment in which replace energy and time are replaced by momentum and position as in the EPR argument. However, this hypothesis seems ultimately unlikely for chronological reasons\footnote{I do not find evidence that Einstein already sent his 1934 note to Einstein already in December}. However, it is plausible that Einstein immediately recognized that Popper’s experimental setup was essentially a variant of his own ETP scheme and thus immediately dismissed it.

\subsection{Einstein's Comment on the \LdF}
\label{sub:einsteinLdF}

Einstein replied to Popper on \datedm{15}{6}{1935}, endorsing his general philosophy of science\footnote{This was possibly more than a merely rhetorical endorsement. In a 1984 letter to Popper, John Stachel, at that time the editor of Einstein's \emph{Collected Papers} pointed out that, in a short newspaper article, \citet{Einstein1919-12-25} had anticipated the main lines of his philosophy, combining deductivism with the invention of hypotheses and their falsification as a means of control \citep[292-12]{KPA}} but criticizing the proposed thought experiment as flawed: \q{It is not correct that place and momentum of the particle at $\mathrm{Y}$ (p. 179) can be predicted. To this end you would have to measure with the help of your \s{Film} time and momentum of the particle [at $X$]  (that is, time and energy) which is impossible. You will see this easily if you think a little more about it} \letteraeap{Einstein}{Popper}{15}{6}{1935}[19-124]. The objection is probably the consequence of Popper's unfortunate choice of measuring apparatus on p.~179 of the \LdF indeed seems to suggest that the \s{film} measures at $X$ momentum, and position and time of arrival. 

However, has we have seen, in Appendix~VI of the \LdF Popper already uses two separate apparatuses, a momentum filter at $X_1$ (electron spectrometer or light filter) and a position-time measuring apparatus (a film) at $X_2$. Still Popper, in his lengthy reply on \datedm{18}{7}{1935} by that time had already accepted that even this setup could not work:

\qt{First, the question of the thought experiment (in the book p. 179). Unfortunately, regarding the measurement of the particle arriving at $X$—as you rightly noted—there are quite a few inaccuracies in the text of the book (for example, the use of the electric field and practically the whole Appendix VI). I admit my mistakes, but after having discussed them thoroughly with all the physicists available to me in Vienna, among others with Professor Thirring and his assistant Guth (with Professor Heisenberg and his assistant Weizsäcker I discussed the matter in writing), it still seems to me that the issue is not yet settled. From the discussions so far (and in this view I am not alone, but, among others, also Guth), the impossibility of my thought experiment has not yet been demonstrated. That is, it seems possible to avoid the aforementioned errors and to maintain the thought experiment in agreement with quantum mechanics. I do not wish to be \s{proven right}; I would only be very glad if this matter could be clarified — even if the final decision goes against me. I hope I am not abusing your time and patience too much by presenting the situation as it currently appears to me}{Zunächst die Frage des Gedankenexperiments (im Buch S. 179). Hier stimmt leider bezüglich der Messung des bei $X$ eintreffenden Teilchens - wie Sie js bemerkten - im Buch in der Tet einiges nicht (z.B. die Verwendung des elektrischen Feldes und so ziemlich der genze Anhang VI). Ich sehe meine Fehler ein. ther wohdem ich mit :llen fur mich in Wien erreichbren Physikern, u.n. mit Professor Thärring und dessen Assistenten Guth eingehend diskutiert habe (mit Professor Heisenberg und dessen ssistenten ïeizsäcker hebe ich schriftlich diskutiert), scheint mir die Bache noch imer so zu sein, duss us der bisherigen Diskussion (dieser 'nsicht bin nicht nur ich, sondera u.n. ruc Guth) die Unasglichkeit neines Geds niceuexperinents noch nicht hervorgel d.h. es scheint möglich zu sein, die erwähnten schler zu vermeiden und dss Gedrnkenexperiment rls mit der uentennechnnik vereinbs rufreoht zu erholten. Ich mocht e nicht "Recht behr lten"; ich wäre nur sehr froh, wean diese Sache zu einer Klarung küne, - ruch drnn, wenn die Katsoheidung gegen mich fillt. Ich hoffe, Ihre zeit und Ihre "eduld nicht llzu seh zu missorauchen, wem ich thon die itustion, wie sie mir geomartig erscheint, drrstelle.}[\letteraeap{Popper}{Einstein}{18}{7}{1935}[19-126]]

Popper now tried to pursue the arrangement in which a light filter is placed at $X_1$. He acknowledged that this setup—the optical version of the one described in the book—would not yield the desired result by using known filters (as Heisenberg showed him). However, Popper suspected that a combination of filters exist that select photons of a sharply defined momentum (frequency) without having them interact for a long time with the medium. 

Popper went into some details, however, so as not to make an already an eight-page letter even longer, Popper promised to send Einstein, as a supplement, which he had written two months earlier at the suggestion of Felix Ehrenhaft \letteraeap{Popper}{Einstein}{18}{7}{1935}[19-126]. Indeed, throughout the spring of 1935 he continued to work on a published response to Leipzig objections that he ultimately expanded the longer manuscript entitled once again \german{Zur Kritik der Ungenauigkeitsrelationen}. He sent it to Carnap, Ehrenhaft, Heisenberg, and Schrödinger, hoping that Ehrenhaft would forward it to Einstein (\letter{Popper}{Carnap}{10}{6}{1935}; Carnap Collection). On \datedm{18}{7}{1935}, he decided to send it himself to Einstein on. Popper confessed he was beset by doubts. However, after further discussion with Weisskopf, Popper realized that the latter could not fully refute his argument. Since he had already announced the manuscript some time before, Popper felt compelled to send it despite the still unresolved situation, having more or less given up hope of a prompt and complete clarification. Moreover, he had heard from Weisskopf about Einstein’s recent paper critical of quantum mechanics (that is, the EPR paper), which might have further encouraged him to finally send the manuscript \letteraeap{Popper}{Einstein}{29}{8}{1935}[19-127]

\todo{\footnote{In his commentary to Einstein's letter Popper claims that Einstein was working on the same problem}.}.

%; \letter{Popper}{von Mises}{26}{6}{1935}, Popper Archives [329.4]
%Do 30 12h plötzlich Popper. Sein Buch ist im Druck; er erzählt von neuer Deutung der Unbestimmtheitsrelation; 

\subsection{Popper's 1935 Manuscript}
\label{sub:1935ms}

The first part of Popper's 1935 manuscript restates his stance toward the \UR\ with additional reference to some literature that he had not cited in the book, such as von~\citep{Neumann1932}'s textbook. Thus, for example, Popper began to use the expression \s{dispersion-free ensemble} to indicate what he had previously called a \s{super-pure case}. There are surely difference in the presentation, but the main message remains the same: the prohibition of a dispersion-free ensemble does not imply the prohibition of arbitrarily sharp measurements. Popper presupposes that reader already knows his November 1934 Note \todo{When did he attached the note?} and once again identified clearly issue with his original setup:

\qt{One would have to apply a \s{non-prognostic measurement} according to case (b)—a position measurement [at $X_2$] preceded by a momentum measurement at [at $X_1$]—arranged in such a way that one can calculate the trajectory not only in the interval \myemph{between} the two measurements, but also for the time \myemph{before} the momentum measurement. The position coordinate is measured precisely by the peak counter (the moment of incidence [at $X_2$]). \emph{Question: Is there a method of momentum measurement [at $X_1$] that could precede this position measurement without destroying (blurring) the position coordinate of the particle?}. Weizsäcker disputes this\footnote{See above \cref{sub:weizobjection}}; the case he discusses (momentum measurement of the particle using the Doppler effect) is indeed unsuitable. In my book I propose the following arrangement: 1. Momentum selection by means of a grating filter,   2. Position measurement by means of the counter.   There I defend the thesis that such a momentum measurement, carried out with a filter, should not disturb the position coordinates; this thesis should here be examined more closely}{}[\citep[427]{Popper1935b}]

In other terms, Popper believes he could devise a filter that could let go through light quanta with unchanged momentum and therefore position. His reasoning seems to have been roughly as follows.

In wave optics, because of the Fourier-relation, a short light pulse (small $\Delta t$) must contain a broad range of frequencies (large $\Delta \nu$), just as a beam sharply localized in space (small $\Delta x$) must contain a broad range of wavelengths (large $\Delta \lambda$). Conversely, a perfectly monochromatic wave ($\Delta \nu = 0$ or $\Delta \lambda = 0$) would have to extend infinitely in both time and space ($\Delta t, \Delta x \rightarrow \infty$). A \emph{real} filter must therefore satisfy the Fourier constraint: the narrower the transmitted range of frequencies $\Delta \nu$ or wavelengths $\Delta \lambda$—that is, the sharper the momentum selection—the longer the temporal response $\Delta t$ and the greater the spatial extension $\Delta x$ of the transmitted wave packet. Popper argues that one could in principle conceive of two different kinds of \emph{ideal} filters, corresponding to the limiting cases of the Fourier relation \citep[428]{Popper1935b}:

\begin{itemize}
\item \emph{Type I apparatus:} Filters capable of spectrally resolving arbitrarily short light flashes ($\Delta t \rightarrow 0 \Rightarrow \Delta \nu \rightarrow \infty$). Such filters must necessarily operate \emph{with spatial blurring}, since decomposing a very short pulse without spreading its position $\Delta x$ would contradict the Fourier relation:

\item \emph{Type II apparatus:} Filters that cannot resolve very short wave packets but act only on infinitely long wave trains ($\Delta \nu \rightarrow 0 \Rightarrow \Delta t \rightarrow \infty$). In principle, such filters could operate \emph{without spatial blurring}, transmitting only radiation of a single wavelength (or momentum).
\end{itemize}
%
To construct a filter of type II, Popper suggests using colored glass filters followed by gas layers \citep[429]{Popper1935b}. The glass filters, say a green filter, are passive filters that work by transmission: they allow a relatively broad band of wavelengths centered around some mean wavelength to pass while blocking both the high-frequency and low-frequency edges. Gas layers act actively, through absorption and delayed re-emission, and provide a further selection (since each gas has sharp absorption lines), ideally centered well away from any resonance edges that cause re-emission or time delay.

Popper seems to believe that the double selection would exclude the spectral region around the resonance band edges, which was the basis of Heisenberg’s objection (see above \cref{sub:heisobjection}). This would ensure that the quanta that go through the multi-stage filter are transmitted directly without delay and not absorbed and re-emitted. For this reason, one might assume that their momentum (frequency) is left unchanged. This implies that no mechanism exists that could alter their positions. The filter, Popper concludes, selects a homogeneous ensemble of light quanta with the same momentum, whose individual positions are completely \emph{unknown} but unchanged. They can then be reconstructed by a subsequent position measurement: \qt{\emph{A momentum-selection apparatus without spatial smearing would therefore not contradict quantum mechanics if it is of type (II)}}{Ein Impulsaussonderungsapparat ohne Ortsverschmierung würde also der Quantenmechanik nicht widersprechen, wenn er vom Typus (II) ist} \citep[429]{Popper1935b}.

%Popper’s reasoning rested on the assumption that a filter operates through two distinct processes: (a) \emph{absorption and re-emission}, where some light quanta are absorbed by atoms in the filter and later re-emitted (typically at the same frequency), and (b) \emph{direct transmission}, where other light quanta pass through the filter without interacting with the atoms at all.



\section{Einstein's Reply}

Einstein replied only a few weeks later. In a letter of \datef{11}{9}{1935}, acknowledging the broad direction of Popper’s argument, he rejected the details of the experimental setup he proposed:

\qt{Dear Mr. Popper,\\
I have looked at your paper, and I largely [weitgehend] agree. Only I do
not believe in the possibility of producing a ‘super-pure case’ which
would allow us to predict position and momentum (colour) of a pho-
ton with ‘inadmissible’ precision. The means proposed by you (a
screen with a fast shutter in conjunction with a selective set of glass
filters) I hold to be ineffective in principle, for the reason that I firmly
believe that a filter of this kind would act in such a way as to ‘smear’ the
position, just like a spectroscopic grid}{}[\letteraeap{Einstein}{Popper}{11}{9}{1935}[19-130]] 
%
This passage is somewhat puzzling. Einstein seems to think that Popper was proposing a method to prepare a super-pure case using a slid with shutter followed by filters. However, just like he did in the \LdF, in the manuscript Popper uses the example of the screen with fast shutter followed by a filter to \emph{exclude} that his setup would allow the construction of a super-pure case by measuring position and the momentum (see above \cref{shutter}). 

Popper's proposal was a light filter followed by a Geiger counter. Einstein may have gone through the manuscript rather quickly; yet, the last part of his objection did hit the mark. The glass filter decomposes short light pulse into quasi-monochromatic wave trains $W_n$. Absorbing filters cut out all colors $W_n$ except one, $W_1$. The filter thereby \emph{prepares} light quanta in a state with sharp momentum $W_1$, but at the cost of making position unsharp. \letteraeap{Einstein}{Popper}{11}{9}{1935}[19-130]. As Popper realized early on, if the filter smears the position, his \TE collapses. The $A$ path reconstruction stops at the momentum measurement; one cannot determine the time of collision at $S$, and thus one cannot determine arbitrarily sharp predictions of the $B$ particles via a conservation law. This is exactly the result that ETP had obtained a few years earlier.

Referring to his joint paper with Podolsky and Rosen, \citet{Einstein1935} explained that he did not have any copies at hand but could briefly summarize its content. The situation is indeed not dissimilar to that of Popper's \TE. After a collision, two particles are described by the joint \s{entangled} wavefunction $\psi(x_1, x_2)$: the total momentum of the two particles is well defined, while the individual momenta are correlated with their relative positions. If one measures, say, the momentum of particle $A$, its state collapses into the corresponding momentum eigenstate $\psi_A$; quantum mechanics then assigns to subsystem $B$ a conditional momentum eigenstate $\psi_B$. Note that, for this state, the position is completely undetermined, which blocks any attempt to beat the \UR, as Popper had hoped. Einstein’s point, rather, was that the specific form of $\psi_B$ would have been different had one performed a position measurement on system $A$. This means that two different $\psi$-functions correspond to one and the same physical state of $B$, which itself has not changed physically: \q{It is therefore not possible to regard the $\psi$-function as the complete description of the state of the system.} \letteraeap{Einstein}{Popper}{11}{9}{1935}[19-130].

%https://www.mprl-series.mpg.de/proceedings/3/6/

\todo{indirect vs. direct} \todo{specific elements of physical reality not mirrored in the theory}\todo{separation principle}

In Einstein's contemporary correspondence, the \s{private} formulation Einstein’s argument ends here, without reference to the \IR' prohibition of simultaneous measurements of position and momentum. Rather, it involves the measurement of a single variable together with a conservation law or geometrical correlation \citepp{Howard1985}[38]{Fine1986}. However, since Popper asked him to summarize the published paper, Einstein adds that since, both position and momentum of $B$ can be predicted with certainty once one knows the corresponding property of $A$. According to the infamous EPR \s{criterion of reality}, \emph{both} these quantities must correspond to something that exists in physical reality—a situation that, because of the \UR, finds no counterpart in the formalism of \qm \letteraeap{Einstein}{Popper}{11}{9}{1935}[19-130]. Although the argument was again meant to convey the incompleteness of the theory, it also seems to imply a violation of the \IR. Indeed, \citep[244\fn{$^\ast$4}]{Popper1959} would later see the EPR argument as presented in this letter as a weaker (no prediction for both position and momentum simultaneously) but correct (both position and momentum are simultaneously part of reality) version of his own failed ETP-like \TE{}.


\section*{Conclusion}

%Victor Weis~kopf to Popper, K August and 17 October 1936. Popper
%Archives (360. 21); Popper to Carnap. 5 September 1936. Carnap Collt•ction; Bohr's tt·stimonial. %Hayek Archives (44, 1).



After Einstein's objection, Popper gave up. Thanks to \Weis's mediation \letterKPAp{\Weis}{Popper}{17}{10}{1936}[360-21], he had the opportunity to meet and discuss with Bohr face-to-face in Copenhagen in June 1936 \citep[see][]{Popper1936a}, where he was invited at the last minute to the Second International Congress for the Unity of Science \citepp{Bohr1937}{Schlick1936}\citep{Frank1936}. However, at that time he felt defeated \citep[see also][]{Strauss1936}. Only at the turn of the 1950s did Popper return to the philosophy of quantum theory, however this time in the name of realism rather than determinism \citep{Howard2012}. The reflections on the topic written between 1950 and 1956 were meant to appear in a \emph{Postscript} to the English edition of the \LdF but their publication was ultimately delayed. Still in the footnotes and appendices added to the new edition (marked by starred numbers) made no secret that the his old \TE was \q{based upon a mistake}, as had been pointed out to him by \vW, Heisenberg, and Einstein: the past of an electron cannot be reconstructed without blurring its position \citep[217]{Popper1959}. 

\todo{check Margenau again}%\citep{Popper1982}

Popper concedes that \vW was correct: \q{for the electric field perpendicular to the direction of a beam of electrons} used in Appendix VI, does not act as he expected: \q{for the width of the beam must be considerable if the electrons are to move parallel to the $x$-axis, and as a consequence, \emph{their position before their entry into the field cannot be calculated} with precision after they have been deflected by the field} \citep[299\fn{$^\ast$1}]{Popper1959}. Moreover, Popper argues, \q{as Einstein shows} in the his 1935 letter\footnote{Reproduced in Appendix *xii of \citet{Popper1959}}, the same can be said \q{for a filter acting upon a light quantum} \citep[299\fn{$^\ast$1}]{Popper1959}. Popper acknowledges that the conclusion is inescapable: \q{non-predictive measurements determine the path of a particle only \myemph{between} two measurements} such as a measurement of momentum followed by one of position (or vice versa): \q{it is not possible, according to quantum theory, to project the path further back, i.e. to the region of time before the first of these measurements} \citep[242\fn{$^\ast$3}]{Popper1959}. As we have seen, Popper realized early on that, if this is the case, his \TE---and more in general ETP-like \TE{}s---\q{collapses} \citep[242\fn{$^\ast$3}]{Popper1959}.

However, in a long footnote attached to Appendix~VI as well in a new Appendix~$^\ast$XI, Popper seems to blame this mistake on the fact that he was misled by Heisenberg. According to Popper, Heisenberg \q{\emph{fails to establish that measurements of position and of momentum are symmetrical}} \citep[451]{Popper1959}. In Heisenberg’s famous $\gamma$-ray microscope thought experiment, position is determined using high-frequency light, which strongly \emph{disturbs} the electron’s momentum. Conversely, momentum can be determined, in principle, using low-frequency (long-wave length) radiation via the Doppler effect so as to \emph{avoid} disturbance; but in this case, the position remains indeterminate \citep[451]{Popper1959}. Popper claims that he had extended this asymmetry to physical selections. Position selection via a slit disturbs the momentum (producing the diffraction pattern); momentum selection via a velocity filter only makes the position \emph{unknown}: \q{But I now believe that I was wrong in assuming that what holds for Heisenberg’s imaginary \s{observations} or \s{measurements} would also hold for my \s{selections}}. Popper even argues that his \TE \q{can be used to point out an inconsistency in Heisenberg's discussion of the observation of an electron} \citep[299\fn{$^\ast$1}]{Popper1959}. 

%One can probably say that in Heisenberg’s 1927 analysis, the position measurement constitutes a genuine \s{measurement} (recording); by contrast, the momentum measurement is, in Popper's terminology, merely a \s{physical selection} of an ensemble characterized by a definite momentum distribution, while the position remains indeterminate.  However, the idea that \s{undetermined} here means unchanged, but \s{unknown} is Popper's responsibility.

Popper's recollection of the events strikes me as rather self-serving and appears to be contradicted by the unpublished textual evidence from that period. As we have seen, in his reply to Heisenberg as well in his unpublished second communication, \citet{Popper1934b} explicitly argues that the momentum determination should be occur not through momentum \emph{measurement} via Doppler effect, as \vW (following Heisenberg) had proposed, because the latter does disturb the position; on the contrary a momentum \emph{selection} via an electron spectrometer or a light filter does not. It leaves the given momentum unchanged, and therefore also the position---albeit the latter remains unknown. Popper was not misled by Heisenberg’s notion of measurement\footnote{At most, Popper was misled by Heisenberg’s ambiguous phrasing on p.~15 of his 1930 lecture notes; see above \cref{sec:past}}, but by the notion of \s{physical selection} he adopted. Indeed, paraphrasing Popper, one can say that the failure his \TE \q{can be used to point out an inconsistency} of Popper's notion of physical selection.

%—a \s{pure case} described by the same plane-wave function~$\psi$

Popper seems to have conceived of the momentum \s{filter} intuitively as analogous to a classical \s{sieve} that lets particles with the \s{right} predetermined momentum \s{pass through unchanged} while blocking the others. In Popper's setup, after the collision at $S$, the $A$-particles possess different momenta; one can select a sub-ensemble of particles with the same sharp momentum by applying, for example, an accelerating potential. However, in \qm\ such a selection does not assign a definite momentum to any \emph{individual} particle; it merely defines a new, anonymous \s{pure} ensemble of systems. It therefore makes no sense to claim, as Popper suggests, that the \emph{same} electron that passes through the filter already possessed the \emph{same} momentum beforehand. Consequently, the assertion that the position of that \emph{same} electron, although unknown, remains unchanged, cannot be upheld within the quantum-mechanical formalism. It is particularly ironic that Popper appeals precisely to this \s{anonymity} argument when discussing position measurement. Since a slit-selection does change the momentum, producing diffraction, even if we filter again to select a sub-ensemble with sharp momentum, we cannot tell whether we are dealing with the same particle as before.

\todo{reveal vs. prepare}



Even if the beam is so weak that only one electron at a time traverses the apparatus, the situation does not change: the filter \emph{prepares} rather than \emph{reveals} the momentum state. \citet[72\psq]{Popper1974} has some reason to argue that his notion of \s{physical selection} was similar in spirit to the modern concept of \s{preparation}, as outlined around the same time by \citepp{Margenau1936}{Margenau1937}. However, it is clearly not equivalent to it. \s{Anonymity} expresses the fact that a momentum preparation defines an ensemble of systems such that each member is in an eigenstate of the momentum operator corresponding to the prepared value. The position distribution of the ensemble is then completely spread, so that a position measurement would yield a different outcome each time. Conversely, if an ensemble is prepared so that each particle has a definite position, the momentum distribution of its members is completely spread. The apparent asymmetry between position and momentum preparations that Popper sought to exploit is therefore illusory. It arises from the fact that position is not an eigenstate of the Hamiltonian operator of a free particle, whereas momentum is. For this reason, momentum selection appears not to produce any analogous \s{disturbance} to that occurring in a position measurement; it merely seems to allow particles to pass through the filter with unchanged momentum. Yet the point Popper missed is that a momentum preparation still projects the system into a momentum eigenstate, rendering its position completely undetermined, since position is not an eigenstate of the momentum operator. Consequently, there is no way to combine information from both preparations to reconstruct an individual trajectory, as Popper had hoped. 

Popper realized that ETP-like arguments was based on \s{gross mistake}, however it does not seem the conceptual reason behind it. Indeed, he concluded, \q{[t]hose of my critics who rightly rejected the idea of this imaginary experiment appear to have believed that they had thereby also refuted the preceding analysis} \citep[239\fn{$^\ast$2}]{Popper1959}. In particular, Popper continues to maintain that the \CH was unwarranted, and that physical selections—that is, in modern terms, preparations—and predictive measurements can be \s{uncoupled}. Yet the failure of his experiments should have suggests that the opposite is the case: preparations and predictive measurements are inevitably \emph{coupled} in \qm. Far from being an additional or auxiliary assumption, the \CH lies at the core of \qm, which is ultimately a theory about \emph{transition probabilities} from prepared to measured states\footnote{Suppose that a system is prepared in the state $\psi(q)$, and that we inquire about the probability that, upon measurement, it will be found in another state $\phi(q)$. According to the Born rule, this probability is given by the integral $\int \phi^*(q)\,\psi(q)\,dq.$ The rule expresses the connection between the preparation of the state $\psi$ and the possible registration of the result represented by $\phi$.}. Contrary to Popper’s claim, in \qm\ (a) predictive measurements \emph{are} always preparations, and (b) non-predictive measurements\footnote{A measurement that reveals an outcome but does not leave behind a system in a well-defined post-measurement state} always \emph{presuppose} a prior preparation. Predictive measurements, in the sense assumed by Popper, are not allowed in \qm.

Popper acted as a kind of \pemo\ constructor against the \CH, a device that able uncouple preparations and predictive measurement. After repeated failures of constructing such a device, instead of concluding that the \CH\ holds, Popper continued to try to modify his setup by searching for a different filter\footnote{One might argue indeed, the non-falsification of energy principle acts as its corroboration}. Even when he was finally forced to admit that the problem plagued all ETP-type experiments, he still remained convinced that he could simply \q{replace my invalid experiment} \citep[232\fn{$^{\ast\ast}$}]{Popper1980} of the ETP-type with a different valid EPR-type experiment. This was ultimately the plan that \textcites{Popper1967} pursued when he returned to work actively to quantum physics. Indeed, in the 1980s \textcites{Popper1982}{Popper1982b}{Popper1985}{Popper1986}, Popper famously formulated a EPR-like thought experiment. Popper, at the height of his fame, was taken much more seriously than in the 1930s \citep{DelSanto2017}. However, his attempt must also ultimately be considered a \s{gross mistake} \citepp{Redhead1995}{Ghirardi2007}.

\todo{451}

%Ghirardi, G. C. (1988). Some Critical Considerations on the Present Epistemological and Scientific Debate on Quantum Mechanics. In G. Tarozzi and A. van der Merwe (eds), The Nature of Quantum Paradoxes: Italian Studies in the Foundations and Philosophy of Modern Physics. Dordrecht: Kluwer Academic Publishers, pp. 89-105.

\todo{redhead}



%

%Popper anticiate preparatio and measrem, was amognt htef  \q{the uncertainty principle restricts the degree of statistical homogeneity which it is possible to achieve in an ensemble of similarly prepared systems}, bit  past \p{from the data of both state preparation and measurement in the time interval between these two operations.} \citep[367]{Ballentine1970}. Howve,r before the perato  the coipl before the be avaeded. that o cannot before the preaptaiom. Teh couplin bewteen cannot be vaded
%
%%, $\Delta q_y \Delta p_y \approx 0$..... o that their relative position is fixed ($x_1 - x_2 = x_0$), while the total momentum is also fixed ($p_1 + p_2 = 0$
%
%%to obtain a predictive measurement that does not depend on a prior preparation, by exploiting correlations between two systems.
%
%%In both experiments, Popper failed to notice that the system he started with was already prepared — not in an individual eigenstate, but in an entangled state that defines its statistical properties.
%
%

\begin{figure}
\centering
 \includegraphics[scale=0.5, trim = 0mm 0mm 0mm 0mm, clip]{1982PopperExperiment}
\label{fig:1982popperexperiment}
\end{figure}


Popper famously imagined pairs of particles emitted in opposite directions from a  common $S$ source along the $x$-axis towards two slits $A$ and $B$, beyond which a semicircle of Geiger counters was placed (\cref{1982popperexperiment}). If one \emph{knows} the position $\Delta q_y \approx 0$ of particle $A$, one can also determine, by symmetry, the position $\Delta q_y \approx 0$ of particle $B$\footnote{If particle $A$ lands at $y = 1 \,\text{mm}$, then particle B will also be found at $y = -1 \,\text{mm}$}\todo{minus sign?}. Popper then proposed closing the slit for $A$, thereby causing diffraction, i.e. a very broad momentum distribution $\Delta p_y$ (all the counters on the right would be activated). Invoking the EPR reasoning, Popper argued that two alternatives were possible: (a) the Cophenhangen view: $B$’s momentum distribution must also spread, even though $B$ never encountered a slit (all the counters on the right would be activated). This would imply an action at a distance; (b) Popper’s view: $B$’s momentum does not spread, so that one could in principle predict both $B$’s position and momentum with arbitrary precision. This would amount to a violation of the \IR.
%
Once again, the (b) alternative favored by Popper is a \s{measurement} that does not require a previous \s{preparation}, thereby beating the \IR understood as \s{scatter relations}. However, Popper failed to recognize that in his setup the particle pair had been previously \emph{prepared} in a state with precisely correlated positions. Because of this, while the total momentum is sharp, their momenta are necessarily broadly spread from the outset.\footnote{It is hard not also to conclude that also in this case Popper committed \s{gross mistake}. Indeed, Popper's setup is nothing but a special case of the EPR paper example, where particles are in a state with perfectly equal positions: $\psi\left(x_A, x_B\right)=\delta\left(x_A-x_B\right)$. As in EPR, Fourier-transforming to the momentum representation gives $\tilde{\psi}\left(p_A, p_B\right)=\frac{1}{2 \pi \hbar} \int d x_A d x_B\, \delta(x_A - x_B)\, e^{-\frac{i}{\hbar}\left(p_A x_A+p_B x_B\right)} = \delta(p_A + p_B)$. The total momentum is sharp $\left(p_A+p_B=0\right)$, but each particle's momentum is completely undetermined from the outset}. Contrary to Popper's claim, orthodox quantum mechanics predicts a broad distribution of detection events on both sides of the apparatus, and thus the activation of all counters, \emph{even without closing the slit } \citep[sec.~11.3]{Ghirardi2007}. Narrowing the $A$'s position via a slit broadens $A$’s momentum distribution, but $B$’s momentum distribution is already broad, so it does not change (all the counters on the right would be activated, anyway). Only the joint correlations are modified, but this can only be established post-facto using classical signals\footnote{By observing the distribution of $B$ alone. You cannot tell if $A$ had one slit, two slits, or no slit just by looking locally. The difference appears only in the correlations: some positions of $B$ are more (or less) likely to coincide with certain positions of $A$, depending on whether there was one slit or a double slit at $A$. To extract this pattern, you must match $B$’s clicks with $A$’s clicks via classical communication}. This is exactly why entanglement does not allow faster-than-light influence.
%\end{comment}
%
%
%
%%(e.g., Redhead, Selleri, Sudbery, Kim & Shih) 
%
%
%%\delta\left(x_A-x_B\right) \propto \delta\left(p_A+p_B\right)$
%
%
%%https://arxiv.org/pdf/1507.02010



%\appendix
%\label{app:1}
%\section{Argument}

\vW\ points out that, in order to determine the momentum of the particle from 
its position on the screen, one must know the free path length precisely, which 
requires the use of narrow diaphragms of width $d$. Yet, if the diaphragms are 
too narrow, diffraction appears, introducing an unavoidable spread in the 
transverse coordinate $y$. Thus, the spread in $y$ at the screen cannot be 
smaller than the slit opening:
\[
\Delta y \geq d.
\]
Since the deflection $EF = y$ is the measure of the momentum, the slit also 
produces a spread in transverse momentum $\Delta p_y$ due to diffraction, which 
in turn leads to a spread in vertical displacement after traveling the distance 
$L$:
\[
\Delta y \geq \frac{\Delta p_y}{mv} \, L.
\]
Narrowing the slit $d$ increases the momentum spread $\Delta p_y$, according to 
the basic uncertainty condition
\[
d \cdot \Delta p_y \geq h.
\]
Multiplying the two inequalities and inserting this condition gives
\[
(\Delta y)^2 \geq \frac{h}{mv} \, L. \tag{1}
\]
This shows that restricting the beam’s transverse position inevitably produces a 
corresponding uncertainty that enforces the Heisenberg principle. The figure 
illustrates this: the more one attempts to collimate the beam tightly at $D$, 
the more it spreads out by the time it reaches the detector at $E\!-\!F$.  

---

After establishing that diffraction at the slit enforces a minimal transverse 
uncertainty, \vW\ extends the analysis to the longitudinal motion of the 
particle, i.e.\ to $p_x$. Because the vertical displacement $y$ depends on the 
velocity $v$, the uncertainty in $y$ translates into an uncertainty in $v$. In 
a uniform force field, the deflection is proportional to $1/v^2$, since the 
vertical displacement follows the free-fall law 
\[
y = \tfrac{1}{2} a t^2,
\]
with the flight time given by $t = L/v$. Applying logarithmic differentiation 
and taking absolute values gives
\[
\frac{\Delta y}{y} = \frac{2 \Delta v}{v}. \tag{3}
\]

From this it follows, on the one hand, that there is an uncertainty in the 
longitudinal momentum,
\[
\Delta p_x = m \, \Delta v,
\]
and, on the other hand, that the determination of the $x$--coordinate,
\[
x = v t,
\]
is also uncertain. If $v$ is uncertain, then $t = L/v$ is likewise uncertain, 
so the total uncertainty in $x$ must include two contributions:
\[
\Delta x = v \Delta t + t \Delta v 
          = \frac{L}{v} \Delta v + \frac{L}{v} \Delta v 
          = \frac{2L}{v} \, \Delta v.
\]
Hence,
\[
\Delta p_x \, \Delta x = (m \Delta v)\!\left(\frac{2 L}{v} \Delta v\right) 
= \frac{2 m L}{v} \, (\Delta v)^2.
\]

---

Finally, combining relation (3) with the transverse inequality (1) yields
\[
\Delta p_x \, \Delta x \;\geq\; \frac{L^2}{y^2} \, h.
\]

Thus the uncertainty in velocity propagates into both momentum and longitudinal 
position, and together with diffraction effects it ensures that the Heisenberg 
uncertainty principle is preserved in both transverse and longitudinal 
directions.

\citetrackerfalse

%\printshorthands
\printbibliography

\end{document}