\documentclass[12pt,submitted]{article}
\usepackage[font=scriptsize]{caption}
\usepackage{els}


\newcommand{\LdF}{\bt{Logik der Forschung}\xspace}
\newcommand{\Lo}{\bt{Logik}\xspace}
\newcommand{\KPA}[1]{KPA #1}
\newcommand{\TE}{thought experiment\xspace}
\newcommand{\IR}{indeterminacy relations\xspace}
\newcommand{\UR}{uncertainty relations\xspace}
\newcommand{\HR}{Heisenberg's relations\xspace}
\newcommand{\Weis}{Weisskopf\xspace}
\newcommand{\vW}{von Weizsäcker\xspace}
\newcommand{\VW}{Von Weizsäcker\xspace}
\newcommand{\CH}{coupling-hypothesis\xspace}
\newcommand{\NW}{\jt{Die Naturwissenschaften}\xspace}
\newcommand{\ERG}{\jt{Erg\"anzung}\xspace}

\newcommand{\KH}{\german{Kopplungshypothese}\xspace}
\newcommand{\GE}{\german{Gedankenexperiment}}
\newcommand{\LF}[2]{\citep[#1/#2]{Popper1935}}

\newcommand{\pos}{\ensuremath{x}\xspace}
\newcommand{\mom}{\ensuremath{p_x}\xspace}
\newcommand{\pam}{\ensuremath{x} and \ensuremath{p_x}\xspace}


\begin{document}

\title{The Past of an Electron: Young Popper's \GE\ Against the Indeterminacy Relations}
\maketitle

\begin{abstract}
Between 1934 and 1936, Popper wrote at least eight manuscripts on the interpretation of quantum mechanics, proposing a \german{Gedankenexperiment} based on electron--photon collision in which reconstructing the past trajectory of an electron would allow one, via momentum conservation, to predict both position and momentum of a photon with arbitrary precision, thus apparently \s{beating} the \IR. Popper corresponded about it with leading physicists such as Weisskopf, Heisenberg, von Weizsäcker, and Einstein, who pointed out the flaws in his reasoning. In retrospect, Popper described these efforts as a \q{gross mistake}. Yet \citet{Jammer1974} suggested that this mistake may have influenced the 1935 Einstein–Podolsky–Rosen argument, while \citet{Margenau1974} credited Popper with anticipating the later distinction between preparation and measurement. Drawing on unpublished material, this paper argues that both claims are doubtful. Popper’s thought experiment is more closely related to the 1931 Einstein–Tolman–Podolsky setup, though he reached the opposite conclusion precisely because he misunderstood the link between preparation and measurement. The same \s{mistake}, the paper concludes, also undermines the EPR-like \TE\ Popper proposed at the turn of the 1980s.
\end{abstract}

%The same misunderstanding also plagues Popper's post-war work on the topic \citepp{Howard2012}{Shields2012}{Freire2014}{DelSanto2017}{DelSanto2019}. 

\intro

Between 1934 and 1936, Karl R.~Popper composed at least eight manuscripts revolving around a \german{Gedankenexperiment} intended to challenge the widespread interpretation of the so-called indeterminacy or uncertainty relations, which, in quantum mechanics, were thought to set a fundamental limit to the joint \emph{measurement} of conjugate quantities such as position and momentum, or time and energy:

\begin{enumerate}[itemsep=0pt, parsep=0pt, topsep=4pt, partopsep=0pt]
  \item \textit{{Zur Kritik der Ungenauigkeitsrelationen}} --- August 1934 \citep{Popper1934}
  \item \textit{Ergänzung zu der vorstehenden kurzen Mitteilung} --- before November 1934 \citep{Popper1934a}
  \item \textit{{Bemerkungen zur Quantenmechanik}} (chapter~2.VII of \cite{Popper1935}) and Appendix~V,~VI,~VII --- Fall 1934
  \item \textit{Zur Kritik der Ungenauigkeitsrelationen, 2.~Mitteilung [A]} --- December 1934 \citep{Popper1934b}
  \item \textit{Zur Kritik der Ungenauigkeitsrelationen, 2.~Mitteilung [B]} --- December 1934 \citep{Popper1934c}
\item \textit{{Erwiderung auf die Kritik Heisenberg--Weizsäcker}} --- late 1934 to early 1935
  \item \textit{{Meßanordnung}} --- February 1935
  \item \textit{Zur Kritik der Ungenauigkeitsrelationen [1935]} --- August 1935 \citep{Popper1935b}
 \item \textit{Bemerkung zum Komplementaritätsproblem der Quantenmechanik} --- 1936 \citep{Popper1936a}
\end{enumerate}
%
Only 1 and 3 were published during Popper's lifetime. Numbers 6 and 7 have been lost, although 6 may in fact be identical with 5. The remaining material was published only recently in a collection of Popper’s early writings \citep{Popper2006}. While drafting these texts, the young Popper---at that time a schoolteacher with only a single short publication to his name (\citeyear{Popper1932/1933})---engaged in extensive correspondence with leading  physicists of his time: \begin{inparaenum}[(1)]
\item Victor Weisskopf, then assistant to Pauli in Zürich; \item Heisenberg and his assistants Carl Friedrich von Weizsäcker and Hans Euler; and \item Einstein. \end{inparaenum} Some of this correspondence is preserved at the KPA and remains unpublished. Moreover, Popper discussed his ideas in person with Viennese physicists he could reach: Franz Urbach, his cousin, the crystallographer Käthe Schiff, Hans Thirring\footnote{According to Popper's recollection, in 1934 (possibly 1935) Thirring invited him to his seminar; see \citet[1125]{Schilpp1974}} and his assistant Eugene Guth, as well as Felix Ehrenhaft. Naturally, \latin{verba volant}: we have little insight into the concrete content of these exchanges.

\todo{apologetic critical}
The late 1920s and early 1930s saw a proliferation of \german{Gedankenexperimente} as tools to probe conceptual issues \emph{outside} the quantum-mechanical formalism---either to demonstrate its consistency or to refute it. \citet{Heisenberg1927} formulated his famous $\gamma$-ray microscope \TE\ to show that the \UR\ reflect the mutual disturbance that arises from the experimental determination of position and momentum. By contrast, Einstein was regarded as having devised \german{Gedankenexperimente}, such as the light-quantum box \TE, to challenge the \UR\ \citep[127\psq\hide{**}]{Bohr1931a}. Popper, for his part, accepted the mathematical \emph{derivation} of the \UR\ from the quantum formalism; he, rather, devised a \GE\ intended to expose the tension between two different \emph{interpretations} of the \UR. Popper aimed to show that the \IR\ impose a limitation on the \emph{physical selection} of an ensemble of systems in which both position and momentum are sharply defined; however, \emph{pace} Heisenberg, they do not impose any limitation on the accuracy of \emph{measurements} of position and momentum on a single system.

%Popper tried to exploit a perfect correlation implied by conservation law to 

Popper sought to exploit the fact that quantum physicists, following Heisenberg himself, generally conceded that \emph{non-predictive measurements} of arbitrary sharpness are permitted by the \UR; nevertheless, they typically denied that \emph{predictive measurements} were. According to Popper, the reason is that they tacitly assumed what he called the \emph{\CH} \origg{\KH}\footnote{\citet[238\psq]{Popper1959} later adopted the translation \s{hypothesis  of linkage,} possibly because \s{coupling} in English typically implies a physical interaction. However, I prefer to retain a rendering closer to the original German, which better reflects its resonance with contemporary physical discourse.} between \s{physical selections} and \s{predictive measurements}. Popper, however, thought he could show that the \CH\ was inconsistent with the formalism of \qm. He devised a \TE\ based on a electron--light quantum collision. On the basis of a \emph{non-predictive measurement} that reconstructed the past trajectory of an electron by invoking momentum conservation, the setup was designed to permit arbitrarily precise \emph{predictive measurements} of the future trajectory of a light quantum after the collision, thereby avoiding any selection that would violate the statistical interpretation of the \UR.

%make single-particle predictions 



In retrospect, Popper himself would describe his \TE\ as \q{a gross mistake}, for which he had been \q{deeply sorry and ashamed} \citep[15\hide]{Popper1982}. Yet early commentators, writing at the height of Popper’s fame, attributed some significance to his early engagement with \qm.  Max~\citet[178\psq]{Jammer1974} conjectured that Popper's mistaken thought experiment was at least instrumental in inspiring the 1935 \citeyear{Einstein1935} Einstein--Podolsky--Rosen (EPR) thought experiment. Indeed, according to Jammer, Popper’s setup bore a \q{striking resemblance} to the EPR setup, raising the question of a possible influence from the work of Einstein and his collaborators \citep[178]{Jammer1974}. Henry~\citet[757\psq]{Margenau1974} credited Popper with anticipating the distinction between \s{preparation} (which Popper called \s{physical selections}) and \s{measurement}, a distinction he himself proposed around the same time \citepp{Margenau1936}{Margenau1937} and which later became standard. In particular, according to Margenau, Popper was among the first philosophers to appreciate that the \IR\ expressed a \s{preparation uncertainty} rather than a \s{measurement uncertainty} \citep[see][72\psq]{Popper1974}.

This paper, based on little-known or unpublished material from the Karl Popper Archives (KPA), aims to demonstrate that both claims fail to withstand critical scrutiny. Popper was wary of implying that the \q{gross mistake made by a nobody (like myself)} might have influenced Einstein, noting only that \q{from a purely temporal standpoint} the possibility could not be ruled out (\letter{Popper}{Jammer}{13}{4}{1967}, \cite[178\fn{30}]{Jammer1974}). However, it is precisely the chronology that seems to exclude any influence of Popper’s thought experiment on the emergence of the 1935 EPR argument. What can be said is that Popper independently formulated a variant of the lesser-known \citeyear{Einstein1931-03} Einstein--Tolman--Podolsky (ETP) thought experiment, though he arrived at the opposite conclusion. The paper argues that the reason was ultimately that Popper deeply misunderstood the relation between \s{preparation} and \s{measurement} that he allegedly anticipated \citepp[see][sec.~6]{Maxwell2016}. Popper's notion of \s{physical selection} is akin to, but still not on par with, the modern notion of \s{preparation}.  In particular, the \emph{coupling hypothesis} between preparations and measurements is not an \s{additional hypothesis}, ultimately contradictory with the rest of the formalism, as he argued, but a central feature of \qm\ that distinguishes it from classical physics.

Although \s{Young Popper} has long been a well-established \s{scholarly field} \citep{Hacohen2004}, this episode has received little attention. The present paper seeks to redress this lacuna in the literature. Against the background of the 1930s debate on the past trajectories of particles in \qm\ (\cref{sec:past}), it reconstructs the development of Popper's \TE\ (\cref{sec:popperquantun,sec:interpreation,sec:TE}) and examines the criticisms raised by Heisenberg and his Leipzig group (\cref{sec:leipizg}), as well as by Einstein (\cref{sec:einstein}). Ultimately, though only after considerable resistance, Popper conceded that his \TE\ was a \s{mistake}. The failure of the \TE\ should have convinced him of the validity of the \CH\ between preparations and measurements---the very principle he had aimed to challenge. Instead, Popper concluded that the flaw lay in the specific ETP-like setup. Accordingly, \textcites{Popper1982} sought to replace the ETP-like experiment with an EPR-like one. For this reason, recent historical literature \citepp{Howard2012}{Shields2012}[sec.~6]{Maxwell2016}{DelSanto2018}{DelSanto2019} tends to dismiss Popper's early attempt as a youthful blunder, focusing instead on the later setup \citepp{Popper1967,Popper1982}. Yet this paper argues that the latter was ultimately grounded in the same \s{mistake}: appealing to a perfect correlation---this time, of relative position---to circumvent the preparation-measurement coupling.

%{Popper1982b}{Popper1985}{Popper1986}

%to know $B$’s properties without physical preparation.
%	•	1981: His aim became statistical --- to alter $B$’s observed distribution without altering its preparation.


%Use correlations to gain predictive measurement without proper preparation, use correlations to change predictive measurement statistics without changing preparation


\section{Setting the Stage: The Past Trajectory Problem}
\label{sec:past}

\subsection{The Heisenberg-Schlick Dispute}
\label{sub:heisenbergschlick}

In the spring of 1929, Heisenberg delivered a series of lectures on quantum mechanics at the University of Chicago. A written account of the lectures was completed in March 1930 and appeared later that summer in print, in both English and German editions \citepp{Heisenberg1930}{Heisenberg1930a}. In presenting the \IR, Heisenberg takes the opportunity to address, although rather in passing, the vexed question of the retrodictability of sharp joint values of position and momentum variables of, say, an electron. As already pointed out in \posscite[583]{Bohr1928} Como lecture, if one performs two position measurements at successive times, then from the time of flight one can, in principle, reconstruct the electron’s momentum \emph{between} these two position measurements with arbitrary precision\footnoteh{The fact that the reconstruction \s{before} the first measurement is not allowed is emphasized in \letter{Ehrenfest}{Goudsmit, Uhlenbeck, and Dieke}{3}{11}{1927}}%
%
\footnoteh{In a famous 1927 letter to his students, Ehrenfest refers to this question: \q{The uncertainty principle does not forbid calculating momentum from past data. It limits the precision with which you can simultaneously measure position and momentum in a single act of observation. Furthermore, one also notices that these position measurements at 3 o'clock do NOT allow an exact evaluation of the momentum BEFORE 0 o'clock and AFTER 3 o'clock, but only within the uncertainty associated with Compton recoil} \letterp{Ehrenfest}{Goudsmit, Uhlenbeck, and Dieke}{3}{11}{1927}}. %
%
Heisenberg discussed a similar \TE in his book, but replaced the first position measurement with a momentum measurement:

\qt{[I]t should be noted that the uncertainty relations apparently do not apply to the past. For if the electron’s \emph{velocity} is initially known and then its \emph{position} is measured precisely, one can also calculate the electron’s positions for the \emph{time before the position measurement} with precision; for this past, $\Delta p \Delta q$ is then smaller than the usual limit. Yet this \emph{knowledge of the past} is purely speculative, since (because of the \emph{change of momentum during the position measurement}) it in no way enters as an initial condition into any calculation concerning the electron’s future and does not appear in any physical experiment at all. Whether one should ascribe any physical reality to the aforementioned calculation about the electron’s past is therefore purely a \emph{matter of taste} \origins{Geschmacksfrage}}{Nehmen wir z. B . an, daß die Geschwindigkeit des Elektrons genau bekannt sei, der Ort dagegen völlig unbekannt. Dann muß jede folgende Beobachtung des Ortes das Impulsmoment des Elektrons
an ern; und zwar muß diese Änderung u m einen derartigen Betrag unbestimmt sein, daß nach Durchführung des Experiments unsere Kenntnis der Elektronenbewegung durch die Ungenauigkeitsrelationen beschränkt ist. Dies soll im folgenden a n einigen Experimenten als Beispielen nachgewiesen werden. Vorher sci jedoch bemerkt, daß die Unbestimmtheitsrelationen sich offenbar nicht auf die Vergangenheit beziehen. Denn wenn zunächst die Elektronengeschwindigkeit bekannt ist, dann der Ort genau gemessen wird, so lassen sich auch für die Zeit vor der Ortsmessung die Elektronenorte genau ausrechnen; für diese Vergangenheit ist 4q 4p dann kleiner als der übliche Grenzwert. Diese Kenntnis der Vergangenheit hat jedoch rein spekulativen
Charakter, denn sie geht (wegen der Impulsänderung bei der Orts-
messung) keineswegs als Anfangsbedingung i n irgendeine Rechnung
über die Zukunft des Elektrons ein und tritt überhaupt in keinem
physikalischen Experiment in Erscheinung. Ob man der genannten
Rechnung über die Vergangenheit des Elektrons irgendeine physi-
kalische Realität zuordnen soll, ist also eine reine Geschmacksfrage.}[\citep[15]{Heisenberg1930}]
%  
Heisenberg likely had in mind the measuring procedure he describes a few pages later \citep[19\psq]{Heisenberg1930}. Electrons move along the $+x$ direction with a known longitudinal momentum $p_x$, while light quanta enter along the $-x$ axis (\cref{fig:heisenberg-doppler}). The Doppler shift $\delta\nu$ of a scattered light quantum at time $t_0$, assuming energy and momentum conservation, allows one to infer the electron’s transverse momentum $\delta p_y$; the point where the electron later arrives on the detection screen at time $t_1$ determines its position $\delta y$\footnote{In the following, we use $\Delta q$ and $\Delta p$ to denote the dispersions of the statistical distributions, and $\delta q$ and $\delta p$ to denote the measurement errors}. This seems to permit reconstruction of the electron’s \s{past path} between $t_0$ and $t_1$ with arbitrary precision, $\delta y\, \delta p_y \approx 0$. Heisenberg’s reference to the \s{time before the position measurement} is somewhat ambiguous, as it seems to imply the possibility of tracing the electron’s path even \emph{before} the first momentum measurement at $t_0$. Popper indeed relied on this interpretation of the passage, as discussed below (\cref{sub:weizobjection}).

%Heisenberg’s choice to use a momentum followed by a position measurement, as we shall see, will prove to have far-reaching implications. 

\begin{figure}
\centering
\includegraphics[scale=0.1]{1930heisenbergdoppler.png}
\caption{Adapted from \cite[19]{Heisenberg1930}.}
\label{fig:heisenberg-doppler}
\end{figure}
%
Heisenberg immediately pointed out that the reconstruction of the electron’s past has no predictive significance for its future behavior, and therefore does not constitute a violation of the \IR. However, the question of whether such reconstructions possess any \s{physical meaning}---which Heisenberg dismissed as a \s{matter of taste}---constitutes, as one might expect, an interesting philosophical puzzle. In fact, the issue was immediately taken up by Moritz Schlick, who was at that time working on a manuscript on the problem of causality. At the end of November, Arnold Berliner (the editor of \NW) confirmed receipt of \art{Die Kausalität in der gegenwärtigen Physik}, which soon circulated among physicists, prompting correspondence with Einstein, Heisenberg, and Pauli. The paper identifies causality with predictability; thus, Schlick accepts that Heisenberg’s relation concerns an indeterminacy of prediction, thereby limiting the applicability of causality on the microscopic scale. Schlick was aware that such indeterminacy does not apply to retrodictions, as \citet[15]{Heisenberg1930} himself points out. Nevertheless, Schlick remained unconvinced by Heisenberg’s agnostic stance toward the physical reality of past trajectories:

\qt{Heisenberg thinks that \s{whether any physical reality should be assigned to the calculation concerning the past of the electron is purely a question of taste} [\cite[15]{Heisenberg1930}]. But I should prefer to put it more strongly, in complete agreement with what I take to be the incontestable basic viewpoint of Bohr and Heisenberg themselves. If a statement about an electron’s position is not \emph{verifiable} within atomic dimensions, \emph{we can attach no meaning to it}; it becomes impossible to speak of the ‘path’ of a particle between two points at which it has been observed. (This is not, of course, true of bodies of molar dimensions, for it is in principle possible to verify afterwards that the projectile was located at the intervening points)}{Es ist vielleicht tröstlich zu bemerken, daß wir in ganz demselben Sinne (und einen anderen Sinn des Wortes „unbestimmt“ vermag ich nicht auszudenken) auch von der Vergangenheit sagen müssen, daß sie in gewisser Hinsicht indeterminiert sei. Nehmen wir z. B. an, daß die Geschwindigkeit eines Elektrons genau gemessen und hierauf sein Ort genau beobachtet wurde, so gestatten zwar die Gleichungen der Quantentheorie, auch frühere Orte des Elektrons genau zu berechnen (dies hebt auch HEISENBERG hervor: \emph{Die physikalischen Prinzipien der Quantentheorie}, 1930, S. 15), aber in Wahrheit ist diese Ortsangabe physikalisch sinnlos, denn ihre Richtigkeit ist prinzipiell nicht prüfbar, da es ja prinzipiell unmöglich ist, nachträglich zu verifizieren, ob das Elektron sich zur angegebenen Zeit am berechneten Orte befunden hat. Hätte man es aber an diesem Orte beobachtet, so würde es gewiß nicht diejenigen Orte erreicht haben, an denen es später aufgefunden wurde, da ja bekanntlich seine Bahn durch die Beobachtung in unberechenbarer Weise gestört wird. HEISENBERG meint (a. a. O., S. 15): „Ob man der Rechnung über die Vergangenheit des Elektrons irgendeine physikalische Realität zuordnen soll, ist eine reine Geschmacksfrage.“ Ich würde mich aber lieber noch stärker ausdrücken, in vollkommener Übereinstimmung mit der, wie ich glaube, unanfechtbaren Grundanschauung BOHRS und HEISENBERGS selbst. Ist eine Aussage über einen Elektronenort in atomaren Dimensionen nicht verifizierbar, so können wir ihr auch keinen Sinn zuschreiben; es wird unmöglich, von der „Bahn“ eines Teilchens zwischen zwei Punkten zu sprechen, an denen es beobachtet wurde (von Körpern molarer Dimensionen gilt das natürlich nicht).}[\citep[159]{Schlick1931}]
%
Schlick argues that although quantum theory’s equations allow one to calculate an electron’s past positions after exactly measuring its velocity and then its position, such results are physically meaningless, since they are \s{unverifiable} in principle: one cannot verify whether the electron was actually moving with that velocity at the calculated point, since any observation would disturb its path in an uncontrollable way. For Schlick, therefore, the very idea of a sharply defined past path of the electron lacked physical meaning, just like the notion of its future path.

\subsection{The ETP Thought Experiment}
\label{sub:ETP}

It is not surprising that, in his comments on Schlick’s article in late 1930, Einstein described Schlick’s stance as \s{too positivistic} \letteraeap{Einstein}{Schlick}{28}{11}{1930}[21-603]. Concluding his letter, Einstein mentioned that he would spend the coming winter in Pasadena. Just a fortnight after \posscite{Schlick1931} paper appeared on \datef{13}{2}{1931}, Einstein, in collaboration with Richard Tolman and Boris Podolsky (ETP), submitted to the \emph{Physical Review} a short letter reporting the results of his American visit. The paper was published on \datedm{15}{3}{1931} \citep{Einstein1931-03}. Neither Heisenberg nor Schlick are mentioned; however, Einstein and his collaborators address the same question of the past path of particles in \qm. They argue that \emph{if} sharp retrodictions about one particle were possible, \emph{then} one could, with the help of conservation laws, infer predictions about the future path of a second particle that would otherwise be excluded by the uncertainty relations. Thus, ETP conclude that retrodictions should \emph{not} be allowed. The \IR apply equally to the past and the future.

The proposed arrangement (\cref{fig:ETP-setup}) resembles the one Einstein reportedly used in the 1930 debate with \citet[127\psq\hide{**}]{Bohr1931a} in Brussels.\footcite[1.4.3]{BacciagaluppiCrull2024} A box $B$ contains identical particles. At time $t_0$, a shutter $S$ releases two particles, $a$ and $b$. By weighing the box before and after the emission, the observer can determine the total energy loss, $E_{\text{tot}} = E_a + E_b$. At $O$, particle $a$ may undergo different measurements. Its momentum $p_a$ (and energy $E_a$) can be determined, for example, by a Doppler-effect measurement at $O_1$, while its arrival position and time $t_1$ can be recorded at $O_2$ by some mechanical time-recording counter. From the measured position of $a$ at $t_1$, together with its past momentum $p_a$, the shutter opening time $t_0$ can, in principle, be reconstructed. If $E_a$ has been measured, then the energy of particle $b$ can be inferred from $E_b = E_{\text{tot}} - E_a$. Once $t_0$ has been determined, the arrival time $t_b$ at $O_2$ can be inferred from the geometrical arrangement of the setup, that is, from $SRO$.


\begin{figure}
\centering
\includegraphics[scale=0.15]{1931ETPmod}
\caption{Adapted from \cite{Einstein1931-03}.}
\label{fig:ETP-setup}
\end{figure}

At this stage, a backdoor appears to open, seemingly offering a way around the energy--time uncertainty relation $\delta E, \delta t \leq \frac{h}{4\pi}.$ Yet ETP quickly close it again \citep[781]{Einstein1931-03}. One might attempt to measure momentum via the Doppler shift of low-energy light quanta, that is, long-wavelength radiation (e.g., in the infrared range). Assuming conservation of momentum and energy, this method, in principle, allows one to infer the momentum of a particle before and after scattering from the frequency shift of the scattered photon. However, to determine the frequency with high precision, many oscillation periods must be observed, which blurs the exact time of the scattering event. As a result, the precise moment at which the shutter $S$ was opened cannot be reconstructed. Conversely, if one tries to determine the scattering time very precisely, one must use high-frequency (short-wavelength) radiation, thereby losing information about the momentum before the collision. The authors summarized their conclusion as follows:

\qt{Thus in our example, although the velocity of the first particle could be determined both before and after interaction with the infrared light, it would not be possible to determine the exact position along the path $SO$ at which the change in velocity occurred as would be necessary to obtain the exact time at which the shutter was open. \\ It is hence to be concluded that the principles of the quantum mechanics must involve an uncertainty in the description of \emph{past} events which is \emph{analogous} to the uncertainty in the prediction of \emph{future} events. It is also to be noted that although it is possible to measure \emph{the momentum of a particle and follow this with a measurement of position}, this will not give sufficient information \emph{for a complete reconstruction of its past path}, since it has been shown that there can be \emph{no method for measuring the momentum of a particle without changing its value}}{}[\citep[781\me]{Einstein1931-03}]
%
Thus, ETP point out the flaw in this kind of experiment: one can reconstruct the path of a particle \emph{between} the momentum and the position measurements, but not \emph{before} the momentum measurement at $O_1$. The uncertainty principle therefore not only restricts predictions about the future but also imposes limitations on the reconstruction of the past: the path reconstruction before the momentum measurement must be physically forbidden \emph{if} the \IR\ are to hold.

%Finally, it is of special interest to emphasize the remarkable conclusion that the principles of quantum mechanics would actually impose limitations on the localization in time of a macroscopic phenomenon such as the opening and closing of a shutter.

\todo{a \pemo against the \CH hypothesis}

As Ehrenfest wrote to Bohr in the summer of 1931, Einstein, contrary to common perception, was clearly not attempting to devise \q{new \latin{perpetua mobilia} = machines against the uncertainty relation} (\letter{Ehrenfest}{Bohr}{9}{7}{1931}, AHQP/BSC 18)\footnote{This brief overview is based on \cite[Ch.~1]{BacciagaluppiCrull2024}} Rather, Einstein’s point was that, \emph{because of} the \UR, the experimental \emph{choice} between determining the time or the momentum of $a$ appeares to retroactively determine whether the emitted light quantum $b$ possesses a definite energy or a definite emission time \emph{after} it had already left the box \citep[see also][]{Einstein1932d}. After a talk in Leiden in November 1931, Einstein might have been stimulated by Ehrenfest to replace the energy--time with the position--momentum uncertainty \letteraeap{Einstein}{Ehrenfest}{5}{4}{1932}[10-231]. Indeed, as later recalled by Léon \textcites[127\psq]{Rosenfeld1967}, Einstein, after a seminar in Brussels in the spring of 1933, apparently reformulated the argument without the aid of the box, considering two particles emitted simultaneously: a measurement of the momentum (or position) of the first particle allowed an inference of the corresponding momentum (or position) of the second. If Rosenfeld's testimony is accurate, Einstein possessed, in embryonic form, the gist of the EPR paper before permanently moving to the United States on \datef{17}{10}{1933}.


\section{Popper’s Turn to Quantum Theory: The Correspondence with Weisskopf}
\label{sec:popperquantun}

At the turn of 1934, Popper was reworking the manuscript of \bt{Die Beiden Grundprobleme der Erkenntnistheorie} \citep{Popper1930-1933} into what would become \emph{Logik der Forschung} \citep{Popper1935}. In the spring, Popper submitted a first version of the manuscript to Schlick, the editor of the series. At the request of the publisher Springer, the manuscript had to be shortened considerably \citep[235--244]{Hacohen2010-06-11}. While Popper was adding two technical chapters on probability and one on \qm, much of the editing was carried out with the help of his uncle, Walter Schiff. Between May and June 1934, Popper likely began to devote special attention to the quantum physics chapter, discussing it with Viennese physicists like Urbach, Urbach's friend Weisskopf, and his cousin Käthe \citep[237]{Hacohen2010-06-11}. At that time, Weisskopf was in Zurich as Pauli's assistant, so their exchange had to take place by mail, leaving us written evidence of Popper’s thinking during that period. Although Popper’s original letter has not survived, much of his position can be inferred from Weisskopf’s reply.

Weisskopf seems to concede to Popper that the uncertainty relations do not exclude the possibility that the position and momentum of a single electron could, in principle, be \s{known} (\ie, measured) with arbitrary accuracy. If the position is fixed by a very narrow slit, the emerging beam no longer consists of electrons with a well-defined momentum or energy. The diffraction pattern on the screen reveals this momentum (or angular) distribution and the associated energy spread. When the electrons strike a target, some atoms are excited even though their excitation requires more energy than the mean kinetic energy of the electrons. From these excitations one may infer both position and energy, and thus momentum of single electrons. As Weisskopf put it, \qt{[t]his consideration is of course not changed if, by some measurements, I determine which energies are present in this accumulation of electrons \emph{without altering their spatial arrangement}. In this, I agree with you and admit my mistake}{Diese Überlegung wird natürlich nicht geändert, wenn ich durch irgendwelche Messungen feststelle, welche Energien in dieser Elektronenanhäufung vorhanden sind, ohne deren räumliche Anordnung zu verändern. Darin gebe ich Ihnen Recht und bekenne meinen Fehler} \letterKPAp{Weisskopf}{Popper}{26}{6}{1934}[360-21].

However, Weisskopf immediately raises a counter-objection: \qt{But now comes the point: if I can measure these energies without disturbing the spatial arrangement, then I have devices with which I can also \emph{produce} \origins{herstellen} beams (aggregations of electrons) that are 1.) spatially confined and 2.) possess a sharp energy (= momentum of free particles)}{Aber nun kommt der Punkt: Wenn ich diese Energien ohne Störung der räumlichen Anordnung messen kann, so habe ich Apparate, mit denen ich auch Strahlen (Elektronenanhäufungen) \emph{herstellen} kann, die 1.) räumlich begrenzt, 2.) eine scharfe Energie (= Impuls bei freien Teilchen) haben} \letterKPAp{Weisskopf}{Popper}{26}{6}{1934}[361-21]. The possibility of producing such aggregates of electrons would indeed violate the \IR, since a wave localized in space must contain infinitely many frequencies, and thus momenta and energies. Popper probably replied that instruments allowing \emph{measurements} of sharp electron energy and position might exist without thereby enabling the \emph{production} of an \emph{aggregation} of such electron beams. However, Weisskopf doubted this, arguing that if we can \emph{measure} electron energy, we should also be able to \emph{separate} electrons according to predetermined values; that is, to produce an aggregate of electrons all with the same energy and the same position:

\qt{(1) Apparatuses that can \emph{measure} both position and energy at the same time are not necessarily able to \emph{produce} electron aggregates with sharp position and energy. I consider this conclusion not entirely correct, since -- in the case that one has the possibility of determining the energy of electrons at sharp position -- it must surely also be technically possible somehow to remove them, for example, if they have too much energy.  \\ (2) You can probably anticipate my line of reasoning, which leads you to say that quantum mechanics is only capable of dealing with that world in which such \emph{apparatuses} are not applied. But I would already regard that as a breakdown. After all, apparatuses are also part of nature and should therefore be valid within it. \\ There would, however, have to exist accumulations of electrons whose position and energy are sharply determined, even though this is impossible according to wave theory. Thus the statement that electrons can in fact always be represented by waves would not always be correct, and in this lies perhaps the end of quantum mechanics}{\begin{enumerate} \item Apparate, die Ort und Energie zugleich messen k\"onnen, m\"ussen deswegen noch nicht im Stande sein, Elektronen- anh\"aufungen mit scharfem Ort und Energie herzustellen. Diesen Schluss halte ich f\"ur nicht ganz richtig, da es -- im Falle, da{\ss} man \"uberhaupt die M\"oglichkeit hat, die Elektronen bei scharfem Ort bzgl. ihrer Energie zu erkennen, -- sicher auch m\"oglich rein technisch m\"oglich
sein muss, sie irgendwie z.B. wegzuschaffen, wenn sie eine zu gro{\ss}e Energie haben. \item Sie sehen meinen Schlussweise wohl kommen, der Sie sagen, da{\ss} die Quantenmechanik eben nur im Stande ist jene Welt zu behandeln, in der solche Apparate nicht angewendet werden. Das w\"urde ich aber schon als einen Zusammenbruch bezeichnen. Die Apparate sind ja auch ein St\"uck Natur und sollten in solcher gelten -- so m\"ussen \"ahnliche Erscheinungen auch irgendwo sp\"urbar auftreten. \end{enumerate} Es m\"ussten aber Elektronenanh\"aufungen existieren, deren Ort und Energie scharf bestimmt ist, ob es nach der Wellentheorie unm\"oglich ist. Somit w\"are es, sog. Elektronen lassen sich tats\"achlich durch Wellen darstellen nicht immer richtig und darin ist wohl das Ende der Qu.-M.}[\letterKPAp{Weisskopf}{Popper}{26}{6}{1934}[360.21]]
%
Here the fundamental issues at stake are laid out quite clearly: (1) the distinction between (1a) \emph{producing} an aggregate of electrons with arbitrarily sharp position and momentum, and (1b) \emph{measuring} the position and momentum of a single electron with arbitrary precision; (2) the \s{coupling} between the two operations, that is, the claim that the possibility of (1b) necessarily implies that of (1a), which in turn would certainly constitute a violation of the \IR. 

%Popper also asked \Weis about the use of the term \emph{pure case} to denote an ensemble of particles in which either position \emph{or} momentum is sharply defined. In the course of these exchanges, he appears to have coined the expression \s{\emph{super-pure case}} to refer to an ensemble in which both position \emph{and} momentum would be sharp simultaneously.

Weisskopf asked Popper to share his opinion on these ideas. However, regarding Popper's request to cite their discussions in his forthcoming book, he asked not to be mentioned by name. In the following weeks, Popper may have come up with a clever trick to counter Weisskopf’s objection: a \GE\ in which an apparatus was laid out that could \emph{measure} with arbitrary precision the position and momentum of a single particle without allowing the \emph{production} of an ensamble more homogeneous than the \IR would permit. Already, on \datef{27}{8}{1934}, Popper submitted a short note, \art{Zur Kritik der Ungenauigkeitsrelationen}, to \jt{Die Naturwissenschaften}, one of the most prestigious scientific journals of that time. As we shall see, Popper independently envisaged the same backdoor to the \IR as the one ETP had already suggested, only using momentum and position instead of energy and time. 

Popper seems to have soon realized that this backdoor could indeed be closed in the same way suggested by ETP. However, contrary to Einstein, Popper wanted to leave the backdoor open. Thus, he wrote a short supplement, \bt{Ergänzung zu der vorstehenden kurzen Mitteilung}, to clarify the matter. By the autumn of 1934, Popper had completed \emph{Logik der Forschung}. Thus, by that time, chapter IX, titled \emph{Bemerkungen zur Quantenmechanik}, together with the three appendices V, VI, and VII devoted to the same topic, can be regarded as complete. The manuscript of the \ERG was probably written after the paper had already been submitted to \jt{Die Naturwissenschaften}, but before it was published\footnote{No reference is made to \vW's response that was published alongside \citets{Popper1934} article; see below \cref{sub:weizobjection}}. It must also have been written prior to Appendices VI and VII, which appear to be Popper's attempts to address problems raised in the \bt{Ergänzung}.

\section{Popper's Interpretation of the Indeterminacy Relations}
\label{sec:interpreation}

Popper starts from the well-known fact that, within the formalism of quantum mechanics, a central role is played by the so-called Heisenberg \IR \origg{Unbestimmtheitsrelation}, that is, the inequality:

\begin{equation}
\label{eq:ur}
\Delta p_x \Delta x > h / 4 \pi
\end{equation}
%
where \pos is the space coordinate of a point mass and the \mom the momentum component along the same direction. Popper’s aim was not to challenge the mathematical \emph{derivation} of Heisenberg's formula, but rather its physical \emph{interpretation}. The \IR\ can be interpreted (a) as \s{imprecision relations,} \origg{Ungenauigkeitsrelationen}---as Heisenberg did---which set a limit on the accuracy of simultaneous measurements of $\pos$ and $\mom$ for a \emph{single} particle; or (b) as \s{dispersion relations,} \origg{Streuungsrelationen}, expressing the limit on the standard deviations of repeated \pos\ and \mom\ measurements on an \emph{ensemble} of particles. It is undeniable that Popper was among the first to make this distinction, at least in the philosophical literature\footnote{In the physics literature this distinction was at least implied in \citet{Kennard1927} and \citet[67\psq]{Weyl1928}. Popper was familiar with the latter}, although similar views were defended around the same time by other Viennese scholars, such as Richard von~\textcites{Mises1930}{Mises1934}, whose work Popper might well have known. However, Popper possibly arrived at a similar conclusion independently, starting from von~\citets{Mises1928} frequency interpretation of probability, which he defends in Ch.~VI.


%https://arxiv.org/pdf/1904.06139

\subsection{Physical Selections and Measurements}

According to Popper, Heisenberg’s \emph{philosophical justification} for \cref{eq:ur} was based on the idea that measurement itself disturbs physical systems: precision in one variable (e.g., position) can be achieved only at the expense of precision in its conjugate (e.g., momentum), with \cref{eq:ur} establishing the bound. Against this view, Popper argued that the \emph{mathematical derivation} of \cref{eq:ur} follows from Born’s interpretation of the $\psi$-function as a probability amplitude, with $|\psi(x)|^2 dx$ giving the probability of finding a particle along the $x$-axis within the infinitesimal interval between $x$ and $x + dx$. Using Fourier analysis, one can show that the position and momentum probability distributions are mathematically related in such a way that a narrower distribution in position entails a broader distribution in momentum, and vice versa. In this reading, \cref{eq:ur} regulates the dispersion relation between these distributions rather than limits the precision of measurement. 

%Physical selections are controllable: the resulting sub-ensemble can be reproduced at will by repeating the same filtering procedure. A conditional selection, by contrast, cannot be reproduced, since it depends on a previous physical selection and on specific measurement outcomes that cannot be controlled in advance



For Popper, the confusion between these two \emph{interpretations} is ultimately the consequence of physicists not clearly distinguishing between \begin{inparaenum}[(1)] \item a \emph{physical\footnote{As opposed to merely \s{mental} or \s{imagined} selection \LF{163}{226}. These correspond to conditional selections, which define a sub-ensemble after the fact, by conditioning on the value of a random variable obtained in a measurement (say, an ensemble of particles $B$ that have the same momentum as $A$)} selection} \origg{Aussonderung}, which occurs when, for example, a beam of particles is actively filtered so that only those passing through a narrow slit (a region $\Delta x$) remain, thereby selecting an \s{aggregate} of particles all with the same position, a so-called \s{pure case}\footnote{An aggregate \origg{Menge} corresponds to what von~Mises called a \german{Kollektiv}, in modern terms an \s{ensemble}. As \citet[225\fn{$^\ast$1}]{Popper1959} later recognized, at that time he did not clearly distinguish between an aggregate of co-present particles, such as a beam (spatial ensemble), and an aggregate of repetitions of an experiment performed with one particle or one system of particles (temporal ensemble)} \item a \emph{measurement} \origg{Messung} involves the detection of a particular system by means of a measuring instrument. Unlike separation, measurement produces an actual registration of a value---such as a spot on a screen or a pointer deflection---an irreversible change in a piece of macroscopic apparatus that constitutes a publicly observable record \LF{163}{225\psq} \end{inparaenum}. Young Popper indeed appears to have articulated a distinction similar to the one between \s{preparation} and \s{measurement} formulated by \textcites{Margenau1936}{Margenau1937} around the same time\footnote{However, below I argue that Popper's notion of \s{physical selection} is not identical to the modern notion of \s{preparation}}. \begin{inparaenum}[(a)] \item He argues that \emph{not every physical selection amounts to a measurement}: for instance, a velocity filter produces an ensemble of electrons with definite momentum (a monochromatic plane wave), yet no momentum value is recorded; and conversely, \item \emph{not every measurement is a physical selection}: electrons from a monochromatic beam may be registered by a Geiger counter at a given location, though they are not physically separated by position \LF{163\psq}{225\psq}.
\end{inparaenum}


%As we have seen, \Weis\ objected that if an apparatus capable of measuring the position and momentum of a single particle with an accuracy exceeding Heisenberg’s \s{imprecision relations} were possible, then the same setup could, in principle, be used to produce an ensemble of particles all having the same momentum and position, thereby violating Heisenberg’s \s{variance relations} as well.

The two possible \emph{interpretations} of \cref{eq:ur} can then be rephrased as follows: \begin{inparaenum}[(a)] \item the \IR limit the degree of \emph{statistical homogeneity} that can be achieved in the \emph{physical selection} of an \emph{ensemble} of identical systems; \item the \IR limit the \emph{precision} of the \emph{measurement} performed on a \emph{single} system \end{inparaenum} \LF{164}{226\psq}. According to (a), given an ensemble of particles with sharp momenta (pure case), we are not allowed to select a sub-ensemble that also has a sharp position (what Popper called a super-pure case); (b) refers to the simultaneous measurement of position and momentum on a single particle, in which the two measurements hinder each other. Popper's program can then be articulated as follows: \begin{inparaenum}[I] \item (a) follows directly from the formalism; \item (b) does not \emph{logically} follow from (a); \item (b) follows from (a) only if one adds an \emph{additional hypothesis} (c); \item the combined system (a) + (c) turns out to be \emph{contradictory} \LF{154\psq}{215\psq}.
\end{inparaenum}

With regard to point~\rom{1}, Popper notes that it has rarely been observed that the mathematical derivability of the Heisenberg formulas from the fundamental equations of quantum mechanics must also be matched by a corresponding derivability of the interpretation of those formulas from the interpretation of the fundamental equations. Some, like Arthur~\citet[1\psq, 57]{March1931}, regard the statistical dispersions as a mere consequence of measurement inaccuracies; others, like \citet[68]{Weyl1928}, seem at least to accept the reverse, namely that the limitations on single measurements follow from the statistical view. Popper rejects both options. Quantum formulas make \emph{frequency predictions} \origg{H\"aufigkeitsprognosen}. The latter cannot imply prohibitions on individual measurements---apart from the trivial cases of probability~1 or~0 \LF{165\psq}{235\psq}. The \UR imply that if, say, 1000 position measurements show the electron to be at the same place, then it may be expected that 1000 momentum measurements would yield widely divergent results. A \s{sharp} measurement of both \pam---that is, an exceptionally precise outcome for both variables---is not impossible but highly improbable. It represents an event analogous to a statistical fluctuation within the ensemble distribution.

In this way Popper addressed point~\rom{2} of his program: from a logical point of view all alleged \s{proofs} that exact position and momentum \emph{measurements} contradict the formalism of quantum mechanics are flawed. The \IR do not \emph{forbid} exact position--momentum measurements; they merely maintain that exact position--momentum \emph{predictions} cannot be \emph{derived} from them \LF{166}{235\psq}. Heisenberg himself could not conceal the tension between these two alternatives: \begin{inparaenum}[(1)]
\item he argues that, since the \IR \emph{forbid} selecting an ensemble of systems with exact values of both position and momentum, \emph{predictive measurements} of the future position--momentum trajectory of a particle with arbitrary precision are excluded; \item he nevertheless concedes, almost in passing, that the \IR \emph{allow} arbitrary \emph{non-predictive measurements} of position and momentum. \end{inparaenum} These reconstructions can, in principle, be obtained by combining two predictive measurements: (a) position followed by position, (b) momentum followed by position, or (c) position followed by momentum \LF{156\psq, 167, 177}{219, 230, 240}.\label{abc} 

Because predictive position--momentum measurements cannot be derived from \qm, physicists generally conclude that the very notion of a \emph{future} trajectory has lost physical meaning in \qm \citep[55]{March1931}. Yet the question of the physical meaning of the \emph{past} trajectory between two measurements remains ambiguous. Popper could then enter the Heisenberg--Schlick debate presented in \cref{sub:heisenbergschlick}. \begin{inparaenum}[(a)] As we have seen, \item Heisenberg dismissed the issue as a \s{question of taste}, and \item Schlick regarded it as a \s{meaningless question}. \end{inparaenum} However, both ultimately agreed that, since such reconstructions cannot be \emph{verified}---that is, since no testable prediction can be derived from them---they pertain to the realm of \s{metaphysics}. As one might expect, however, for Popper the question in this form is ill-posed. The point is that if sharp past trajectory reconstructions were not possible, the frequency predictions of \qm could not be \emph{falsified}, and the theory would therefore belong to the realm of \s{metaphysics}. Without the possibility of reconstructing the past paths of particles, one could not subject the theory to empirical control \origg{Nachprüfung}. The latter is always, indeed, a comparison between the observed distribution of reconstructed paths and the predicted distribution \LF{169}{230\psq}.

%The observed distribution of these reconstructed paths (obtained through a position selection followed by a momentum measurement) must conform to the predicted distribution as the diffraction experiment is repeated many times; otherwise, the theory would have to be rejected \LF{169}{}.
%\todo{That position are real}

%In, say, a diffraction experiment, the position is physically selected at $x_1$ by means of a screen containing a slit. The theory predicts that the particles’ energy---and hence their speed---remains constant, while their direction, and thus the transverse component of momentum $p_y$, is spread. This can be subjected to empirical control by placing a photographic plate at $x_2$ and observing the distribution of spots. Each spot, corresponding to a single particle detected at position $\delta y$, allows one to infer the deflection angle $\delta \varphi$ and the corresponding past momentum component $\delta p_y$, so that the error product $\delta y \delta p_y$ can be made arbitrarily small. The observed distribution of these reconstructed paths (obtained through a position selection followed by a momentum measurement) must conform to the predicted distribution as the diffraction experiment is repeated many times; otherwise, the theory would have to be rejected \LF{169}{}.





\subsection{The Coupling-Hypothesis}
\label{sub:couplinghypothesis}

As one can see, Popper articulated an early version of what might be called an \s{objective statistical interpretation of \qm} \citep[50.2]{Pechenkin2022}, in which the collapse of the wave function is reinterpreted as the mere selection of a sub-ensemble. When a momentum measurement is performed, one in fact selects an ensemble of systems that have the same sharply defined momentum \LF{171}{235\psq}. In this view, the \IR imply that one cannot further select a sub-ensemble whose members also possess the same precisely defined position. By June 1934, Popper had managed to convince Weisskopf that the \IR \emph{forbid} physical selections of an ensemble more homogeneous than a pure case, but still \emph{allow} arbitrarily precise measurements on single systems (see above \cref{sec:popperquantun}). However, Weisskopf may have prompted Popper to take a further step. As we have seen, Weisskopf argued that if an apparatus capable of \emph{measuring} the position and momentum of a single particle with arbitrary sharpness were possible, then the same device could also be used to \emph{produce} ensembles of particles whose position and momentum spread would be smaller than that allowed by the \IR. In this sense, the \IR, as a prohibition of a super-pure case, also imply the prohibition of arbitrarily sharp predictive measurements. 

In \LdF, Popper attributes this objection to an unnamed physicist---without any doubt Weisskopf. Indeed, \citet[173--174]{Popper1935} reproduces several passages from Weisskopf’s letter almost verbatim, though without explicitly naming him as the source, as the latter had requested. Popper now offers a more articulated reply to \Weis{}’s objection. According to Popper, \Weis{}’s argument does not \emph{prove} that sharp predictions contradict quantum mechanics. Rather, it tacitly introduces an \emph{additional hypothesis}, which Popper calls the \emph{\CH} \origg{Kopplungshypothese}. The latter maintains that predictive measurements and physical selections are inevitably linked, so that any predictive measurement is also a physical selection---or, operationally speaking, that any instrument capable of measuring position and momentum with arbitrary precision could also be used to select a corresponding super-pure case, at least in principle:

\qt{But a strict proof of the contention \textelp{} has not been given \textelp{} None of these arguments prove that the precise predictions would contradict the quantum theory. They all introduce an \emph{additional  hypothesis}. The statement (which corresponds to Heisenberg’s view) that exact single predictions are impossible turns out to be equivalent to the hypothesis that predictive measurements and physical selections are inseparably coupled. With this new theoretical system---the conjunction of quantum theory with this auxiliary \emph{coupling-hypothesis}---my conception must indeed clash. What remains to be shown is that the system, consisting of statistically interpreted quantum mechanics (together with the laws of momentum and energy conservation), combined with the coupling hypothesis, is contradictory}{{Zu diesen Überlegungen bemerken wir zunächst daß sie vielleicht ganz plausibel sind ein strenger \emph{Beweis} dafür daß wenn eine prognostizierende Messung möglich ist auch eine entsprechende physikalische Aussonderung möglich sein müßte, kann jedoch (wie wir gleich sehen werden aus guten Gründen) nicht erbracht werden: infolgedessen beweisen diese Überlegungen nicht daß exakte Prognosen der Quantenmechanik widersprechen würden sondern sie führen eine \emph{zusätzliche Hypothese} ein; der (Heisenbergs Auffassung entsprechende) Satz, daß genaue Einzelprognosen unmöglich sind, erweist sich als äquivalent mit der Hypothese, da $\beta$ prognostizierende Messungen und physikalische Aussonderungen gekoppelt ${ }^1$ sind. Und dem theoretischen System: Quantenmechanik plus Koppelungshypothese muß unsere Auffassung natürlich widersprechen}}[\LF{174}{238\psq}]
%
Popper had now realized that the most deep-seated prejudice of quantum physicists lay in the unexamined belief that physical selections and measurements are necessarily coupled. With this, he considered point~III of his program accomplished. What remained to be addressed was point~IV: to demonstrate that the system---statistically interpreted quantum mechanics, together with the conservation laws of momentum and energy, when combined with the coupling hypothesis---leads to a contradiction. Popper deserves credit for having recognized with remarkable clarity the crucial point on which his whole critique of quantum mechanics rests---and upon which it also founders.

\section{Popper’s ETP-like Thought Experiment}
\label{sec:TE}

The \CH ultimately takes the form of a \emph{non-existence statement} \LF{33}{59}: no device exists that measures \pam\ with arbitrary sharpness without allowing the selection of a corresponding super-pure ensemble. Thus, to disprove the \CH, Popper needed to demonstrate the existence of such a device. In particular, he aimed to show that a device could be constructed which, on the basis of a \emph{non-predictive measurement}, would yield a \emph{predictive} one, but could not be used to perform a \emph{physical selection} producing a super-pure case \citep[\S77]{Popper1935}. As we have mentioned (\cref{abc}), there are three possible sequences of predictive measurements that yield a non-predictive measurement of a particle’s trajectory: (a) position followed by position, (b) momentum followed by position, and (c) position followed by momentum. The case considered by Popper---momentum selection followed by position measurement---belongs to type (b). The choice is not accidental. Indeed, in Popper's reading, this is \qt{just that case which, according to Heisenberg \textelp{}, permits \s{a calculation about the past of the electron}\footnote{See~\cref{sub:heisenbergschlick}}}{Es ist das wohl gerade jener Fall, der nach Heisenberg eine \s{Rechnung über die Vergangenheit des Elektrons} gestattet} \LF{177}{241\psq}

\subsection{The Experimental Setup}
\label{sub:setup}

As we have discussed above (\cref{sub:heisenbergschlick}), \citet[15]{Heisenberg1930} uses a sequence of momentum followed by position measurements (b) to argue for the possibility of a \s{calculation of the past} of the electron before the second position measurement. Popper seems to read this claim as implying that (b) allows one to calculate the past trajectory also \emph{before} the first momentum measurement, whereas (a) and (c) allow it only \emph{between} the two measurements\footnoteh{In my view, there is little evidence to support this interpretation. Heisenberg’s remark is simply somewhat careless}. As will become clear, this is the key but shaky premise on which Popper's argument hinges. On this basis, Popper proposed a thought experiment involving an electron--light quantum collision. After the collision, by performing a \emph{non-predictive} measurement on the electron (momentum followed by position), and using momentum conservation, one could in principle determine an arbitrarily sharp \emph{predictive} measurement of the position and momentum of the quantum of radiation. As one can see, Popper independently arrived at a \TE analogous to the ETP scenario (see \cref{sub:ETP} above), except that energy and time are replaced by position and momentum. Of course, Popper's goal was the opposite of ETP's: to show that the asymmetry between past and future does indeed allow one to overcome the \IR when interpreted as \s{inaccuracy relations}.

\begin{figure}
\centering
 \includegraphics[scale=0.22, trim = 0mm 0mm 0mm 0mm, clip]{popperexperimentGmod.png}
\caption{Popper's experimental setup. Adapted from \cite[179]{Popper1935}.}
 \label{fig:setup}
 \end{figure}

The idealized experimental setup (\cref{fig:setup})  consists of two beams intersecting at $S$ \origg{Schnittpunkt}. Each beam is treated as if it were a \s{pure case}. The electron beam $A$ is monochromatic, carrying a sharp known momentum $\mathfrak{a}_1$, while the light quantum beam $B$ is slit-selected, with a sharp position at $Bl$, and thus momentum of unsharp direction but known magnitude $\mathfrak{b}_1$. One can then perform a \emph{post hoc} mental selection focusing on the partial rays $[A]$ and $[B]$ that collide at $S$, in a manner similar to the Compton--Simon and Bothe--Geiger experiments. An electron from beam $A$ emerges at detector $X$ with momentum $\mathfrak{a}_2$, while a light quantum from beam $B$ emerges at detector $Y$ with momentum $\mathfrak{b}_2$. The process is governed by momentum conservation, such that $\mathfrak{a}_1 + \mathfrak{b}_1 = \mathfrak{a}_2 + \mathfrak{b}_2$. Since the light quantum has a broad momentum-direction distribution, in some runs of the experiment the electron \s{keeps} its original momentum, while in others it \s{inherits} the light quantum’s momentum. If one considers only the electrons, disregarding the light quanta with which they interacted, the electron beam is a non-pure case after the collision.

Popper suggests placing an apparatus at $X$ to measure the momentum $\mathfrak{a}_2$ of the electron after the collision and its position at the time $t_A$. In the Note for \NW, as well as in the main body of \LdF, he proposes using a moving photographic film \origg{Filmanstreifen}. When a particle strikes, it leaves a mark at a specific spatial coordinate (position) on the strip, and because the strip is moving, this also indicates a time coordinate (moment of impact). In this setup, the photographic film is also supposed to measure the energy of the electron indirectly through the energy transferred to the atoms of its sensitive material\footnote{See next section for a criticism of this setup}. From the energy one can calculate the electron’s momentum $\mathfrak{a}_2$ in the $SX$ direction. Popper contends that \qt{[t]he \emph{accuracy} of this calculation is, for the $SX$ direction (with a suitable arrangement), subject to no fundamental limitation of the kind implied by the uncertainty relations. It is assumed here that the magnitude of momentum and the time of an incoming particle---\s{non-predictive} measurements---can be measured with arbitrary precision}{Die \emph{Genauigkeit} dieser Berechnung ist für die $SX$-Richtung (bei geeigneter Anordnung) keiner grundsätzlichen Beschränkung von der Art der Ungenauigkeitsrelationen unterworfen, dabei wird vorausgesetzt, daß Impulsbetrag und Zeitpunkt eines einfallenden Teilchens \s{nichtprognostische} Messung beliebig genau meßbar ist} \citep[807]{Popper1934}.



Popper does not discuss this premise at this stage; he only argues that, \emph{if} this is the case, from $t_A$ and the momentum $\mathfrak{a}_2$ the time of collision $t_S$ can be reconstructed. Via momentum conservation, one can infer the light quantum’s momentum after the collision $\mathfrak{b}_2$ from the electron’s momentum, $\mathfrak{b}_2 = (\mathfrak{a}_1 + \mathfrak{b}_1) - \mathfrak{a}_2$. Consequently, from the time of the collision and the light quantum’s momentum $\mathfrak{b}_2$, one can make a predictive measurement sharper than allowed by the \IR: the light quantum from beam $B$ will reach $Y$ at time $t_B$ with momentum $\mathfrak{b}_2$. This prediction can, in principle, be tested empirically. Reconstructing the \emph{past of an electron} via an arbitrarily sharp non-predictive measurement together with a conservation law appears to allow an arbitrarily predictive measurement of the \emph{future of a light quantum}.  

Popper clarifies that the apparatus \qt{does \emph{not} permit the assignment of a definite position and a definite momentum arbitrarily to \myemph{particular} particles}{ ist charakteristiseh, dab es uns nicht gestattet, bestimmten TeiIehen einen bestimmten Ort und Versuchsschema. einen bestimmten hnpuIs willkfirlieh zu erteilen} \citep[807\me]{Popper1934}. One can infer something about $B$ once one measures $A$, but one cannot control or reproduce that same condition on $B$ in the next trial without physically selecting $A$ again. Repeating the experiment many times would yield a statistical distribution of $AB$ particle pairs, all with \emph{different} but correlated positions and momenta, not an ensemble of $B$ particles all having the \emph{same} position and momentum: \qt{The experiment therefore provides no indication for the production of a collection of particles that is more homogeneous than a \emph{pure case}}{Das Experiment gibt also keinen Anhaltspunkt zur Herstellung einer Teilchenmenge, die homogener ist als ein reiner Fall} \citep[807]{Popper1934}. Popper can then conclude that the \TE shows that one can determine a predictive measurement of the position and momentum of a single $B$-particle without allowing for the selection of an ensemble of $B$-particles having both sharp momentum and sharp position\footnote{According to \citet[166]{Redhead1995}, Popper is \q{confusing} conditional and predictive selection. I would rather say that Popper, more or less consciously, is \emph{exploiting} this very difference to infer $\mathfrak{b}_2$, given $\mathfrak{a}_2$, without physically selecting a new ensemble of $B$-particles. Popper should perhaps have used the label $[\mathfrak{b}_2]$ to emphasize that this is only an epistemic selection and not a physical one}. While fully complying with Heisenberg's inequalities \cref{eq:ur} understood as statistical dispersion relations \origg{Streuungsrelationen}, Popper believed that one could nevertheless produce a case that violates Heisenberg’s inequalities understood as limits of measurement accuracy \origg{Ungenauigkeitsrelationen}. 

\subsection{The Issue of the First Momentum Measurement}
\label{sub:momentum}

According to Popper, the possibility of making an arbitrarily sharp predictive measurement should suggest that each particle in the ensemble possesses a \emph{pre-assigned, yet unknown} momentum and position, thus opening the possibility that deterministic \s{precision laws} might underlie the quantum \s{frequency laws} \citep[\S78]{Popper1935}. As already noted, Popper was aware that the device underlying the \TE works only if the momentum of $A$ remains unaffected by the selection at $X$, allowing the entire path of $A$ to be reconstructed up to the collision at $S$. Popper was probably initially confident that this was the case, relying on \posscite[15]{Heisenberg1930} somewhat elliptical remark (see \cref{sec:past} above). An unpublished \emph{Ergänzung} to the Note, written sometime between summer and fall 1934, shows that Popper soon realized a legitimate objection could be raised against this assumption: \qt{One could, namely, take the following view: Non-predictive measurements allow us to make exact calculations only for certain intervals, namely for the interval \emph{between} two measurements. For the time \emph{before} the first or \emph{after} the second measurement, however, the uncertainty relation holds, whose validity is therefore (contrary to Heisenberg’s remark) symmetrical for past and future}{Man könnte nämlich folgende Auffassung vertreten: Nichtprognostische Messungen gestatten uns eine genaue Berechnung nur für bestimmte Intervalle, nämlich für die Intervalle zwischen zwei Messungen, etwa zwischen zwei Ortsmessungen oder zwischen einer Orts- und darauffolgenden Impulsmessung. Für den Zeitpunkt vor der ersten oder nach der zweiten Messung gilt die Ungenauigkeitsrelation, deren Geltung also (im Gegensatz zu Heisenbergs Bemerkung) für Vergangenheit und Zukunft symmetrisch wäre.} \citep[397\psq]{Popper1934a}. 
%
It may be recalled (\cref{sub:ETP}) that this was precisely the upshot of the ETP paper, which Popper, without any doubt, did not know of. Yet Popper sought to reach the opposite conclusion. He therefore had to offer at least a clarification concerning the measurement of momentum. In particular, he had to justify the assumption that the path of the $A$-particles could be reconstructed even \emph{before} the first momentum measurement, up to $S$. This discussion was later incorporated into Appendix~VI. This proves to be the true Achilles’ heel of Popper’s setup. As mentioned, Popper first proposed placing the film strip at $X$ to measure not only position and time but also energy, and hence momentum, of the incoming particle. He soon realized that such an apparatus would violate the \IR\ \citep[400]{Popper1934a}\footnote{See also Einstein's objection, \cref{sub:einsteinLdF}.}. In the \ERG, Popper therefore outlined a revised setup, later included in Appendix~VI of \LdF.

It can be described as follows (\cref{fig:setup}): \begin{inparaenum}[(1)] \item At $X_1$, an electrostatic analyzer with parallel plates producing a uniform electric field $E$ perpendicular to the beam direction\footnote{See below \cref{fig:weizsaeckerelectrical}} filters a sub-ensemble of electrons with the same \emph{predetermined} momentum $\mathfrak{a}_2$ \emph{without changing it} \LF{220\psq}{299\psq}. Since, for Popper, filtering momentum leaves both momentum and position unchanged, though the latter remains unknown; \item the unknown position may later be revealed by a second measurement at $X_2$, where a Geiger counter (or clock-driven moving film strip) records position $x$ and arrival time $t$. As the first measurement leaves the electron’s state unchanged, one might think it possible to reconstruct the past of the electron not only \emph{between} the two measurements---momentum followed by position---but also \emph{before} the first measurement at~$X_1$\footnoteh{Popper does not distinguish explicitly between $X_1$ and $X_2$, but I follow \posscite[164]{Redhead1995} suggestion.} \end{inparaenum}

Popper realized that two possible objections can be raised against the possibility of a momentum measurement as described:

\begin{itemize}
\item \emph{The filter leaves momentum unchanged but alters position}. Popper argues that, unlike a position selection (a narrow slit), which spreads the trajectory, a momentum selection (an electron spectrometer) allows particles of a certain momentum to pass along the same path while blocking others. Assuming that momentum selection unpredictably disturbs position would imply a discontinuous (superluminal) jump to another point, contradicting quantum mechanics, which permits discontinuities only for bound, not free, particles. Popper concludes that, according to the statistical interpretation of the \IR, momentum selection leaves positions unknown but \emph{unchanged} \citep[401]{Popper1934}.

%calls any theory postulating such position disturbances an \s{imprecision theory} \origg{Ungenauigkeitstheorie}; such a theory would be logically possible but empirically indistinguishable from standard \qm.\todo{check} He 

%There appears to be a comparatively simple crucial experiment for deciding between the ‘theory of indeterminacy’ (described above) and the quantum theory. According to the former theory, photons would arrive on a screen behind a highly selective filter (or spectrograph) even after the extinction of the source of light, for a period of time; and further, this ‘after-glow’ produced by the filter would last the longer the more highly selective the filter was.*2

\item \emph{If the filter left momentum unchanged, one could produce a \s{super-pure case}}. By reversing the order of selection---first localizing the particle through a narrow slit or short \s{momentary shutter} (\german{Momentverschluß}), then selecting momentum with a filter\label{shutter}\footnote{I mention this detail since Einstein referred to this apparatus; see below \cref{sub:einsteinEPR}}---one might expect, after many runs, to obtain an ensemble with sharp~$\pos$ and~$\mom$. Popper replies that, unlike momentum selection, position selection spreads momentum: the sharper the position, the greater the diffraction and the fewer particles pass the filter. Only rare detections occur, revealing merely the \emph{statistical distribution} of position and momentum in an \s{anonymous} ensemble, not sharp individual trajectories \LF{221\psq}{300\psq}.
\end{itemize}
%
Popper’s reasoning rests on an alleged asymmetry between position and momentum selection that, as will be shown below\footnote{See \nameref{sec:concl}}, is only apparent. What is relevant here is that Popper clearly sees that the question of the momentum measurement decides the destiny of his \TE{}: \q{[a] Assuming that the $x$-coordinates of the particles are not disturbed by the momentum measurement, then the exact determination of position and momentum also extends to the time \emph{before} the momentum selection [at $X_1$]. [b] Assuming that the momentum selection disturbs the $x$-coordinates, then we can calculate the trajectory exactly only for the time \emph{between} the two measurements [at $X_1$ and $X_2$]} \citep[220]{Popper1935}. As we shall see, Popper was compelled to respond to several interlocutors---some of the greatest physicists of all time---who challenged the legitimacy of~[a] and argued that~[b] was the case. He eventually gave it up, but not without a fight.

\section{The Leipzig Response}
\label{sec:leipizg}

\LdF\ was completed and sent to the publisher in the fall, and printed at the end of December 1934 (despite the colophon bearing the date 1935). In the intervening months, Berliner, the editor of the \textit{Naturwissenschaften}, had sent Heisenberg the galley proofs of the short Note that Popper had submitted in August. Heisenberg wrote to Popper on \datedm{23}{11}{1934}, explaining that, at Berliner’s request, he discussed the Note with his assistant \vW. They subsequently sent Berliner \vW's brief rejoinder, asking him to forward it to Popper: \q{You will then see what kind of criticism it is. Essentially, it concerns the concept of the \s{non-prognostic measurement,} which in our view appears to be misunderstood in your work} \letterKPAp{Heisenberg}{Popper}{23}{11}{1934}[305-32]. Let us not forget that Popper was virtually unknown at the time, and it is quite remarkable that Heisenberg (newly appointed to a chair at Leipzig) and his group chose to engage in a detailed refutation of various variants of his \TE rather than dismiss him as a crank.

\subsection{\VW's Objection and Popper's Reply}
\label{sub:weizobjection}

%Diese Bahn ist jedoeh prinzipiell unkontrolIierbar. 

In his brief response, \vW, like Heisenberg, suggested using a scattering method (the Doppler effect) to measure the electron’s momentum at $X_1$, followed by a Geiger counter and clockwork to record its position and time at $X_2$. With this setup, \vW reached precisely the conclusion Popper sought to avoid: the electron \s{trajectory} $A$ can be reconstructed only for the interval \emph{between} $X_1$ (momentum determination) and $X_2$ (position and time recording), but not \emph{before} $X_1$. He conceded that one may know the momentum before and after the measurement exactly; however, measuring momentum via the Doppler effect requires determining the frequency ($\Delta \nu$) of the scattered radiation over a finite observation time ($\Delta t$); improving one worsens the other. Thus, one cannot know when the impact between the apparatus and the particle occurred, or for how long it had the \s{momentum before the impact} or \s{after the impact}. This uncertainty blurs the knowledge of position $X$ before the impact; hence, the collision time at $S$ remains indeterminate, and the trajectory [B] cannot be predicted. \vW concluded:

\qt{This trajectory, however, is \emph{in principle} uncontrollable. It is valid only for the time interval between the end of the momentum measurement and the beginning of the position measurement, during which the particle has no interaction whatsoever with its surroundings, and it cannot be extended into the period before the momentum measurement, since the latter itself destroys the knowledge of the position in accordance with the uncertainty relation. \textelp{} Thus, even if the position [at $X_2$] after the momentum measurement is exactly known, one still cannot infer the position \emph{before} the momentum measurement [at $X_1$] with an accuracy greater than the error $h / 4 \pi \Delta p$. That is, our knowledge of the trajectory of particle $A$ before the momentum measurement at $X$, and thus also of the trajectory of particle $B$ after the collision at $S$, is in accordance with the uncertainty relation}{Diese Bahn ist jedoeh prinzipiell unkontrolIierbar. Sie gilt n~imlich nur ffir das Zeitintervali zwischen dem Ende der Impulsmessung und denl Beginn der Orts-
messung, in dem das Teilchen iiberhaupt keine Weehselwirkung mit seiner Umgebung hat, und EiBt sich nieht in den Zeitraum vet der Impulsmessung fortsetzen, da diese ihrerseits die Kenntnis des Orts gem/ifl der Ungenauigkeitsrelation zerstSrt. }[(\vW in \cite[808]{Popper1934})]
%
\VW concedes that the proof applies only to the \emph{specific} experimental setup he described. However, he had no reason to doubt that \emph{any} other combination of measuring devices at $X$ would yield the same result. The problem is that Popper does not distinguish \s{non-predictive measurements} and verifiable measurements concerning the past. The former escape the \IR since they are not physical measurements at all; the latter, by contrast, do satisfy the \IR, which are symmetric for the past and future.

On \datedm{26}{11}{1934}, Popper replied by sending the galley proofs of Appendices VI and VII of the \Lo to Heisenberg \letterKPAp{Popper}{Heisenberg}{26}{11}{1934}[305-52]. Popper, somewhat prone to paranoia, felt that \vW\ had not taken him seriously and reacted rather aggressively, emphasizing two points: \begin{inparaenum} \item in the Note he did not \emph{prove} the legitimacy of {calculating the past of the electron}, but merely \emph{assumed} it, citing Heisenberg’s own remark on p.~15 of his book as justification; \item von {Weizsäcker} had refuted only one {method} of \emph{momentum measurement}---that by scattering. Popper maintained instead that \emph{momentum selection} of an aggregate of electrons \origg{Teilchenmenge} by means of filters could determine momentum \mom \emph{without} disturbing the \pos-coordinate, thus allowing for a non-predictive retrodiction before $X_1$. \end{inparaenum} Popper wrote to Heisenberg:

\qt{You will now ask why I did not elaborate further in the note on what I am writing here. The reason is that I assumed that the so-called \s{non-prognostic measurement} (momentum selection followed by a position measurement) and the possibility of calculating the \s{past} prior to the momentum measurement were already known. This assumption was based on a remark of yours (\textit{Phys. Prinzipien}, p. 15) concerning case (b) and the \s{calculation of the past of the electron}. In my note, however, I had to be brief and restrict myself to what was essential, that is, above all, to what was new. Only from Mr. Weizsäcker’s note did I realize that at least he was not familiar with this case}{Sie werden nun fragen, warum ich des, was ich hier schreibe, in der Note nicht näher ausgeführt habe. Der Grund ist, der, dass ich annahm, dass die fragliche \s{nichtprognostische Messung} (Impulsaussonderung mit nachfolgender Ortsmessung) und die Möglichkeit, die \s{Vergangenheit} vor der Impulsmessung zu berechnen, bekannt ist. Diese Annahme gründete sich auf eine Bemerkung von Ihnen (Phys. Prinzipien, S. 15) über den Fall (b) und die \s{Rechnung über die Vergangenheit des Elektrons}. In meiner Note musste ich mich aber kurz fassen und nur das Wichtigste bringen, also vor allem das Neue. Erst aus Herrn Weizsäckers Note bemerkte ich, dass zumindest ihm dieser Fall nicht bekannt ist, und ich muss nun annehmen, dass er entweder auch Ihnen nicht bekannt war (und ich aus der zitierten Stelle Ihres Buches zuviel herausgelesen habe), oder dass Sie, was ich sehr begreiflich finden würde, meine Note nicht recht ernst genommen haben, so dass Ihnen dieser Zusammenhang entgangen ist}[\letterKPAp{Popper}{Heisenberg}{26}{11}{1934}[305-52]] 
%
Since he mentions p.~15 of the German edition of Heisenberg's 1930 book, Popper probably refers to Heisenberg's claim that the \s{calculation of the past of the electron} can be reconstructed \emph{before the position measurement}---without specifying whether it can also be reconstructed before the momentum selection. At any rate, the time was too short for Popper to prepare a rejoinder. On \datef{30}{11}{1934}, the Note appeared in volume 22 of \NW, together with \vW's critical comment.

%He now had to assume either that it was also unknown to Heisenberg (and that he had read too much into the cited passage) or that Heisenberg, as he could well understand, had not taken his note very seriously, so that this connection had escaped his attention.





%(406)
 On \datedm{6}{10}{1934}, Popper sent Heisenberg a second communication \citep{Popper1934b} in response to \vW, which he hoped to publish. Popper also attached a copy of the \LdF that had been just printed but still not distributed. Popper's counterargument is revealing of his way of thinking at that time\footnote{As discussed below (see \nameref{sec:concl}), his reconstruction of the event twenty years later seems to me quite different}. He observes that \vW\ had used momentum \emph{measurement} at $X_1$ via the Doppler effect, that is, by using light quantum collision. The latter, however, has the shortcoming that it \qt{disturbs the particle’s momentum (recoil) and that, since the exact instant of the disturbance is not known}{den Impuls des Teilchens stört (Rückstoß) und daß, da der genaue Zeitpunkt der Störung nicht bekannt ist} (406). Popper concedes that no reconstruction of the momentum before the scattering event is possible in this way. However, following the same argument he had developed in Appendix VI, Popper suggests that one can resort to \q{a momentum \emph{selection} (electron spectrometer, light filter)}, which \qt{unlike the momentum measurement of individual particles, for example by means of the Doppler effect---has the property of not affecting the momenta or momentum components of the selected particles}{eine Impuls\emph{aussonderung} (Elektronenspektralapparat, Lichtfilter) hat nämlich-- im Gegensatz zu der Impulsmessung individueller Teilchen, etwa mit Hilfe des Dopplereffektes -- die Eigenschaft, die Impulse bzw. die Impulskomponenten der ausgesonderten Teilchen nicht zu beeinflussen} \citep[406]{Popper1934b}. Thus, if $[A]$ is an electron beam, \qt{only electrons with a certain \myemph{predetermined} amount of momentum enter the Geiger counter}{nur Elektronen mit einem gewissen vorgegebenen Impulsbetrag in den Spitzenzähler einfallen}, and if $[A]$ is a light beam, the filter lets through only those light quanta that \emph{already} possessed the required momentum before entering it \citep[406]{Popper1934b}.

One might object, Popper points out, that we do not really know in detail what a light filter does, and that its operation might itself disturb the light quantum's state. However, Popper argues, a filter produces not only light but also sharp images \origg{Bilder}---thus, it does not, in any case, disturb the directions of the quanta. If the filter did disturb the light quanta, one would have to assume some mechanism that restores their direction after passage: either (1) the quanta pass through the filter but are unpredictably displaced along their paths, with their previous direction suddenly reestablished \origg{Zurückversetzung auf der Bahn}, or (2) the filtering process involves absorption and directed re-emission \origg{gerichtete Emission} in the same direction \citep[407]{Popper1934b}. Yet both options appear problematic: the first contradicts the continuity of paths still required by quantum theory; the second assumes a kind of \s{directed fluorescence} not known to occur\footnote{\s{Fluorescence} is the emission of electromagnetic radiation by atoms or molecules after absorbing radiation, occurring when they return to a lower energy state. The emission is generally spontaneous and isotropic, not directed} \citep[407]{Popper1934b}. Popper therefore concluded that the most plausible hypothesis is that, also in the case of the filter, the quanta with the \s{right} momentum are allowed to pass through unchanged. 

\subsection{\vW's and Heisenberg's Counter-Replies}
\label{sub:heisobjection}

\begin{figure}
\centering
 \includegraphics[scale=0.3, trim = 0mm 0mm 0mm 0mm, clip]{1934WeizsaeckerElectrical.png}
 \caption{From the letter by \letterKPA{\vW}{Popper}{6}{12}{1934}[360-21]}
\label{fig:weizsaeckerelectrical}
\end{figure}

% D: The slit system (diaphragm) of width $d$. This restricts the transverse position of the incoming particle beam.
%- A and B: The plates of a capacitor (or deflecting field) that bend the particle trajectory.
%- C: The curved trajectory of the particle inside the field.
%- E-F: The vertical displacement $y$ at the detector plane (the "screen" at the right).
%- $L$ : The free flight distance $L$ (the length of the field region).

On \datedm{6}{12}{1934}, the same day as Popper's letter, \vW drafted a detailed response \letterKPAp{\vW}{Popper}{6}{12}{1934}[360-21], which Heisenberg forwarded to Popper on \datedm{10}{12}{1934}. \VW\ considered the setup suggested by Popper in Appendix VI, an electrostatic analyzer (see \cref{fig:weizsaeckerelectrical}), and showed that the very geometry of the deflection experiment already contains the limits expressed by the uncertainty principle\footnoteh{(See appendix \cref{app:1} for more details)}. To determine the momentum of the particle from its impact point on the screen $EF$, the beam must pass through diaphragms that fix its path length precisely. But narrowing the slit $D$ to improve accuracy inevitably produces diffraction, so that the transverse spread $\Delta y=EF$ of the beam at the detector cannot be made smaller than a definite bound. In other words, the position on the screen, which is supposed to measure momentum, is blurred by the same diffraction effects introduced by the diaphragms \letterKPAp{\vW}{Popper}{6}{12}{1934}[360-21].  

Since the vertical displacement $y$ depends on the flight time $t=L/v$, this 
transverse uncertainty also propagates into the longitudinal motion. An 
uncertainty in $y$ translates into an uncertainty in the velocity $v$, and thus 
into both the longitudinal momentum $p_x$ and the coordinate $x=vt$. When these effects are consistently taken into account, one finds that the product 
$\Delta p_x \Delta x$ is likewise bounded below by Planck’s constant. \Cref{fig:weizsaeckerelectrical} illustrates the intuition: the more tightly the beam is collimated at $D$, the more it spreads out when it reaches the screen at $E\!-\!F$, so that no arrangement of diaphragms or fields can evade the uncertainty relation. The situation was not dissimilar to the one described by \citet[21\psq]{Heisenberg1930} in his book, where he used a homogeneous magnetic field instead of an electric field to measure the momentum \letterKPAp{\vW}{Popper}{6}{12}{1934}[360-21].

In his reply of \datedm{10}{12}{1934}, Heisenberg enclosed \vW's criticism of the electrostatic analyzer and also addressed Popper’s suggestion of using a light filter instead of an electron spectrometer \letterKPAp{Heisenberg}{Popper}{10}{12}{1934}[305-32]. Heisenberg suggested that one might use reflective resonance filters, such as sodium or mercury vapor, following Wood’s classical residual-ray experiments \origg{Reststrahlenverfahren}. When light strikes such a medium, radiation at the resonance frequencies is strongly absorbed and re-emitted (or reflected), while other frequencies pass through unaffected. Thus, a broad-spectrum beam incident on sodium vapor is filtered and yields only the reflected sodium D-lines---a doublet of closely spaced spectral lines in the yellow region of the visible spectrum. Heisenberg invoked Wood’s experiment to show that the filtering process always involves finite line widths, not infinitely sharp spectra of a single frequency $\nu$. When an atom absorbs a light quantum at its resonance frequency, it is excited to a higher state and, after a short lifetime $t$, decays by re-emitting a light quantum at the same frequency. Because the excited state exists only for a finite time, its energy cannot be perfectly sharp, in compliance with the time--energy uncertainty relation.

Even when light is \s{filtered} into sharp momentum (frequency) states, the finite lifetime of the excited atoms ensures a residual spread $\Delta \nu$ (and hence $\Delta p$). This constitutes a momentum selection, since light quantum momentum is $p = h/\lambda$, but its precision is limited by the finite resonance linewidth, determined by the lifetime of the excited state, $\Delta t = 1/\Delta \nu$. Thus, light quanta cannot be prepared with arbitrarily sharp momentum. Even if the light quantum’s later position is known precisely, its position before reflection cannot be reconstructed, because the time spent in the reflecting atom is indeterminate. Hence there is an uncertainty $\Delta x$ such that $\Delta x \cdot \Delta p \approx \frac{h}{4\pi}$. The filter selects not an exact frequency, but a narrow band, whose spread is intrinsic to quantum mechanics through the lifetime--linewidth relation.

In his reply of \datedm{16}{12}{1934}, Popper sent Heisenberg a revised version of the second communication summarizing the debate’s main points \citep{Popper1934c}. He conceded that \vW's argument against the electrostatic selector was correct but defended his idea of using a light quantum filter against Heisenberg’s objections. He noted that Heisenberg’s remarks on Wood’s experiment concerned only absorbed and re-emitted light quanta at the resonance line, whereas \q{the light of other frequencies \myemph{passes through}, and it is only to this transmitted light that my considerations apply} \letterKPAp{Popper}{Heisenberg}{16}{12}{1934}[305-32]. Only by assuming that transmitted light also underwent absorption and re-emission would the \TE lose validity. He further dismissed the claim that the transmitted light was too inhomogeneous\todo{check}: \q{It is in fact possible, by means of a set of filters, or even with a green filter alone, to isolate a finite range of transmitted frequencies $\Delta \nu$, i.e.\ a finite $\Delta p$ interval} \letterKPAp{Popper}{Heisenberg}{16}{12}{1934}[305-32]. Once the light quantum had passed through the filter, its position could later be measured with arbitrary accuracy $\Delta x \to 0$, so that \q{the \emph{product} of the two uncertainties could approach zero, even for imperfect filters} \letterKPAp{Popper}{Heisenberg}{16}{12}{1934}[305-32].

Popper did not relent. He proposed a new idea for an apparatus, which Heisenberg left to his assistant Hans Euler to examine \citep[293-29]{KPA}. On \datef{4}{2}{1935}, Euler sent Popper a detailed refutation. On \datedm{12}{2}{1935}, Popper drafted a somewhat impatient response that was never sent, and another on \datedm{14}{4}{1935}. I will not go into this exchange, since Popper’s original proposal has been lost, and he seems to have later abandoned the idea. At this point, however, the Leipzig group was starting to become mildly annoyed by Popper's insistence. In a mid-March letter to Grete Hermann, \vW, at her request for clarification, reconstructs the entire series of experimental conjectures and refutations, concluding somewhat humorously that it has become \q{a parlor game for us in Leipzig, to refute Popper’s setups} (\letter{\vW}{Hermann}{13}{3}{1935}, in \cite[Brief 15]{Herrmann2019})\footnote{The correspondence between \vW and Hermann, and \posscites{Hermann1935}[55\fn{10}]{Hermann1935a} on Popper's interpretation of \qm would require a separate investigation\hide{\citep[see][]{Frappier2016}. However, it is worth noting that Hermann goes to the heart of the matter, challenging Heisenberg’s concession that the reality of past trajectories is merely a \emph{matter of taste} \citep[55\psq]{Hermann1935}. According to Hermann, such reality is, so to speak, a \emph{matter of context}. In Popper’s parlance, each sequence of a selection and a measurement defines a complete and self-consistent \s{trajectory}. Yet this trajectory does not exist in itself. Different measurement sequences (momentum-position vs. position-momentum) yield different reconstructed paths}}. It was Heisenberg who ultimately decided that it was time to bring the game to an end a few days later. 

In his final letter to Popper, on \datedm{19}{3}{1935}, Heisenberg explained that all the considerations of his previous letter apply to transmitted light as well \letterKPAp{Heisenberg}{Popper}{19}{3}{1935}[305-32]. He considered a substance opaque over nearly the entire spectrum due to strong absorption bands, yet transparent within a narrow region between two band heads \origg{Bandk\"opfe}. Light quanta with frequencies inside this window pass quickly, while those near the band heads are absorbed and re-emitted after a delay comparable to the lifetime of the excited states. As the interval between the band heads narrows, more quanta experience delayed re-emission, lengthening and blurring the passage time. In the limit of an infinitesimally narrow band, the mean delay grows without bound, consistent with the \s{principle of harmonic resolving power} or Fourier relation $\Delta \nu \Delta t$ or $\Delta x \Delta k$: a sharp frequency (or momentum) definition entails an indeterminacy in time (or position) \letterKPAp{Heisenberg}{Popper}{19}{3}{1935}[305-32]. At this stage, Popper might have sought support from Weisskopf. Yet the latter confessed that he had met with Heisenberg and had come to side with him and \vW \letterKPAp{\Weis}{Popper}{21}{1}{1934}. The game had run its course---but Popper still had one card up his sleeve.


%https://cqi.inf.usi.ch/qic/grete.pdf

\section{The Popper--Einstein Correspondence}
\label{sec:einstein}

On \datef{25}{3}{1935}, Einstein, Podolsky, and Rosen submitted the eponymous EPR paper to the \emph{Physical Review}; it was published on \datem{1}{5}{1935}. In the intervening weeks, Popper managed to get a copy of \LdF\ to Einstein. Popper’s friend, the pianist Rudolf Serkin, performed with the Busch Quartet and had recently married the daughter of Frieda and Adolf Busch, who knew Einstein. Frieda sent him a copy of Popper's book on \datef{28}{4}{1935} \letteraeap{Busch}{Einstein}{28}{4}{1935}[34--338]. As one can see, Einstein received Popper’s work after the EPR paper had already been submitted. \citets{Jammer1974} hypothesis that Popper's paper might have served as an intermediary between the ETP and the EPR paper is intriguing. As mentioned above, Popper's experiment is essentially a version of the ETP experiment in which energy and time are replaced by momentum and position, as in the EPR argument. It therefore seems natural to suggest that he may have served as a bridge between ETP and EPR. However, this hypothesis is ultimately unlikely for chronological reasons\footnote{I could not find any indication that Popper had already sent an offprint of the 1934 Note to Einstein in December, as claimed by \citet[178]{Jammer1974}}.\todo{check} However, it is plausible that Einstein immediately recognized that Popper’s experimental setup was essentially a variant of his own ETP scheme and thus immediately dismissed it.

\subsection{Einstein's Comment on the \textnormal{Logik der Forschung}}
\label{sub:einsteinLdF}

Einstein replied to Popper on \datedm{15}{6}{1935}, endorsing his general philosophy of science%
%
\footnoteh{This was possibly more than a merely rhetorical endorsement. In a 1984 letter to Popper, John Stachel, at that time the editor of Einstein's \emph{Collected Papers}, pointed out that, in a short newspaper article, \citet{Einstein1919-12-25} had anticipated the main lines of his philosophy, combining deductivism with the invention of hypotheses and their falsification as a means of control \citep[292-12]{KPA}} 
%
but criticizing the proposed thought experiment as flawed: \q{It is not correct that position and momentum of the particle at $Y$ (p. 179) can be predicted. To this end you would have to measure with the help of your \s{Film} time and momentum of the particle [at $X$] (that is, time and energy) which is impossible. You will see this easily if you think a little more about it} \letteraeap{Einstein}{Popper}{15}{6}{1935}[19-124]. The objection is probably the consequence of Popper's unfortunate choice of measuring apparatus on p.~179 of \LdF, a \s{film} at $X$ measuring momentum and energy, and position and time of arrival. 

However, as discussed above (\cref{sub:momentum}), in Appendix~VI of \LdF Popper had already proposed a different two-stage arrangement, a momentum filter at $X_1$ (electron spectrometer or light filter) and a position--time measuring apparatus (a film) at $X_2$. Still, in his lengthy reply on \datedm{18}{7}{1935}, he had by that time already accepted that even this setup could not work:

\qt{First, the question of the thought experiment (in the book p. 179). Unfortunately, regarding the measurement of the particle arriving at $X$---as you rightly noted---there are quite a few inaccuracies in the text of the book (for example, the use of the electric field and practically the whole Appendix VI). I admit my mistakes, but after having discussed them thoroughly with all the physicists available to me in Vienna, among others with Professor Thirring and his assistant Guth (with Professor Heisenberg and his assistant Weizsäcker I discussed the matter in writing), it still seems to me that the issue is not yet settled. From the discussions so far (and in this view I am not alone, but, among others, also Guth), the impossibility of my thought experiment has not yet been demonstrated. That is, it seems possible to avoid the aforementioned errors and to maintain the thought experiment in agreement with quantum mechanics. I do not wish to be \s{proven right}; I would only be very glad if this matter could be clarified --- even if the final decision goes against me. I hope I am not abusing your time and patience too much by presenting the situation as it currently appears to me}{Zunächst die Frage des Gedankenexperiments (im Buch S. 179). Hier stimmt leider bezüglich der Messung des bei $X$ eintreffenden Teilchens - wie Sie js bemerkten - im Buch in der Tet einiges nicht (z.B. die Verwendung des elektrischen Feldes und so ziemlich der genze Anhang VI). Ich sehe meine Fehler ein. ther wohdem ich mit :llen fur mich in Wien erreichbren Physikern, u.n. mit Professor Thärring und dessen Assistenten Guth eingehend diskutiert habe (mit Professor Heisenberg und dessen ssistenten ïeizsäcker hebe ich schriftlich diskutiert), scheint mir die Bache noch imer so zu sein, duss us der bisherigen Diskussion (dieser 'nsicht bin nicht nur ich, sondera u.n. ruc Guth) die Unasglichkeit neines Geds niceuexperinents noch nicht hervorgel d.h. es scheint möglich zu sein, die erwähnten schler zu vermeiden und dss Gedrnkenexperiment rls mit der uentennechnnik vereinbs rufreoht zu erholten. Ich mocht e nicht "Recht behr lten"; ich wäre nur sehr froh, wean diese Sache zu einer Klarung küne, - ruch drnn, wenn die Katsoheidung gegen mich fillt. Ich hoffe, Ihre zeit und Ihre "eduld nicht llzu seh zu missorauchen, wem ich thon die itustion, wie sie mir geomartig erscheint, drrstelle.}[\letteraeap{Popper}{Einstein}{18}{7}{1935}[19-126]]
%
Popper tried to investigate a different arrangement, in which a light filter is placed at $X_1$. He acknowledged that this setup---the optical version of the one described in the book---would not yield the desired result by using known filters (as Heisenberg showed him). However, Popper suspected that a combination of filters might exist that selected photons of a sharply defined momentum (frequency) without having them interact for a long time. 

Popper went into some detail (see next Section) but, not wanting to make an already long eight-page letter even longer, he promised to send Einstein a supplement he had written two months earlier at Felix Ehrenhaft’s suggestion \letteraeap{Popper}{Einstein}{18}{7}{1935}[19-126]. Throughout the spring of 1935, Popper had worked on a published response to the Leipzig objections, expanding it into a longer manuscript again titled \bt{Zur Kritik der Ungenauigkeitsrelationen}. Apparently he sent it to Carnap, Ehrenhaft, Heisenberg, and Schrödinger, hoping Ehrenhaft would forward it to Einstein (\letter{Popper}{Carnap}{10}{6}{1935}; Carnap Collection). On \datedm{18}{7}{1935}, he decided to send it himself. Popper admitted he was beset by doubts but, after further discussion with Weisskopf, realized the latter could not fully refute his argument. Having already announced the manuscript, he felt compelled to send it despite lingering uncertainties, having given up hope of a prompt clarification. Moreover, Weisskopf had informed him of Einstein’s recent critique of quantum mechanics (the EPR paper), which might have nurtured his hopes of finding in Einstein a more receptive audience \letteraeap{Popper}{Einstein}{29}{8}{1935}[19-127].

\todo{\footnote{In his commentary to Einstein's letter Popper claims that Einstein was working on the same problem}.}.

%; \letter{Popper}{von Mises}{26}{6}{1935}, Popper Archives [329.4]
%Do 30 12h plötzlich Popper. Sein Buch ist im Druck; er erzählt von neuer Deutung der Unbestimmtheitsrelation; 

\subsection{Popper's 1935 Manuscript}
\label{sub:1935ms}

The first part of Popper's 1935 manuscript restates his stance toward the \UR\ with additional reference to some literature that he had not cited in the book, such as von Neumann's \citeyear{Neumann1932} textbook. Thus, for example, Popper began to use the expression \s{dispersion-free ensemble} to indicate what he had previously called a \s{super-pure case} \citep[48]{Popper1935b}. However, the main message remains the same: the prohibition of dispersion-free ensembles does not imply the prohibition of arbitrarily sharp measurements. Popper presupposes that the reader already knows his November 1934 Note \todo{When did he attached the note?} and once again clearly identified the issue with his original setup:

\qt{One would have to apply a \s{non-prognostic measurement} according to case (b)---a position measurement [at $X_2$] preceded by a momentum measurement at [at $X_1$]---arranged in such a way that one can calculate the trajectory not only in the interval \myemph{between} the two measurements, but also for the time \myemph{before} the momentum measurement. The position coordinate is measured precisely by the peak counter (the moment of incidence [at $X_2$]). \emph{Question: Is there a method of momentum measurement [at $X_1$] that could precede this position measurement without destroying (blurring) the position coordinate of the particle?}. Weizsäcker disputes this\footnote{See above \cref{sub:weizobjection}}; the case he discusses (momentum measurement of the particle using the Doppler effect) is indeed unsuitable. In my book I propose the following arrangement: 1. Momentum selection by means of a grating filter,   2. Position measurement by means of the counter.   There I defend the thesis that such a momentum measurement, carried out with a filter, should not disturb the position coordinates; this thesis should here be examined more closely}{}[\citep[427]{Popper1935b}]
%
In other terms, Popper believes he could devise a filter that would let light quanta with unchanged momentum---and therefore position---pass through. His reasoning seems to have been roughly as follows.

A \emph{real} filter must satisfy the Fourier constraint: the narrower the transmitted range of frequencies $\Delta \nu$ or wavelengths $\Delta \lambda$---that is, the sharper the momentum selection---the longer the temporal response $\Delta t$ and the greater the spatial extension $\Delta x$ of the transmitted wave packet. However, Popper argues that one could in principle conceive of two different kinds of \emph{ideal} filters, corresponding to the limiting cases of the Fourier relation \citep[428]{Popper1935b}:

\begin{itemize}
\item \emph{Type I apparatus:} Filters capable of spectrally resolving arbitrarily short light flashes ($\Delta t \rightarrow 0,  \Delta \nu \rightarrow \infty$). Such filters must necessarily operate \emph{with spatial blurring}, since decomposing a very short pulse without spreading its position $\Delta x$ would contradict the Fourier relation:

\item \emph{Type II apparatus:} Filters that cannot resolve very short wave packets but act only on infinitely long wave trains ($\Delta \nu \rightarrow 0,  \Delta t \rightarrow \infty$). In principle, such filters could operate \emph{without spatial blurring}, transmitting only radiation of a single wavelength (or momentum).
\end{itemize}
%
To construct a filter of type II, Popper suggests using colored glass filters followed by gas layers \citep[429]{Popper1935b}. The glass filters, say a green filter, are passive filters that work by transmission: they allow a relatively broad band of wavelengths centered around some mean wavelength to pass while blocking both the high-frequency and low-frequency edges. Gas layers act actively, through absorption and delayed re-emission, and provide a further selection (since each gas has sharp absorption lines), ideally centered well away from any resonance edges that cause re-emission or time delay.

Popper probably believed that the double selection would exclude the spectral region around the resonance band edges, which was the basis of Heisenberg’s objection (see above \cref{sub:heisobjection}). This would ensure that the quanta that go through the multi-stage filter are transmitted directly without delay and not absorbed and re-emitted. For this reason, one might assume that their momentum (frequency) is left unchanged. This implies that no mechanism exists that could alter their positions. The filter, Popper concludes, selects a homogeneous ensemble of light quanta with the same momentum, whose maximally spread individual positions are completely \emph{unknown} yet remain unchanged. They can then be reconstructed by a subsequent position measurement: \qt{\emph{A momentum-selection apparatus without spatial smearing would therefore not contradict quantum mechanics if it is of type (II)}}{Ein Impulsaussonderungsapparat ohne Ortsverschmierung würde also der Quantenmechanik nicht widersprechen, wenn er vom Typus (II) ist} \citep[429]{Popper1935b}.

\subsection{Einstein's Comment on Popper's Manuscript}
\label{sub:einsteinEPR}

Einstein replied only a few weeks later. In a letter of \datedm{11}{9}{1935}, acknowledging the broad direction of Popper’s argument, he rejected the details of the experimental setup Popper proposed:

\qt{Dear Mr. Popper,\\
I have looked at your paper, and I largely \origins{weitgehend} agree. Only I do
not believe in the possibility of producing a ‘super-pure case’ which
would allow us to predict position and momentum (colour) of a pho-
ton with ‘inadmissible’ precision. The means proposed by you (a
screen with a fast shutter in conjunction with a selective set of glass
filters) I hold to be ineffective in principle, for the reason that I firmly
believe that a filter of this kind would act in such a way as to ‘smear’ the
position, just like a spectroscopic grid}{}[\letteraeap{Einstein}{Popper}{11}{9}{1935}[19-130]] 
%
This passage is somewhat puzzling. Einstein seems to think that Popper was proposing a method to prepare a super-pure case using a slit with shutter followed by filters. However, just as he did in \LdF, in the manuscript \citet[401--403]{Popper1935b} uses the example of the screen with fast shutter followed by a filter to \emph{exclude} that his setup would allow the construction of a super-pure case by measuring position and momentum (see above \cref{shutter}). 

Popper's proposal was a light filter followed by a Geiger counter. Einstein may have gone through the manuscript rather quickly; yet the last part of his objection did hit the mark. The glass filter decomposes a short light pulse into quasi-monochromatic wave trains $W_n$. Absorbing filters cut out all colors $W_n$ except one, $W_1$. The filter thereby places light quanta in a state with sharp momentum $W_1$, but at the cost of making position unsharp \letteraeap{Einstein}{Popper}{11}{9}{1935}[19-130]. As Popper realized early on, if the filter smears the position, his \TE breaks down. The $A$ path reconstruction stops at the momentum measurement; one cannot determine the time of collision at $S$, and thus one cannot determine arbitrarily sharp predictions of the $B$ particles via a conservation law. This is exactly the result that ETP had obtained a few years earlier.



Referring to his joint paper with Podolsky and Rosen (\citeyear{Einstein1935}), Einstein explained that he did not have any copies at hand but could briefly summarize its content. The situation is indeed not dissimilar to that of Popper’s \TE. After, say, a collision, two particles are described by the joint, nonfactorizable wave function\footnote{As is well known, the term \s{entanglement} was first introduced by \textcites{Schroedinger1935}{Schrodinger1935b} in the ensuing months} $\psi(x_1, x_2)$, with sharp relative position ($x_B - x_A = 0$) and sharp total momentum ($p_A + p_B = 0$)---but spread individual particle momenta. If one measures, say, the momentum $p_A$ of particle $A$, its state collapses into the corresponding momentum eigenstate $\psi_A$; quantum mechanics then assigns to subsystem $B$ a conditional momentum eigenstate $\psi_B$. For this state, however, the position $x_B$ is completely undetermined, which blocks any attempt to beat the \UR, as Popper had hoped. Einstein’s point, rather, was that the specific form of $\psi_B$ would have been different had one performed a position measurement on system $A$. This means that two different $\psi$-functions correspond to one and the same physical state of $B$, which itself has not changed physically: \q{It is therefore not possible to regard the $\psi$-function as the complete description of the state of the system.} \letteraeap{Einstein}{Popper}{11}{9}{1935}[19-130].

%https://www.mprl-series.mpg.de/proceedings/3/6/

In Einstein's contemporary correspondence, the \s{private} formulation of Einstein’s argument ends here \letteraeap{Einstein}{Schrödinger}{19}{6}{1935}[22-047], without reference to the \IR{}. Rather, it involves the measurement of a single variable together with a conservation law or geometrical correlation \citepp{Howard1985}[38]{Fine1986}. However, since Popper had asked him to summarize the published paper, Einstein adds that, since both position and momentum of $B$ can be predicted with certainty, according to the infamous EPR \s{criterion of reality}, \emph{both} quantities must correspond to something that exists in physical reality. This situation, because of the \IR{}, finds no counterpart in the formalism of \qm{} \letteraeap{Einstein}{Popper}{11}{9}{1935}[19-130]. Although the argument was again meant to convey the incompleteness of the theory, it also seems to conflict with the spirit of the \IR{}. Indeed, \citet[244\fn{$^\ast$4}]{Popper1959} would later describe the EPR argument, as presented in this letter, as a weaker (no prediction for both position and momentum simultaneously) but correct (both position and momentum are simultaneously part of reality) version of his own failed ETP-like \TE{}.


\section*{Conclusion}
\label{sec:concl}

%Victor Weis~kopf to Popper, K August and 17 October 1936. Popper
%Archives (360. 21); Popper to Carnap. 5 September 1936. Carnap Colltction; Bohr's tt·stimonial. %Hayek Archives (44, 1).

%\citepp{Bohr1937}{Schlick1936}{Frank1936} \citep[see also][]{Strauss1936}

After the objection from Einstein, Popper’s scientific hero, he gave up. Thanks to \Weis's mediation \letterKPAp{\Weis}{Popper}{17}{10}{1936}[360-21], he had the opportunity to meet Bohr in Copenhagen in June 1936 \citep[see][]{Popper1936a}, where he was invited, at the last minute, to the Second International Congress for the Unity of Science. However, at that time he felt defeated. Only at the turn of the 1950s did Popper return to the philosophy of quantum theory \citep{DelSanto2019}---however, this time in the name of realism rather than determinism \citep{Howard2012}. The reflections on the topic written between 1950 and 1956 were meant to appear in a \emph{Postscript} to the English edition of \LdF, but their publication was ultimately delayed\footnote{See \cite{Popper1982}}. Still, in the footnotes and appendices added to the new edition (marked by starred numbers) Popper made no secret that his old \TE was \q{based upon a mistake}, as had been pointed out to him by \vW, Heisenberg, and Einstein: the past of an electron cannot be reconstructed without blurring its position \citep[217\fn{$^\ast$1}]{Popper1959}. In trying to evade this objection by devising ever more convoluted filter setups, the young Popper clearly failed to follow his own advice: in formulating a critical thought experiment, one should make only special assumptions that favor one’s opponents \citep[444--446]{Popper1959}.

Popper ultimately concedes that \vW\ was correct: \q{for the electric field perpendicular to the direction of a beam of electrons} used in Appendix~VI does not act as he expected: \q{for the width of the beam must be considerable if the electrons are to move parallel to the $x$-axis, and as a consequence, \myemph{their position before their entry into the field cannot be calculated} with precision after they have been deflected by the field} \citep[299\fn{$^\ast$1\me}]{Popper1959}. Moreover, Popper argues, \q{as Einstein shows} in his 1935 letter\footnote{Reproduced and translated in Appendix~$^\ast$xii of \citet{Popper1959}.}, the same can be said \q{for a filter acting upon a light quantum} \citep[299\fn{$^\ast$1}]{Popper1959}. Popper acknowledges that the conclusion is inescapable: \q{non-predictive measurements determine the path of a particle only \myemph{between} two measurements}, such as a measurement of momentum followed by one of position (or vice versa): \q{it is not possible, according to quantum theory, to project the path further back, \ie, to the region of time \myemph{before} the first of these measurements} \citep[242\fn{$^\ast$3\me}]{Popper1959}. As we have seen, Popper realized early on that, if this is the case, his \TE---and, more generally, ETP-like \TE{}s---\q{collapses} \citep[242\fn{$^\ast$3}]{Popper1959}.

However, in a long footnote attached to Appendix~VI, as well as in a new Appendix~$^\ast$XI, Popper seems to blame his \s{mistake} on the fact that he was misled by Heisenberg. According to Popper, Heisenberg \q{\emph{fails to establish that measurements of position and of momentum are symmetrical}} \citep[451]{Popper1959}. In Heisenberg’s famous $\gamma$-ray microscope thought experiment, position is determined using high-frequency light, which strongly \emph{disturbs} the electron’s momentum. Conversely, momentum can be determined, in principle, using low-frequency (long-wavelength) radiation via the Doppler effect so as to \emph{avoid} disturbance; but in this case, the position remains indeterminate \citep[451]{Popper1959}. Popper claims that he had extended this asymmetry to physical selections. Position selection via a slit disturbs the momentum (producing the diffraction pattern); momentum selection via a velocity filter only makes the position \emph{unknown}: \q{But I now believe that I was wrong in assuming that what holds for Heisenberg’s imaginary \s{observations} or \s{measurements} would also hold for my \s{selections}{}} \citep[299\fn{$^\ast$1}]{Popper1959}. Popper even argues that his \TE \q{can be used to point out an inconsistency in Heisenberg's discussion of the observation of an electron} \citep[299\fn{$^\ast$1}]{Popper1959}. 

Popper's recollection of the events strikes me as rather self-serving and appears to be contradicted by the unpublished textual evidence from that period. As we have seen, in his reply to Heisenberg, as well as in his unpublished second communication, \citet{Popper1934b} explicitly argues that the momentum determination should occur not through momentum \emph{measurement} via the Doppler effect, as \vW had proposed, because the latter does disturb the position; on the contrary, a momentum \emph{selection} via an electron spectrometer or a light filter does not. The latter leaves the given momentum unchanged, and therefore also the position---albeit the latter remains unknown. Popper was not misled by Heisenberg’s notion of \s{measurement}%
%
\footnoteh{At most, Popper was misled by Heisenberg’s ambiguous phrasing on p.~15 of his 1930 lecture notes; see above \cref{sec:past}}%
%
, but by the notion of \s{physical selection} he adopted%
%
\footnoteh{One might argue that in \citets{Heisenberg1927} analysis, the position measurement constitutes a genuine \s{measurement} (recording); by contrast, the momentum measurement is, in Popper's terminology, merely a \s{physical selection} of an ensemble characterized by a definite momentum distribution, while the position remains indeterminate. However, the idea that \s{undetermined} here means \s{unchanged, but unknown} is Popper's responsibility}. %
%
It was, indeed, Popper rather than Heisenberg who tried to exploit a non-existent asymmetry between position and momentum selections. \textcites[72\psq]{Popper1974}[23\fn{32}]{Popper1982} had some reason to claim that his notion of \s{physical selection} was similar in spirit to the modern concept of \s{preparation}, as outlined around the same time by \citepp{Margenau1936}{Margenau1937}. Yet it is clearly not equivalent to it. Paraphrasing Popper, one can say that the failure of his \TE \q{can be used to point out an inconsistency} in his notion of \s{physical selection}.


%Every accusation Popper levels against Heisenberg’s notion of \s{measurement} is, in fact, a confession about his own notion of \s{physical selection}.

Popper seems to have conceived of the momentum \s{filter} intuitively as analogous to a classical \s{sieve} that merely \emph{selects} an ensemble of particles that already possess the \emph{old} preassigned sharp momentum state, and simply leaves the position unknown\footnote{That momentum has a preassigned but unknown value is what Popper was supposed to prove, not to assume}. However, unlike such a classical device, the quantum-mechanical filter actually \emph{prepares} an ensemble of systems such that each member is in a \emph{new} momentum eigenstate, and thus completely spread position. The asymmetry between position and momentum preparations on which Popper relies is apparent. It arises from the fact that position is not an eigenstate of the Hamiltonian operator of a free particle, whereas momentum is. For this reason, momentum selection appears not to produce any analogous \s{disturbance} to that occurring in a position measurement; it merely seems to allow particles to pass through the filter with unchanged momentum. Yet the point Popper missed is that a momentum preparation still projects the system into a momentum eigenstate, rendering its position completely undetermined, since position is not an eigenstate of the momentum operator. Consequently, there is no way to combine information from a preparation and a measurement to reconstruct an individual trajectory before the momentum selection, as Popper had hoped. 

Popper did realize that his ETP-like arrangement was based on a \s{mistake}; however, he ultimately does not seem to have appreciated the deeper conceptual reason behind it. In the second edition of the \bt{Logic}, Popper complained that those \q{critics who rightly rejected the idea of this imaginary experiment appear to have believed that they had thereby also refuted the preceding analysis} \citep[239\fn{$^\ast$2}]{Popper1959}. However, Popper was adamant that this was not so. He not only rightly continued to maintain that preparation and measurement should be \emph{distinguished}, but also held on to the idea that, for this reason, the two can be \emph{uncoupled}. To his credit, Popper made it unmistakably clear that this was the central point of his argument. However, the repeated failure of his experimental setups should have suggested that this point does not stand. Far from being an additional or auxiliary assumption, the \CH lies at the core of \qm, which is ultimately a theory of \emph{transition} probabilities between prepared and measured states%
%
\footnoteh{Suppose that a system is prepared in the state $\psi(q)$, and that we inquire about the probability that, upon measurement, it will be found in another state $\phi(q)$. According to the Born rule, this probability is given by the integral $\int \phi^*(q)\,\psi(q)\,dq.$ The rule expresses the connection between the preparation of the state $\psi$ and the possible registration of the result represented by $\phi$.}. %
%
Popper's repeated attempts to refute the \s{preparation-measurement coupling hypothesis} inadvertently demonstrate its centrality to quantum mechanics: a measurement outcome is meaningful only relative to a prepared quantum state.

\begin{figure}
\centering
 \includegraphics[scale=0.5, trim = 0mm 0mm 0mm 0mm, clip]{1982PopperExperiment.png}

\caption{From \cite[28]{Popper1982}\label{fig:popperEPR}}
\end{figure}

%{Popper1982b}{Popper1985}{Popper1986}

Even though he acknowledged the problem that affected his setup, he remained convinced that one could \q{replace [his] invalid experiment} \citep[232\fn{$^{\ast\ast}$}]{Popper1972} of the ETP-type with a valid EPR-type experiment. After \textcites{Popper1967} returned to work on quantum physics, he outlined, around 1977 or 1978, a thought experiment of this kind, which he made public in the early 1980s \citep{Popper1982}. Popper was at the height of his fame, and his variant of the EPR argument sparked considerable public controversy \citep{DelSanto2018}. However, on closer inspection, even this much more famous \TE\ rests on a similar maneuver. Once again, Popper tries to exploit a perfect correlation (\ie, relative position) inherent in a jointly prepared state to bypass the preparation--measurement coupling. He starts from an entangled state with sharp relative position ($y_B - y_A = 0$) and sharp total momentum ($p_A + p_B = 0$) for two particles emitted from a common source $S$ (\cref{fig:popperEPR}). He argues that changing the preparation of system $A$ (narrowing $\Delta y_A$ via a slit) would change the measurement statistics of the distant system $B$ (increasing the spread $\Delta p_{y_B}$) \emph{without changing its own preparation}. Once again, the \TE\ fails for the same \s{mistake}: perfect correlations belong to the joint preparation of the composite system and cannot be used to circumvent the preparation--measurement coupling for one of the subsystems.

%Sudbery1985


%%

%Popper anticiate preparatio and measrem, was amognt htef  \q{the uncertainty principle restricts the degree of statistical homogeneity which it is possible to achieve in an ensemble of similarly prepared systems}, bit  past \p{from the data of both state preparation and measurement in the time interval between these two operations.} \citep[367]{Ballentine1970}. Howve,r before the perato  the coipl before the be avaeded. that o cannot before the preaptaiom. Teh couplin bewteen cannot be vaded
%
%%, $\Delta q_y \Delta p_y \approx 0$..... o that their relative position is fixed ($x_1 - x_2 = x_0$), while the total momentum is also fixed ($p_1 + p_2 = 0$
%
%%to obtain a predictive measurement that does not depend on a prior preparation, by exploiting correlations between two systems.
%
%%In both experiments, Popper failed to notice that the system he started with was already prepared — not in an individual eigenstate, but in an entangled state that defines its statistical properties.
%
%

\begin{figure}
\centering
 \includegraphics[scale=0.5, trim = 0mm 0mm 0mm 0mm, clip]{1982PopperExperiment}
\label{fig:1982popperexperiment}
\end{figure}


Popper famously imagined pairs of particles emitted in opposite directions from a  common $S$ source along the $x$-axis towards two slits $A$ and $B$, beyond which a semicircle of Geiger counters was placed (\cref{1982popperexperiment}). If one \emph{knows} the position $\Delta q_y \approx 0$ of particle $A$, one can also determine, by symmetry, the position $\Delta q_y \approx 0$ of particle $B$\footnote{If particle $A$ lands at $y = 1 \,\text{mm}$, then particle B will also be found at $y = -1 \,\text{mm}$}\todo{minus sign?}. Popper then proposed closing the slit for $A$, thereby causing diffraction, i.e. a very broad momentum distribution $\Delta p_y$ (all the counters on the right would be activated). Invoking the EPR reasoning, Popper argued that two alternatives were possible: (a) the Cophenhangen view: $B$’s momentum distribution must also spread, even though $B$ never encountered a slit (all the counters on the right would be activated). This would imply an action at a distance; (b) Popper’s view: $B$’s momentum does not spread, so that one could in principle predict both $B$’s position and momentum with arbitrary precision. This would amount to a violation of the \IR.
%
Once again, the (b) alternative favored by Popper is a \s{measurement} that does not require a previous \s{preparation}, thereby beating the \IR understood as \s{scatter relations}. However, Popper failed to recognize that in his setup the particle pair had been previously \emph{prepared} in a state with precisely correlated positions. Because of this, while the total momentum is sharp, their momenta are necessarily broadly spread from the outset.\footnote{It is hard not also to conclude that also in this case Popper committed \s{gross mistake}. Indeed, Popper's setup is nothing but a special case of the EPR paper example, where particles are in a state with perfectly equal positions: $\psi\left(x_A, x_B\right)=\delta\left(x_A-x_B\right)$. As in EPR, Fourier-transforming to the momentum representation gives $\tilde{\psi}\left(p_A, p_B\right)=\frac{1}{2 \pi \hbar} \int d x_A d x_B\, \delta(x_A - x_B)\, e^{-\frac{i}{\hbar}\left(p_A x_A+p_B x_B\right)} = \delta(p_A + p_B)$. The total momentum is sharp $\left(p_A+p_B=0\right)$, but each particle's momentum is completely undetermined from the outset}. Contrary to Popper's claim, orthodox quantum mechanics predicts a broad distribution of detection events on both sides of the apparatus, and thus the activation of all counters, \emph{even without closing the slit } \citep[sec.~11.3]{Ghirardi2007}. Narrowing the $A$'s position via a slit broadens $A$’s momentum distribution, but $B$’s momentum distribution is already broad, so it does not change (all the counters on the right would be activated, anyway). Only the joint correlations are modified, but this can only be established post-facto using classical signals\footnote{By observing the distribution of $B$ alone. You cannot tell if $A$ had one slit, two slits, or no slit just by looking locally. The difference appears only in the correlations: some positions of $B$ are more (or less) likely to coincide with certain positions of $A$, depending on whether there was one slit or a double slit at $A$. To extract this pattern, you must match $B$’s clicks with $A$’s clicks via classical communication}. This is exactly why entanglement does not allow faster-than-light influence.
%\end{comment}
%
%
%
%%(e.g., Redhead, Selleri, Sudbery, Kim & Shih) 
%
%
%%\delta\left(x_A-x_B\right) \propto \delta\left(p_A+p_B\right)$
%
%
%%https://arxiv.org/pdf/1507.02010


%
%%\appendix
%%\label{app:1}
%%\section{Argument}

\vW\ points out that, in order to determine the momentum of the particle from 
its position on the screen, one must know the free path length precisely, which 
requires the use of narrow diaphragms of width $d$. Yet, if the diaphragms are 
too narrow, diffraction appears, introducing an unavoidable spread in the 
transverse coordinate $y$. Thus, the spread in $y$ at the screen cannot be 
smaller than the slit opening:
\[
\Delta y \geq d.
\]
Since the deflection $EF = y$ is the measure of the momentum, the slit also 
produces a spread in transverse momentum $\Delta p_y$ due to diffraction, which 
in turn leads to a spread in vertical displacement after traveling the distance 
$L$:
\[
\Delta y \geq \frac{\Delta p_y}{mv} \, L.
\]
Narrowing the slit $d$ increases the momentum spread $\Delta p_y$, according to 
the basic uncertainty condition
\[
d \cdot \Delta p_y \geq h.
\]
Multiplying the two inequalities and inserting this condition gives
\[
(\Delta y)^2 \geq \frac{h}{mv} \, L. \tag{1}
\]
This shows that restricting the beam’s transverse position inevitably produces a 
corresponding uncertainty that enforces the Heisenberg principle. The figure 
illustrates this: the more one attempts to collimate the beam tightly at $D$, 
the more it spreads out by the time it reaches the detector at $E\!-\!F$.  

---

After establishing that diffraction at the slit enforces a minimal transverse 
uncertainty, \vW\ extends the analysis to the longitudinal motion of the 
particle, i.e.\ to $p_x$. Because the vertical displacement $y$ depends on the 
velocity $v$, the uncertainty in $y$ translates into an uncertainty in $v$. In 
a uniform force field, the deflection is proportional to $1/v^2$, since the 
vertical displacement follows the free-fall law 
\[
y = \tfrac{1}{2} a t^2,
\]
with the flight time given by $t = L/v$. Applying logarithmic differentiation 
and taking absolute values gives
\[
\frac{\Delta y}{y} = \frac{2 \Delta v}{v}. \tag{3}
\]

From this it follows, on the one hand, that there is an uncertainty in the 
longitudinal momentum,
\[
\Delta p_x = m \, \Delta v,
\]
and, on the other hand, that the determination of the $x$--coordinate,
\[
x = v t,
\]
is also uncertain. If $v$ is uncertain, then $t = L/v$ is likewise uncertain, 
so the total uncertainty in $x$ must include two contributions:
\[
\Delta x = v \Delta t + t \Delta v 
          = \frac{L}{v} \Delta v + \frac{L}{v} \Delta v 
          = \frac{2L}{v} \, \Delta v.
\]
Hence,
\[
\Delta p_x \, \Delta x = (m \Delta v)\!\left(\frac{2 L}{v} \Delta v\right) 
= \frac{2 m L}{v} \, (\Delta v)^2.
\]

---

Finally, combining relation (3) with the transverse inequality (1) yields
\[
\Delta p_x \, \Delta x \;\geq\; \frac{L^2}{y^2} \, h.
\]

Thus the uncertainty in velocity propagates into both momentum and longitudinal 
position, and together with diffraction effects it ensures that the Heisenberg 
uncertainty principle is preserved in both transverse and longitudinal 
directions.

\citetrackerfalse

%\printshorthands
\printbibliography

\end{document}
