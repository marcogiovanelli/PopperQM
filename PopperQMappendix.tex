\section{Argument}

\vW\ points out that, in order to determine the momentum of the particle from 
its position on the screen, one must know the free path length precisely, which 
requires the use of narrow diaphragms of width $d$. Yet, if the diaphragms are 
too narrow, diffraction appears, introducing an unavoidable spread in the 
transverse coordinate $y$. Thus, the spread in $y$ at the screen cannot be 
smaller than the slit opening:
\[
\Delta y \geq d.
\]
Since the deflection $EF = y$ is the measure of the momentum, the slit also 
produces a spread in transverse momentum $\Delta p_y$ due to diffraction, which 
in turn leads to a spread in vertical displacement after traveling the distance 
$L$:
\[
\Delta y \geq \frac{\Delta p_y}{mv} \, L.
\]
Narrowing the slit $d$ increases the momentum spread $\Delta p_y$, according to 
the basic uncertainty condition
\[
d \cdot \Delta p_y \geq h.
\]
Multiplying the two inequalities and inserting this condition gives
\[
(\Delta y)^2 \geq \frac{h}{mv} \, L. \tag{1}
\]
This shows that restricting the beam’s transverse position inevitably produces a 
corresponding uncertainty that enforces the Heisenberg principle. The figure 
illustrates this: the more one attempts to collimate the beam tightly at $D$, 
the more it spreads out by the time it reaches the detector at $E\!-\!F$.  

---

After establishing that diffraction at the slit enforces a minimal transverse 
uncertainty, \vW\ extends the analysis to the longitudinal motion of the 
particle, i.e.\ to $p_x$. Because the vertical displacement $y$ depends on the 
velocity $v$, the uncertainty in $y$ translates into an uncertainty in $v$. In 
a uniform force field, the deflection is proportional to $1/v^2$, since the 
vertical displacement follows the free-fall law 
\[
y = \tfrac{1}{2} a t^2,
\]
with the flight time given by $t = L/v$. Applying logarithmic differentiation 
and taking absolute values gives
\[
\frac{\Delta y}{y} = \frac{2 \Delta v}{v}. \tag{3}
\]

From this it follows, on the one hand, that there is an uncertainty in the 
longitudinal momentum,
\[
\Delta p_x = m \, \Delta v,
\]
and, on the other hand, that the determination of the $x$--coordinate,
\[
x = v t,
\]
is also uncertain. If $v$ is uncertain, then $t = L/v$ is likewise uncertain, 
so the total uncertainty in $x$ must include two contributions:
\[
\Delta x = v \Delta t + t \Delta v 
          = \frac{L}{v} \Delta v + \frac{L}{v} \Delta v 
          = \frac{2L}{v} \, \Delta v.
\]
Hence,
\[
\Delta p_x \, \Delta x = (m \Delta v)\!\left(\frac{2 L}{v} \Delta v\right) 
= \frac{2 m L}{v} \, (\Delta v)^2.
\]

---

Finally, combining relation (3) with the transverse inequality (1) yields
\[
\Delta p_x \, \Delta x \;\geq\; \frac{L^2}{y^2} \, h.
\]

Thus the uncertainty in velocity propagates into both momentum and longitudinal 
position, and together with diffraction effects it ensures that the Heisenberg 
uncertainty principle is preserved in both transverse and longitudinal 
directions.